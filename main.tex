\documentclass[a4paper,12pt,oneside]{report}
\usepackage{OvidiusFMI}
\usepackage{times}
\usepackage{graphicx}
\usepackage{hyperref}
\usepackage{color,xcolor}
\usepackage{amsmath}
\usepackage{framed}
\usepackage{indentfirst}
\usepackage{enumerate}
\usepackage[shortlabels]{enumitem}
\usepackage{listings}
\usepackage{amsmath,amsfonts,amssymb,amsthm,epsfig,epstopdf,url,array}
\usepackage{multicol,multirow}
\definecolor{code}{rgb}{0.97,0.97,0.97}
\lstdefinestyle{customc}{
  belowcaptionskip=1\baselineskip,
  backgroundcolor=\color{code},
  breaklines=true,
%  frame=L,
%  xleftmargin=\parindent,
  language=C,
  showstringspaces=false,
  morekeywords={bool,
  				 glutMainLoop, glutIdleFunc, glMatrixMode, glLoadIdentity, glPushMatrix, glPopMatrix, 
  				 glBegin, glEnd, glTranslatef, glRotatef, glScalef, glColor3f, glColor4f, glutSolidCube, glutWireCube, glutSolidSphere,
  				 glutWireSphere, glutSolidCone,glutSetWindowTitle,glutGet,glClear,glutSwapBuffers,glDepthFunc,
  				 glutWireCone, glutSolidTorus, glutWireTorus, glutSolidDodecahedron, glutWireDodecahedron,
  				 glutSolidOctahedron, glutWireOctahedron, glutSolidTetrahedron, glutWireTetrahedron, 
  				 glVertex3f,glVertex2f,glPointSize,
  				 glutSolidIcosahedron, glutWireIcosahedron, glutSolidTeapot, glutWireTeapot,glutReshapeFunc,
  				 glFlush, gluPerspective, glutPostRedisplay, glutInit, glutKeyboardFunc,glutKeyboardUpFunc,
  				 glutInitWindowSize, glutInitWindowPosition, glutInitDisplayMode, glutCreateWindow, glutDisplayFunc,glutPassiveMotionFunc,
  				 glClear,glTexCoord2f,
  				 glEnable, glDisable, glLightfv, glMaterialfv, glCullFace,glViewport,
  				 glFrontFace,glColor3ub, glShadeModel,
  				 glGenLists, glGetFloatv,glGentextures,glTexImage2D,glTexParameteri, free, glDeleteTextures,
  				 glLineStipple, glLineWidth, glBindTexture,glGenTextures,
  				 glNewList, glEndList, glCallList,
  				 glMap1f,glEvalCoord1f,glMapGrid1d,glEvalMesh1,glMap2f,glEvalCoord2f,glMapGrid2f,glEvalMesh2,
  				 gluBeginTrim, gluEndTrim, gluPwlCurve,glHint,
  				 GLUnurbsObj, gluBeginSurface, gluNurbsSurface, gluEndSurface, gluNewNurbsRenderer, 									gluNurbsProperty,gluQuadricNormals,
  				 glNormal3f,
  				 gluQuadricTexture,GLUquadricObj,gluSphere,
  				 glPolygonMode, glBlendFunc,glFogi,glFogiv,glFogfv,
  				 GLfloat, GLdouble, GLint,GLuint, GLushort,GLubyte, glRasterPos2f,
  				 gluBeginCurve, gluNurbsCurve, gluEndCurve,
  				 glOrtho, gluLookAt, glutBitmapCharacter, 
  				 glInitNames, glPushName, glLoadName, glSelectBuffer, glRenderMode,gluPickMatrix, glGetIntegerv, glutMouseFunc,glutMotionFunc,system,
  				 glPushAttrib, glPopAttrib, glMultMatrixf, sprintf, glClearStencil, glStencilFunc,glStencilOp,glStencilMask,glColorMask,glActiveStencilFaceEXT,fprintf},  
%   numbers=left,                    % where to put the line-numbers; possible values are (none, left, right)
  %numbersep=5pt,                   % how far the line-numbers are from the code
  %numberstyle=\tiny\color{code}, % the style that is used for the line-numbers
  basicstyle=\footnotesize\ttfamily,
  keywordstyle=\bfseries\color{green!40!black},
  commentstyle=\itshape\color{purple!40!black},
  identifierstyle=\color{blue},
  stringstyle=\color{orange},
}

\lstset{escapechar=@,style=customc}


\newtheorem{definition}{Defini\c tie}
\newtheorem{proposition}{Propozi\c tie}
\newtheorem{demonstration}{Demonstra\c tie}
\newtheorem{example}{Exemplu}
\newtheorem{theorem}{Teorem\u a}
\newtheorem{solve}{Rezolvare}
\newtheorem{remark}{Observa\c{t}ie}
\newtheorem{comentary}{Comentariu}
\newtheorem{corollary}{Corolar}
\newtheorem{lemma}{Lem\u{a}}
\newtheorem{problem}{Problema}
\renewcommand*{\proofname}{\rm\bf{Demonstra\c{t}ie}:}

\facultatea{Matematic\u a \c si Informatic\u a}
\specializarea{Informatic\u a}
\teza{Licen\c t\u a}
\titlu{Ecua\c tii integrale de tip Volterra \c{s}i Fredholm }

\coordonatorPrincipal{Cosma Lumini\c ta}
\autor{T\u anase Ramona Elena}
\data{2022}

\begin{document}
\maketitle

\pagenumbering{roman}
\tableofcontents

\newpage
\pagenumbering{arabic}

\chapter*{Abstract}

\^{I}n aceast\u{a} lucrare voi prezenta ecua\c{t}iile integrale Volterra \c{s}i Fredholm. \^{I}n matematic\u{a} , ecua\c{t}iile integrale Volterra sunt un tip special de ecua\c{t}ii integrale .Ele sunt \^{i}mp\u{a}r\c{t}ite \^{i}n dou\u{a} grupuri, denumite primul \c{s}i al doilea fel.\^{I}n teoria operatorilor \c{s}i \^{i}n teoria Fredholm , operatorii corespunz\u{a}tori sunt numi\c{t}i operatori Volterra . 

\^{I}n matematic\u{a} , teoria Fredholm este o teorie a ecua\c{t}iilor integrale . \^{I}n sensul cel mai restr\^{a}ns, teoria Fredholm se preocup\u{a} de solu\c{t}ia ecua\c{t}iei integrale Fredholm . \^{I}ntr-un sens mai larg, structura abstract\u{a} a teoriei lui Fredholm este dat\u{a} \^{i}n termenii teoriei spectrale a operatorilor Fredholm \c{s}i a nucleelor Fredholm pe spa\c{t}iul Hilbert .

The aim of this paper is to present the integral equations Volterra and Fredholm.In mathematics, Volterra integral equations are a special type of integral equation. They are divided into two groups, called the first and second kind. In operator theory and Fredholm theory, the corresponding operators are called Volterra operators. 

In mathematics, Fredholm theory is a theory of integral equations. In the narrowest sense, Fredholm's theory deals with the solution of the Fredholm integral equation. In a broader sense, the abstract structure of Fredholm's theory is given in terms of the spectral theory of Fredholm operators and Fredholm nuclei on the Hilbert space.

\chapter*{Introducere}

Teoria ecua\c{t}iilor integrale reprezint\u{a} un capitol important \^{i}n matematica aplicat\u{a}. Primele lucr\u{a}ri, av\^{a}nd ca tematic\u{a} ecua\c{t}iile integrale, au ap\u{a}rut \^{i}n secolul 19 ¸\c{s}i la \^{i}nceputul secolului 20, av\^{a}nd ca autori matematicieni renumi\c{t}i ca Niels Abel, Augustin Cauchy , Edouard Goursat , Maxime Bocher, David Hilbert , Vito Volterra ,Ivar Fredholm , Emile Picard , Traian Lalescu.

Lucrarea este structurat\u{a} \^{i}n 3 capitole. 

Primul capitol este unul introductiv \^{i}n care sunt prezentate no\c{t}iunile \c{s}i teoremele de baz\u{a} ce vor fi aplicate sau generalizate pe parcursul celorlalte capitole \c{s}i anume: Principiul contrac\c{t}iei al lui Banach, Operatorii liniari \c{s}i compac\c{t}i precum \c{s}i Cazul operatorilor compac\c{t}i, Principalele rezultate \c{s}i aproxim\u{a}rile identit\u{a}\c{t}ii Friedrichs.

Cel de al doilea capitol, cuprinde ecua\c{t}iile integrale de tip Volterra \c{s}i Fredholm. \^{I}n ecua\c{t}iile Volterra, limita superioar\u{a} de integrare este variabila x, \^{i}n timp ce \^{i}n ecua\c{t}iile Fredholm, limita superioar\u{a} de integrare este o constant\u{a} fix\u{a} . A\c{s}a-numitele ecua\c{t}ii de primul fel implic\u{a} doar func\c{t}ia necunoscut\u{a} \(\varphi \) \^{i}n interiorul integralei.

Printre cele mai importante aspecte abordate, putem aminti: Teorema de existen\c{t}\u{a} \c{s}i unicitate pentru care am prezentat trei demonstra\c{t}ii diferite, Alternativa lui Fredholm, Nucleul rezolvent \c{s}i Cazul nucleelor hermitiene.

Cel de al treilea capitol prezint\u{a} aplica\c{t}ii ale ecua\c{t}iilor Volterra \c{s}i Fredholm. Am calculat nucleele rezolvente ale ecua\c{t}iilor Volterra \c{s}i am gasit solu\c{t}iile corespunz\u{a}toare, am ar\u{a}tat c\u{a} alternativa lui Fredholm pentru ecua\c{t}ia corespunz\u{a}toare poate fi exprimat\u{a} ca o alternativ\u{a} echivalent\u{a} pentru  sistemul algebric.



%
%
%CAPITOLUL 1
%
%

\chapter{Preliminarii}

\nocite{morosanu}
\nocite{kreyszig}

\textbf{Teorema de existen\c{t}\u{a} a lui Peano }
\begin{theorem}
Fie a , b \(\in \left ( 0, \infty  \right ) , t_{0} \in \mathbb{R}, x_{0} \in \mathbb{R}^{k}\) ( pe \(\mathbb{R}^{k}\) vom considera norma \(\left \| v \right \| =  max _{1\leq i\leq k }\left | v_{i} \right |.\)) Fie D mul\c{t}imea
\begin{displaymath}
  D = \left \{ \left ( t,v \right )  \in \mathbb{R} \times \mathbb{R}^{k} ; \left | t - t_{_{0}} \right | \leq a , \left \| v - x_{0} \right \| \leq b\right \} \subset \mathbb{R}^{k + 1}
\end{displaymath}
\c{s}i fie \(f : D \rightarrow \mathbb{R}^{k+1}\) o func\c{t}ie continu\u{a}. Atunci, exist\u{a} o func\c{t}ie derivabil\u{a}, cu derivata continu\u{a} \(x : \left [ t_{0}-\delta ,t_{0}+\delta   \right ]\rightarrow \mathbb{R}^{k}\) care verific\u{a} ecua\c{t}ia diferen\c{t}ial\u{a}
  \begin{displaymath}
    {x}' \left ( t \right ) = f\left ( t,x\left ( t \right ) \right ), \forall t \in \left [ t_{0}-\delta ,t_{0}+\delta   \right ]
    \label{eq:1.1} \tag{1.1}
  \end{displaymath}
\c{s}i condi\c{t}ia  ini\c{t}ial\u{a}
\begin{displaymath}
  x\left ( t_{0} \right ) = x_{0} \label{eq:1.2} \tag{1.2}
\end{displaymath}
unde \(\delta = min\left ( a,\frac{b}{M} \right )\) cu \(M = sup \left \{\left \| f\left ( t,v \right ) \right \| ;\left ( t,v \right ) \in D \right \}.\) M se presupune a fi un num\u{a}r pozitiv, deoarece cazul \(M = 0 \Leftrightarrow f \equiv 0\) este trivial.
\end{theorem}

\section{Principiul contrac\c{t}iior al lui Banach}

Dac\u{a} au loc ipotezele teoremei de existen\c{t}a a lui Peano ( teorema 1) plus condi\c{t}ia ca $f$ sa fie Lipschitz in raport cu $x$, atunci problema Cauchy
\begin{displaymath}
  {x}'\left ( t \right ) = f\left ( t,x\left ( t \right ) \right ), x\left ( t_{0} \right ) = x_{0} \label{eq:1.3} \tag{1.3}
\end{displaymath}
are o solu\c{t}ie unic\u{a} \(x \in C^{1} \left ( I ; \mathbb{R}^{k} \right ),\) unde \(I = \left [ t_{0}-\delta ,t_{0}+\delta   \right ] ,\) cu \(\delta\) a\c{s}a cum este definit\u{a} \^{i}n teorema 1. Acest rezultat poate fi ob\c{t}inut \c{s}i prin aplicarea Principiului contrac\c{t}iilor al lui Banach. \^{I}nainte de a enun\c{t}a acest principiu, s\u{a} ar\u{a}t\u{a}m cum anume problema 1.3 poate fi redus\u{a} la o problem\u{a} de punct fix.

Problema 1.3 este echivalent\u{a} cu ecua\c{t}ia integral\u{a}
\begin{displaymath}
  x\left ( t \right ) = x_{0} + \int_{t_{0}}^{t}f\left ( s,x\left ( s \right ) \right ) ds. \label{eq:1.4} \tag{1.4}
\end{displaymath}
Not\u{a}m \(X = \left \{ v \in C \left ( I, \mathbb{R}^{k} \right ); \left \| v\left ( t \right ) -x_{0}\right \| \leq b, \in I\right \}.\)

Aceast\u{a} mul\c{t}ime este un spa\c{t}iu metric real din moment ce este o submul\c{t}ime \^{i}nchis\u{a} a spa\c{t}iului Banach  \(C \left ( I, \mathbb{R}^{k} \right )\) \^{i}nzestrat cu norma sup, notat \(\left \| \cdot  \right \|_{C}\), care induce metrica \(d\left ( u,v \right ) = \left \| u-v \right \|_{C}\).

Definim pe X operatorul T prin
\begin{displaymath}
  \left ( T_{v} \right )\left ( t \right ) = x_{0} + \int_{t_{0}}^{t}f\left ( s,v\left ( s \right ) \right ) ds, \forall v \in X.
\end{displaymath}
Prefer\u{a}m nota\c{t}ia \(T_{v}\) \^{i}n loc de \(T\left ( v \right ).\)

Se vede cu u\c{s}usin\c{t}\u{a} faptul c\u{a} \^{i}n ipotezele de mai sus \(T_{v} \in X\) pentru orice \(v \in X\) adic\u{a} \(T : X \rightarrow X.\) Ecua\c{t}ia 1.4 poate fi scris\u{a} simplu ca
\begin{displaymath}
  x = Tx, \label{eq:1.5} \tag{1.5}
\end{displaymath}
deci problema Cauchy de mai sus se reduce la rezolvarea ecua\c{t}iei 1.5 \^{i}n \(X\). Cu alte cuvinte, problema Cauchy 1.3 are o solu\c{t}ie unic\u{a} \(x\) definit\u{a} pe \(I\) dac\u{a} \c{s}i numai dac\u{a} \(T\) are un punct unic fix \(x\).

\begin{theorem}
(Principiul contrac\c{t}iilor al lui Banach ) Fie \(\left ( X,d \right )\) un spa\c{t}iu metric complet. Presupunem c\u{a}  \(T : X \rightarrow X\) este o contrac\c{t}ie, adic\u{a}, \(\exists \alpha  \in \left ( 0,1 \right ) \) astfel \^{i}nc\^{a}t
\[d\left ( Tx, Ty  \right ) \leq \alpha d\left ( x,y \right )\]

pentru orice \(x,y \in X.\) Atunci T are un punct fix unic ( adic\u{a}, \(\exists ! x^{\ast } \in X\) astfel \^{i}nc\^{a}t \(T x^{\ast }  = x^{\ast }).\)
\end{theorem}
\begin{proof}
Vom folosi metoda aproxim\u{a}rilor succesive.

Definim \(x_{n} = Tx_{n - 1}\) pentru \(n \in \mathbb{N}\) cu \(x_{0} \in X\) arbitrar. Vom avea, prin induc\c{t}ie
\begin{displaymath}
  d\left ( x_{n + 1}, x_{n} \right ) \leq  \alpha ^{n}d\left ( x_{1}, x_{0} \right ) = \alpha ^{n}d\left ( Tx_{0}, x_{0} \right ), \forall n \in \mathbb{N}. \label{eq:1.6} \tag{1.6}
\end{displaymath}
Acum, vom demonstra faptul c\u{a} \(\left (x_{n}  \right )\) este \c{s}ir Cauchy \^{i}n \(\left (X, d \right ):\)
\begin{displaymath}
  d\left ( x_{n + p} , x_{n}\right ) \leq d\left ( x_{n + p} , x_{n + p - 1}\right ) + d\left ( x_{n + p - 1} , x_{n + p -2}\right ) + \cdots +d\left ( x_{n + 1} , x_{n}\right )
\end{displaymath}
care din 1.6 este
\begin{equation}\nonumber
    \begin{split}
          & \leq \alpha ^{n}\left ( 1 + \alpha  + \cdots +\alpha ^{p-1} \right )d\left ( Tx_{0}, x_{0} \right ) \\ &  =\alpha  ^{n}\frac{1 -\alpha ^{p}}{1-\alpha }d\left ( Tx_{0}, x_{0} \right ) \\ &   \leq  \frac{\alpha ^{n}}{1-\alpha }\left ( Tx_{0}, x_{0} \right ).
    \end{split}
\end{equation}
Deci \c{s}irul este  Cauchy \^{i}n \(\left ( X,d \right )\) pentru c\u{a} \(\alpha ^{n}\rightarrow 0.\) Cum \(\left ( X,d \right )\) este complet, sirul  \(\left (x_{n}  \right )\) converge la un \(x^{\ast } \in X \Leftrightarrow d\left ( x_{n} ,x^{\ast }\right ) \rightarrow 0.\)

Dar,
\begin{equation} \nonumber
    \begin{split}
        d\left ( x^{\ast }, Tx^{\ast } \right ) &   \leq  d\left ( Tx^{\ast } , x_{n}\right ) + d\left ( x_{n},x^{\ast } \right ) \\ &  =  d\left ( Tx^{\ast } , x_{n-1}\right ) + d\left ( x_{n}, x^{\ast } \right ) \\ &  \leq \alpha d\left ( x^{\ast }, x_{n-1} \right ) + d\left ( x_{n} , x^{\ast }\right ),
    \end{split}
\end{equation}
care converge la 0 pentru \(n \rightarrow \infty,\) deci \(d\left ( x^{\ast } ,Tx^{\ast }\right ) \leq 0\) \c{s}i astfel \(x^{\ast }\) este un punct fix al lui T.

\^{I}n continuare vrem s\u{a} ar\u{a}t\u{a}m c\u{a} \(x^{\ast }\) este unic. Presupunem c\u{a} \(y^{\ast }\) este de asemenea un punct fix al lui \(T\), atunci \(d\left ( x^{\ast }, y^{\ast } \right ) = d\left ( Tx^{\ast }, Ty^{\ast } \right ) \leq \alpha d\left ( x^{\ast }, y^{\ast } \right )\) deci \(\left ( 1-\alpha  \right ) d\left ( x^{\ast }, y^{\ast } \right ) \leq\) 0 ceea ce implic\u{a} \(x^{\ast } = y^{\ast }. \)
\end{proof}
\begin{remark}
Presupunerea \(\alpha < 1\) \^{i}n teorema 2 este esen\c{t}ial\u{a} a\c{s}a cum ne arat\u{a} urm\u{a}torul contra-exemplu.

Dac\u{a} \(X = \mathbb{R}\) \c{s}i \(T : \mathbb{R}\rightarrow \mathbb{R}\) este dat de
\[Tx = x + \frac{\pi }{2} - \arctan x,
 \]
 atunci \(T\) nu are un punct fix deoarece \(\frac{\pi }{2} - \arctan x > 0, \forall x \in  \mathbb{R}.\)

 Pe de alt\u{a} parte, din teorema lui Lagrange, avem, pentru orice \(x,y \in \mathbb{R}, x\neq y, \)
\begin{equation} \nonumber
    \begin{split}
        \left | Tx - Ty \right |  & \leq \left | x - y - \arctan x + \arctan y \right | \\ &  = \left | x - y - \frac{x - y}{1 + z^{2}} \right |,
\\   \noindent \text{pentru un z \^{i}ntre x \c{s}i y}
       \\ & =\left | x - y  \right | \cdot \left ( 1 - \frac{1}{1 + z^{2}} \right ) \\ &  <  \left | x - y  \right |,
    \end{split}
\end{equation}
deci, chiar dac\u{a} inegalitatea este strict\u{a},  \( \alpha  = 1\) \c{s}i prin urmare T nu este o contrac\c{t}ie. Astfel, faptul c\u{a} acest \(T\) nu are un punct fix este un contraexemplu.
\end{remark}
\begin{remark}
Din demonstra\c{t}ia de mai sus observ\u{a}m faptul c\u{a}
\begin{displaymath}
  d\left ( x_{n}, x^{\ast } \right ) \leq  \frac{\alpha ^{n}}{1 - \alpha }d\left ( Tx_{0}, x_{0} \right )
\end{displaymath}
care ne ofer\u{a} o aproximare a lui \(x^{\ast }.\)
\end{remark}
\begin{remark}
Presupunem c\u{a} \(T^{k} = \underbrace {T\circ \cdots \circ T}_{k} , k\geq 2,\) este o contrac\c{t}ie, atunci exist\u{a} un punct unic pentru \(T\).
\end{remark}
\begin{proof}
Un punct fix al lui \(T\) este \^{i}n mod evident un punct fix al lui \(T^{k}.\) Invers, dac\u{a} \(x^{\ast }\) este un punct fix al lui \(T^{k}\) atunci \(Tx^{\ast } = T^{k+1}x^{\ast } = T^{k}\left ( Tx^{\ast } \right ),\) deci ambele \(x^{\ast } \)\c{s}i \(Tx^{\ast }\) sunt puncte fixe ale lui \(T^{k}, \) ca urmare \(Tx^{\ast } =x^{\ast }.\)
\end{proof}

\section{Operatori liniari \c{s}i compac\c{t}i}

Dac\u{a} X, Y sunt spa\c{t}ii normate \c{s}i  $A: X \rightarrow Y$ este un operator liniar, atunci
A se nume\c{s}te complet continuu sau compact dac\u{a} A duce mul\c{t}imi m\u{a}rginite din X \^{i}n submul\c{t}imi relativ compacte \^{i}n Y.

Un operator complet continuu  este evident continuu.
Vom nota
\[
K(X, Y ) = \{A\in L (X, Y); A~~ \mbox{compact}~~ \}.
\]
Evident $K(X, Y )$ este un subspa\c{t}iu liniar al lui $L(X, Y ).$ Mai mult, avem

\begin{theorem}
Dac\u{a} \(X\) este un spa\c{t}iu normat \c{s}i \(Y\) este un spa\c{t}iu Banach, atunci \(K\left ( X,Y \right )\) este un subspa\c{t}iu liniar \^{i}nchis al lui \(L\left ( X,Y \right ),\) adic\u{a} \(K\left ( X,Y \right )\) este spa\c{t}iu Banach \^{i}n raport cu norma operatorial\u{a}.
\end{theorem}
\begin{proof}
Vom nota similar, cu \(\left \| \cdot  \right \| \) cele trei norme ale spa\c{t}iilor \(X, Y\) \c{s}i \(L\left ( X,Y \right )\). Fie \(\left ( A_{n} \right )\) un \c{s}ir \^{i}n \(L\left ( X,Y \right ),\) care converge la \(A \in L\left ( X,Y \right )\), adic\u{a}, \(\left \| A_{n} - A\right \| \rightarrow 0.\)

Astfel, pentru \(\varepsilon > 0\) exist\u{a}
\(m \in \mathbb{N}\) suficient de mare astfel \^{i}nc\^{a}t
\begin{displaymath}
  \left \| A_{m} - A\right \| < \frac{\varepsilon }{3r}. \label{eq:1.7} \tag{1.7}
\end{displaymath}
Fie \(\left ( x_{n} \right )\) un \c{s}ir din \(B\left ( 0,r \right ) \subset X,\) unde \(r> 0\) este arbitrat fixat. Deoarece \(A_{m}\) este compact, exist\u{a} un sub\c{s}ir al lui \(\left ( x_{n} \right ),\) notat \(\left ( x_{n_{k}} \right ) _{k\geq 1},\) astfel \^{i}nc\^{a}t \(\left ( Ax_{n_{k}} \right )_{k\geq 1}\) este convergent, deci Cauchy. Astfel, pentru orice \(\varepsilon > 0,\) exist\u{a} \(N \in \mathbb{N} \) astfel \^{i}nc\^{a}t
\begin{displaymath}
  \left \| A_{m} x_{n_{k}} - A_{m} x_{n_{j}}\right \| < \frac{\varepsilon }{3}, \forall k,j > N  \label{eq:1.8} \tag{1.8}
\end{displaymath}
Folosind 1.7 \c{s}i 1.8 deducem c\u{a}
\begin{equation} \nonumber
    \begin{split}
      \left \| A x_{n_{k}} - A x_{n_{j}}\right \| &    \leq \left \| A x_{n_{k}} - A_{m} x_{n_{k}}\right \| + \left \| A_{m} x_{n_{k}} - A_{m} x_{n_{j}}\right \| +\left \| A_{m} x_{n_{k}} - A x_{n_{j}}\right \|  \\ & \leq \left \| A - A_{m} \right \|\cdot \left \| A_{m} x_{n_{k}} - A_{m} x_{n_{j}}\right \| + \left \| A_{m} - A\right \| \cdot \left \| x_{n_{j}} \right \| \\ &  < r\cdot \frac{\varepsilon }{3r} + \frac{\varepsilon }{3} + r\cdot \frac{\varepsilon }{3r}  \\ & = \varepsilon
    \end{split}
\end{equation}
cu alte cuvinte, \(\left ( Ax_{n_{k}} \right )\) este \c{s}ir Cauchy, prin urmare converge, deci \(A \in K\left ( X,Y \right ). \)
\end{proof}

\section{Teorema lui Fredholm}

Let $(H, (\cdot, \cdot), \parallel\cdot\parallel)$ un spatiu Hilbert.
Not\u{a}m cu \(K\left ( H \right ):= K\left ( H,H \right )\) spa\c{t}iul operatorilor liniari \c{s}i compac\c{t}i din \(H\) \^{i}n H. Acesta este un subspa\c{t}iu \^{i}nchis al lui \(L\left ( H \right ):= L\left ( H,H \right )\), \^{i}n raport cu norma operatorial\u{a} (din torema 3), prin urmare \(K\left ( H \right )\) este un spa\c{t}iu Banach \^{i}n raport cu aceast\u{a} norm\u{a}.

\begin{theorem}
Dac\u{a} \(\left ( H, \left ( \cdot ,\cdot  \right ),\left \| \cdot  \right \| \right )\) este un spa\c{t}iu Hilbert \c{s}i \(A \in K\left ( H \right )\), atunci spa\c{t}iul nul al lui \(I – A,\) notat \({ \mathcal{N}} = N\left ( I - A \right )\), este un subspa\c{t}iu  finit dimensional al lui \(H\), unde \(I\) operatorul identitate al lui \(H. \)
\end{theorem}
\begin{proof}
Evident, \({ \mathcal{N}}\) este un subspa\c{t}iu liniar \^{i}nchis al lui \(\left ( H,\left \| \cdot  \right \| \right ).\) Fie \(Q\) o submul\c{t}ime m\u{a}rginit\u{a} a lui \({ \mathcal{N}}\). Cum \(A\) este compact \c{s}i \(Q = AQ\) deducem faptul c\u{a} \(Q\) este relativ compact \^{i}n \(\left ( { \mathcal{N}} ,\left \| \cdot  \right \| \right ).\) Ca urmare, cum orice mul\c{t}ime \^{i}nchis\u{a} \c{s}i m\u{a}rginit\u{a} din $\mathcal{N}$ este compact\u{a}, deducem c\u{a} $\mathcal{N}$ este un subspa\c{t}iu  finit dimensional al lui \(H\).
\end{proof}
\begin{theorem}(Schauder)
Dac\u{a} \(\left ( H, \left ( \cdot ,\cdot  \right ),\left \| \cdot  \right \| \right )\) este un spa\c{t}iu Hilbert \c{s}i \(A \in K\left ( H \right )\) atunci \(A^{\ast} \in K\left ( H \right )\).
\end{theorem}

 (Am notat cu $A^{\ast}:H\rightarrow H$ operatorul adjunct al lui A.)
\begin{proof}
 Fie $r > 0$ arbitrar fixat. Deoarece  $A^{\ast } \in L(H),$ mul\c{t}imea\(A^{\ast}B\left ( 0,r \right )\) este m\u{a}rginit\u{a}:
 \[\left \| x \right \| < r \Rightarrow \left \| A^{\ast }x \right \|\leq r\left \| A^{\ast } \right \|.\]
 Cum A este compact, rezult\u{a} c\u{a} pentru orice \c{s}ir \(\left ( x_{n} \right )_{n\geq 1}\), din \(B\left ( 0,r \right )\), \c{s}irul \(\left ( \left ( A\circ A^{\ast } \right ) x_{n}\right )_{n\geq 1}\) are un sub\c{s}ir convergent, \(\left ( \left ( A\circ A^{\ast } \right ) x_{k}\right )_{k\geq 1}.\) Avem de asemenea
\begin{equation} \nonumber
    \begin{split}
     \left \| A^{\ast } x_{n_{k}} - A^{\ast }x_{n_{j}}\right \|^{2}  & = \left ( A^{\ast }\left ( x_{n_{k}} - x_{n_{j}} \right ),A^{\ast }\left ( x_{n_{k}} - x_{n_{j}} \right )  \right )  \\ & = \left ( x_{n_{k}} - x_{n_{j}} \right ), A \left ( A^{\ast }\left ( x_{n_{k}} - x_{n_{j}} \right ) \right ) \\ & \leq 2r\left \| \left (A\circ A^{\ast }  \right ) x_{n_{k}} - \left ( A \circ A^{\ast } \right ) \right \|,
    \end{split}
\end{equation}
deci \(\left ( A^{\ast } x_{n_{k}}\right )_{k\geq 1}\) este convergent.
\end{proof}
\begin{remark}
Fie \(A \in L\left ( H \right ).\)  Atunci \(A\) este compact dac\u{a} \c{s}i numai dac\u{a} \(A^{\ast }\) este compact. Acest lucru rezult\u{a} din Teorema lui Schauder de mai sus \c{s}i \(\left ( A^{\ast } \right )^{\ast } = A. \)
\end{remark}
\begin{remark}
Dac\u{a} \(A, B \in L\left ( H \right )\) \c{s}i cel pu\c{t}in unul este compact, atunci \(A \circ B\) este compact de asemenea.

\noindent Continu\u{a}m cu un rezultat important, datorat lui Fredholm, care ne ofer\u{a} o condi\c{t}ie necesar\u{a} \c{s}i suficient\u{a} pentru ca o ecua\c{t}ie operatorial\u{a} ce e dat\u{a} de un operator liniar \c{s}i compact s\u{a} admit\u{a} solu\c{t}ie.
\end{remark}
\begin{theorem}
(Fredholm) Fie \(\left ( H, \left ( \cdot ,\cdot  \right ), \left \| \cdot  \right \| \right )\) un spa\c{t}iu Hilbert \c{s}i fie \(A \in K\left ( H \right ).\) Ecua\c{t}ia \(x - A^{\ast }x = f\) are solu\c{t}ie dac\u{a} \c{s}i numai dac\u{a} \(f \in { \mathcal{N}} ^{\perp }\) unde \({ \mathcal{N}} = N\left ( I - A \right ). \)
\end{theorem}
\begin{corollary}
Dac\u{a} \(\left ( H, \left ( \cdot ,\cdot  \right ), \left \| \cdot  \right \| \right )\) este un spa\c{t}iu Hilbert \c{s}i \(A \in K\left ( H \right )\) , atunci ecua\c{t}ia \(x - Ax = f\) are solu\c{t}ie dac\u{a} \c{s}i numai dac\u{a} \(f \in \left ( N\left ( I - A^{\ast } \right ) \right )^{\perp }. \)
\end{corollary}
Este consecin\c{t}\u{a} imediat\u{a} a teoremei lui Fredholm aplicat\u{a} pentru $A^{\ast }.$
\begin{lemma}
Fie \(\left ( H, \left ( \cdot ,\cdot  \right ), \left \| \cdot  \right \| \right )\) un spa\c{t}iu Hilbert \c{s}i fie  \(A \in K\left ( H \right ).\) Atunci exist\u{a} o constant\u{a} \(C > 0\) astfel \^{i}nc\^{a}t
\begin{displaymath}
   C\left \| x \right \| \leq \left \| \left ( I - A \right )x \right \|, \forall x \in { \mathcal{N}}^{\perp } \label{eq:1.9} \tag{
   1.9}
\end{displaymath}
unde  \({ \mathcal{N}} = N \left ( I - A \right ).\)
\end{lemma}
\begin{proof}
Presupunem, prin absurd c\u{a} 1.9 nu este adev\u{a}rat\u{a}, adic\u{a} pentru orice \(n \in \mathbb{N} \), exist\u{a}  \(x_{n} \in { \mathcal{N}}^{\perp }\) astfel \^{i}nc\^{a}t \(\left \| x_{n} \right \| = 1\) \c{s}i
\begin{displaymath}
  \left \| \left ( I - A \right )x_{n} \right \| < \frac{1}{n}.
\end{displaymath}
Prin urmare,
\begin{displaymath}
  x_{n} - Ax_{n} \rightarrow 0. \label{eq:1.10} \tag{1.10}
\end{displaymath}
Cum \(A\) este compact exist\u{a} un sub\c{s}ir al lui \(\left (x_{n}  \right )_{n\geq 1} \) notat \(\left (x_{n_{k}}  \right )_{k\geq 1}\) astfel \^{i}nc\^{a}t $ \big(A(x_{n_{k}})\big)_{k\geq 1}$ este convergent. Din 1.10 deducem c\u{a} \(\left (x_{n_{k}}  \right )_{k\geq 1}\) este de asemenea convergent  \c{s}i limita lui \(x \in { \mathcal{N}}^{\perp }.\) Folosind din nou 1.10, observ\u{a}m faptul c\u{a} \(x – Ax = 0,\) adic\u{a} \(x \in { \mathcal{N}} .\) Din moment ce \({ \mathcal{N}} \cap { \mathcal{N}}^{\perp } = \left \{ 0 \right \}\) avem \(x = 0,\) ceea ce contrazice \(\left \| x_{n} \right \| = 1, \forall n\geq 1. \)
\end{proof}
\begin{proof} \textbf{Teorema lui Fredholm}

\noindent \textbf{Necesitatea}:  Presupunem c\u{a} ecua\c{t}ia \(x-A^{\ast }x = f\) are o solu\c{t}ie \(x\in H.\) Atunci, pentru orice \(y\in { \mathcal{N}}\), vom avea
\begin{equation}\nonumber
    \begin{split}
      \left ( f,y \right ) &    = \left ( x,y \right ) - \left ( A^{\ast }x,y \right ) \\ &  = \left ( x,y \right ) - \left ( x,Ay \right ) \\ &  = \left ( x, \underbrace{ \left ( I-A \right )y}_{=0} \right )  \\ & = 0.
    \end{split}
\end{equation}
Prin urmare, \(f \in {\mathcal{N}} ^{\perp }. \)

\noindent \textbf{Suficien\c{t}a}: Presupunem \(f \in {\mathcal{N}} ^{\perp }.\) Cum \({\mathcal{N}} ^{\perp }\) este un subspa\c{t}iu \^{i}nchis al lui \( \left ( H, \left ( \cdot ,\cdot  \right ), \left \| \cdot  \right \| \right ), {\mathcal{N}} ^{\perp }\) este un spa\c{t}iu Hilbert cu acela\c{s}i produs scalar \c{s}i norm\u{a}. Conform lemei 1, \(\left \| \cdot  \right \|\) este echivalent\u{a} cu norma definit\u{a} de produsul scalar
\begin{displaymath}
  \left \langle x,y \right \rangle = \left ( Tx,Ty \right ), \forall x,y \in {\mathcal{N}} ^{\perp },
\end{displaymath}
unde \(T=I-A\). Din moment ce func\c{t}ionala \(x \mapsto \left ( x,f \right )\) este liniar\u{a} \c{s}i continu\u{a} pe \({\mathcal{N}} ^{\perp }\), rezult\u{a} din Teorema de Reprezentare a lui Riesz faptul c\u{a} exist\u{a} \(x_{f }\in{\mathcal{N}} ^{\perp }\) astfel \^{i}nc\^{a}t
\begin{displaymath}
  \left ( x,f \right ) = \underbrace{\left \langle x,x_{f} \right \rangle}_{= \left ( Tx,Tx_{f} \right )}, \forall x \in {\mathcal{N}} ^{\perp } \label{eq1.11} \tag{1.11}
\end{displaymath}
De fapt, 1.11 are loc pentru orice \(x \in H\) deoarece \(x = {x}' + {x}''\), cu \({x}' \in {\mathcal{N}}, {x}'' \in {\mathcal{N}}^{\perp }\). Not\^{a}nd \(\tilde{x} = Tx_{f}\), putem scrie
\begin{displaymath}
  \underbrace{\left ( Tx,\tilde{x} \right )}_{=\left ( x,\tilde{x} -A^{\ast }\tilde{x}\right )} = \left ( x,f \right ), \forall x \in H,
\end{displaymath}
deci
\begin{displaymath}
  \tilde{x} - A^{\ast }\tilde{x} = f.
\end{displaymath}
\end{proof}
Urm\u{a}torul rezultat ofer\u{a} c\^{a}teva informa\c{t}ii care completeaz\u{a} teorema 6.
\begin{theorem}
Fie \(\left ( H,\left ( \cdot ,\cdot  \right ),\left \| \cdot  \right \| \right )\) un spa\c{t}iu Hilbert \c{s}i fie \(A \in K\left ( H \right )\). Atunci,
\begin{displaymath}
  R\left ( I-A \right ) = H \Leftrightarrow {\mathcal{N}} = \left \{ 0 \right \}\Leftrightarrow {\mathcal{N}}^{\ast } = \left \{ 0 \right \}\Leftrightarrow R \left ( I - A^{\ast } \right ) = H,
\end{displaymath}
unde \({\mathcal{N}} =  N\left ( I - A \right ), {\mathcal{N}} ^{\ast } = N\left ( I - A^{\ast } \right ) \) \c{s}i \(R \left ( I - A\right ) , R \left ( I - A^{\ast } \right )\) reprezint\u{a} imaginile operatorilor \(\left ( I - A \right ) , \left ( I - A^{\ast } \right ). \)
\end{theorem}
\begin{proof}
\c{T}in\^{a}nd cont de teorema 6 \c{s}i corolarul 1, este suficient s\u{a} demonstr\u{a}m c\u{a}
\begin{displaymath}
  R\left ( I - A \right ) =H \Leftrightarrow  R\left ( I - A^{\ast } \right ) =H\label{eq:1.12} \tag{1.12}
\end{displaymath}
Presupunem c\u{a} \(R\left ( I - A \right ) =H\).

S\u{a} demonstr\u{a}m c\u{a} \({\mathcal{N}} = \left \{ 0 \right \}\). Presupunem prin absurd c\u{a} \({\mathcal{N}} \neq  \left \{ 0 \right \}\), adic\u{a} exist\u{a} un \(x_{0 } \in {\mathcal{N}}, x_{0}\neq 0\). Deoarece \(R\left ( I-A \right ) = H\) putem construi un \c{s}ir \(\left (x_{n}  \right )_{n\geq 1}\) \^{i}n \(D\left ( A \right )\) astfel \^{i}nc\^{a}t
\begin{displaymath}
  Tx_{n} = x_{n-1}, \forall n\geq 1
\end{displaymath}
unde \(T:= I - A\). Vom avea
\begin{displaymath}
  T^{n}x_{n} = x_{0}\neq 0
\end{displaymath}
\c{s}i
\begin{displaymath}
  T^{n+1}x_{n} = 0,
\end{displaymath}
unde \(T^{k}:= T\circ T\circ \cdots \circ T\). Deci, not\^{a}nd \(H_{n} = N\left ( T^{n} \right )\), vom avea c\u{a} \(H_{n}\) este un subspa\c{t}iu liniar propriu al lui \(H_{n+1}\), pentru orice \(n \in \mathbb{N}\). Conform teoremei 4, orice \(H_{n}\) este un spa\c{t}iu finit dimensional, deci \^{i}nchis, deoarece
\begin{displaymath}
  T^{n} = \left ( I - A \right )^{n} = I - \underbrace{\sum_{k=1}^{n}\begin{pmatrix}
n\\
k
\end{pmatrix}\left ( -1 \right )^{k+1}A^{k}} _{\text{operator compact.}}
\end{displaymath}
Exist\u{a} astfel un \c{s}ir \(\left ( u_{n} \right )_{n\geq 1}\) astfel \^{i}nc\^{a}t
\begin{displaymath}
  u_{n} \in H_{n+1}, \left \| u_{n} \right \| = 1, \left \| u_{n}-u \right \|\geq \frac{1}{2}, \forall u \in H_{n}.
\end{displaymath}
Cum \(1\leq m< n\)
\begin{displaymath}
  T^{n}\left ( Tu_{n} +Au_{m}\right ) = T^{n+1}u_{n} + AT^{n}u_{m} = 0,
\end{displaymath}
avem \(Tu_{n} +Au_{m} \in H_{n} \) \c{s}i
\begin{displaymath}
  \left \| Au_{n}-Au_{m} \right \| = \left \| u_{n}-\left ( Tu_{n}  + Au_{m}\right ) \right \|\geq \frac{1}{2}.
\end{displaymath}
Astfel, \c{s}irul \(\left ( Au_{n} \right )_{n\geq 1}\) nu poate avea sub\c{s}iruri Cauchy. Acest lucru contrazice faptul c\u{a} A este compact \^{i}mreun\u{a} cu \(\left \| u_{n} \right \| = 1\) pentru orice \(n\geq 1\). Prin urmare \({\mathcal{N}} = \left \{ 0 \right \}\), care (conform teoremei 6) implic\u{a} faptul c\u{a} \(R\left ( I - A^{\ast } \right ) = H\). Astfel am demonstrat implica\c{t}ia
\begin{displaymath}
  R\left ( I - A \right ) = H \Rightarrow R\left ( I - A^{\ast } \right ) = H
\end{displaymath}
Implica\c{t}ia invers\u{a} rezult\u{a} din \^{i}nlocuirea lui \(A\) cu \(A^{\ast }\).
\end{proof}
\begin{remark}
Din corolarul 1 \c{s}i teorema 7 deducem faptul c\u{a} dac\u{a} ecua\c{t}ia \(x-Ax = f\) are o solu\c{t}ie \(u_{f}\) pentru orice \(f \in H\) atunci \(u_{f}\) este unic\u{a}. Deci,  putem formula \textbf{alternativ\u{a} a lui Fredholm} pentru ecua\c{t}ia \( a - Ax = f\) cu \(A \in K\left ( H \right )\) \c{s}i anume c\u{a} are loc una din urm\u{a}toarele afirma\c{t}ii:
\begin{itemize}
   \item pentru orice \(f \in H\), ecua\c{t}ia \(x - Ax = f\) are o solu\c{t}ie unic\u{a} (echivalent, \(N\left ( I - A \right ) = \left \{ 0 \right \}\) );
   \item \(N\left ( I - A \right ) \neq  \left \{ 0 \right \}\), caz \^{i}n care ecua\c{t}ia \(x - Ax = f\) are solu\c{t}ie dac\u{a} \c{s}i numai dac\u{a} \(f\perp N\left ( I- A^{\ast } \right )\).
 \end{itemize}
\end{remark}

\section{Teorema Hilbert - Schmidt}

\begin{theorem}
Fie \(\left ( X, \left \| \cdot  \right \| \right )\) un spa\c{t}iu normat \c{s}i fie \(A \in K\left ( X \right )\). Atunci A are o mul\c{t}ime num\u{a}rabil\u{a} de valori proprii \c{s}i singurul punct de acumulare posibil al mul\c{t}imii de valori proprii este \( \lambda = 0 \). Mai mult, pentru orice valoare proprie \( \lambda \neq 0\), \(\dim N\left ( \lambda I - A \right )< \infty \).
\end{theorem}
\begin{proof}
Demonstra\c{t}ia este trivial\u{a} dac\u{a} X este finit dimensional, s\u{a} presupunem deci c\u{a} X este infinit dimensional. Pentru a demonstra prima afirma\c{t}ie a teoremei, este suficient s\u{a} ar\u{a}t\u{a}m c\u{a} pentru orice \(r> 0\) mul\c{t}imea \(\left \{ \lambda \in \mathbb{K};\left | \lambda  \right |\geq r \right \}\) con\c{t}ine un num\u{a}r finit de valori proprii.

Presupunem c\u{a} prin absurd c\u{a} nu este a\c{s}a, adic\u{a} exist\u{a} \(r_{0}> 0\) \c{s}i un num\u{a}r infinit de valori proprii \(\lambda _{1}, \lambda _{2},...\) astfel \^{i}nc\^{a}t \(\left | \lambda _{n} \right | \geq r_{0}, \forall n\geq 1\). Atunci exist\u{a} un \c{s}ir \(u_{n} \in X -\left \{ 0 \right \}\) astfel \^{i}nc\^{a}t \(Au_{n} = \lambda _{n}u_{n}, \forall n\geq 1\) \c{s}i putem presupune c\u{a} \(\left \| u_{n} \right \| = 1, \forall n\geq 1\). Deoarece \(\lambda _{n}\) sunt distincte dou\u{a} c\^{a}te dou\u{a}, \(B_{n} = \left \{ u_{1},u_{2},\cdots,u_{n} \right \}\) sunt sisteme liniar independente. Not\u{a}m \(X_{n} = \ SpanB_{n} , n=1,2,...\).

Exist\u{a} astfel \(y_{n} \in X_{n} - X_{n-1}\) astfel \^{i}nc\^{a}t \(\left \| y_{n} \right \| = 1, \forall n\geq 2\) \c{s}i
\begin{displaymath}
  \left \| y_{n}-v \right \|\geq \frac{1}{2}, \forall v \in X_{n-1}, n\geq 2 \Rightarrow \left \| y_{n} - y_{m} \right \|\geq \frac{1}{2}, \forall n \neq m.
\end{displaymath}
Astfel,\(\left ( y_{n} \right )\) nu are sub\c{s}iruri Cauchy.

Pe de alt\u{a} parte, presupun\^{a}nd c\u{a} \(1\leq m< n\), vom avea
\begin{equation} \nonumber
    \begin{split}
        Ay_{n} - Ay_{m}  & = \underbrace{\lambda _{n}y_{n}}_{\in X_{n}-X_{n-1}} - \underbrace{\lambda _{n}y_{m}}_{\in X_{m}}+\underbrace{\left ( Ay_{n} - \lambda _{n}y_{n}\right )}_{\in X_{n-1}} - \underbrace{\left ( Ay_{m}  - \lambda _{m}y_{m}\right )}_{\in X_{m}\subset X_{n-1}} \\ & \lambda _{n}y_{n} - v_{mn}
    \end{split}
\end{equation}
cu \(v_{mn} \in X_{n-1}\), deoarece
\begin{equation} \nonumber
    \begin{split}
        Ay_{n} - \lambda _{n}y_{n}  & = A \left ( \sum_{i =1 }^{n} \alpha _{i}^{n}u_{i} \right ) - \lambda _{n}\left ( \sum_{i = 1}^{n} \alpha _{i}^{n}u_{i} \right ) \\ & = \left ( \sum_{i =1 }^{n} \alpha _{i}^{n}\lambda _{i}u_{i} \right ) - \lambda _{n}\left ( \sum_{i = 1}^{n} \alpha _{i}^{n}u_{i} \right ) \\ & = \sum_{i =1 }^{n} \alpha _{i}^{n}\left ( \lambda _{i} - \lambda _{n} \right )u_{i} \\ & = \sum_{i =1 }^{n-1} \alpha _{i}^{n}\left ( \lambda _{i} - \lambda _{n} \right )u_{i}
    \end{split}
\end{equation}
care apar\c{t}ine lui \(X_{n-1}\). Deci, vom avea
\begin{equation} \nonumber
    \begin{split}
        \left \| Ay_{n} - Ay_{m} \right \| & = \left \| \lambda_{n}y_{n} - v_{nm} \right \| \\ & = \left | \lambda _{n} \right |\cdot \left \| y_{n} - \lambda _{n}^{-1}v_{nm} \right \| \\ & \geq  r_{0} \left \| y_{n}- \lambda _{n}^{-1}v_{nm} \right \| \\ & \geq  \frac{r_{0}}{2},
    \end{split}
\end{equation}
deci \(\left ( Ay_{n} \right )\) nu are sub\c{s}iruri Cauchy. Dar \(A\) este compact \c{s}i \(\left \| y_{n} \right \| = 1, \forall n\geq 1\) deci \(\left ( Ay_{n} \right )\) trebuie s\u{a} aib\u{a} un sub\c{s}ir convergent. Aceast\u{a} contradic\c{t}ie arat\u{a} c\u{a} \(\left \{ \lambda \in \mathbb{K}; \left | \lambda  \right | \geq  r \right \}\) con\c{t}ine un num\u{a}r finit de valorii proprii ale lui A pentru orice \(r > 0\) , a\c{s}a cum am afirmat.
\end{proof}
\begin{proposition}
Fie \(\left ( H, \left ( \cdot ,\cdot  \right ), \left \| \cdot  \right \| \right )\)  un spa\c{t}iu Hilbert \c{s}i fie $A$ un operator simetric pe $H$. Atunci,
\begin{enumerate}
    \item orice valoare proprie a lui A este real\u{a}, chiar dac\u{a} \(\mathbb{K}=\mathbb{C}\)
    \item oricare doi vectori proprii ai lui A care corespund unor valori proprii distincte, sunt ortogonali.
\end{enumerate}
\end{proposition}
\begin{proof}
Pentru a demonstra 1., presupunem c\u{a} \( \lambda \) este o valoare proprie a lui A. Fie \(u \in H - \left \{ 0 \right \}\)  un vector propriu corespunz\u{a}tor, adic\u{a} \(Au = \lambda u\). Atunci
\begin{displaymath}
  \left ( Au, u  \right ) = \left ( \lambda u, u  \right ) = \lambda \left \| u \right \|^{2}
\end{displaymath}
\begin{displaymath}
  \left ( u, Au  \right ) = \left ( u, \lambda u  \right ) = \bar{\lambda} \left \| u \right \|^{2}.
\end{displaymath}
Cum A este simetric \c{s}i \(\left \| u \right \| \neq 0\), deducem c\u{a} \(\lambda  = \bar{\lambda }\).

\noindent Pentru a demonstra 2., lu\u{a}m \^{i}n considerare dou\u{a} valorii proprii ale lui A, \(\left ( u_{1}, \lambda _{1}\right ), \left ( u_{2} , \lambda _{2}\right )\), unde \(\lambda _{1},\lambda _{2} \in \mathbb{R}\) ( din 1.) \c{s}i \(\lambda _{1}\neq \lambda _{2} \).

Avem
\begin{displaymath}
  \lambda _{1}\left ( u_{1},u_{2} \right ) = \left ( Au_{1}, u_{2} \right ) = \left ( u_{2}, Au_{1} \right ) = \lambda _{2}\left ( u_{1}, u_{2}. \right )
\end{displaymath}
Prin urmare
\begin{displaymath}
  \underbrace{\left ( \lambda _{1} - \lambda _{2} \right )}_{\neq 0}\left ( u_{1}, u_{2} \right ) = 0,
\end{displaymath}
deci \(\left ( u_{1}, u_{2} \right ) \) = 0.
\end{proof}
\begin{proposition}
Fie \(\left ( H, \left ( \cdot ,\cdot  \right ), \left \| \cdot  \right \| \right )\) un spa\c{t}iu Hilbert, \(H \neq \left \{ 0 \right \}\)  \c{s}i fie \(A \in L\left ( H \right )\) un operator simetric. Atunci,
\begin{displaymath}
  \left \| A \right \| = \sup \left \{ \left | \left ( Ax,x \right ) \right |;x \in H, \left \| x \right \| =1\right \}.
\end{displaymath}
\end{proposition}
\begin{proof}
Dac\u{a} \(A = 0\), este trivial. Presupunem \(A\neq 0\) \c{s}i not\u{a}m
\begin{displaymath}
  a = \sup \left \{ \left | \left ( Ax,x \right ) \right |; \in H , \left \| x \right \| = 1\right \}.
\end{displaymath}
Cum
\begin{displaymath}
  \left | \left ( Ax,x \right ) \right | \leq  \left \| Ax \right \| \cdot \left \| x \right \| \leq  \left \| A \right \| \cdot   \left \| x \right \|^{2}, \forall x \in H,
\end{displaymath}
deducem c\u{a}
\begin{displaymath}
  a \leq  \left \| A \right \|. \label{eq:1.13} \tag{1.13}
\end{displaymath}
Acum pentru un \(b> 0\) dat \c{s}i \( x \in H\) astfel \^{i}nc\^{a}t \(\left \| x \right \|=1\) \c{s}i \(\left \| Ax \right \| > 0\), avem
\begin{displaymath}
  \left \| Ax \right \|^{2} = \frac{1}{4}\left [ \left ( A\left ( bx + \frac{1}{4}Ax\right ),bx + \frac{1}{b} Ax\right ) - \left ( A\left ( bx - \frac{1}{b} Ax \right ), bx - \frac{1}{b}Ax \right )\right ]. \label{eq:1.14} \tag{1.14}
\end{displaymath}
Avem de asemenea
\begin{displaymath}
  \left | \left ( Av,v \right ) \right |\leq a \left \| v \right \|^{2}, \forall v \in H. \label{eq:1.15} \tag{1.15}
\end{displaymath}
Combin\^{a}nd 1.14 \c{s}i 1.15 ob\c{t}inem
\begin{equation}\nonumber
    \begin{split}
        \left \| Ax \right \|^{2} & \leq \frac{a}{4}\left ( \left \| bx + \frac{1}{b}Ax \right \| ^{2} + \left \| bx - \frac{1}{b}Ax \right \| ^{2} \right ) \\ & \leq \frac{a}{2}\left ( b^{2}\left \| x \right \|^{2} + \frac{1}{b^{2}} \left \| Ax \right \|^{2}\right ),
    \end{split}
\end{equation}
deci pentru \(\left \| x \right \| = 1\) \c{s}i \(b = \left \| Ax \right \| > 0\) avem
\begin{displaymath}
  \left \| Ax \right \|^{2} \leq  a\left \| Ax \right \|.
\end{displaymath}
Prin urmare
\begin{displaymath}
  \left \| Ax \right \| \leq a, \forall x \in H, \left \| x \right \| = 1 \Rightarrow \left \| A  \right \|\leq a.
\end{displaymath}
\end{proof}
\begin{theorem}
\textbf{Hilbert - Schmidt}

\noindent Fie \(\left ( H, \left ( \cdot ,\cdot  \right ), \left \| \cdot  \right \| \right )\) un spa\c{t}iu Hilbert infinit dimensional, separabil \c{s}i fie \(A:H\rightarrow H\) un operator liniar compact, simetric, cu \(N\left ( A \right ) =\{ 0 \}\). Atunci exist\u{a} un \c{s}ir de valori proprii ale lui A, \(\left ( \lambda _{1},\lambda _{2},\cdots,\lambda _{n},... \right )\) astfel \^{i}nc\^{a}t \(\left ( \left | \lambda _{n} \right | \right )\) este un \c{s}ir descresc\u{a}tor de numere pozitive convergente la 0 \c{s}i un sistem ortonormal complet (baz\u{a}) \^{i}n H de vectori proprii corespunz\u{a}tori  \(\left \{ u_{n} \right \}_{n=1}^{\infty }\) ( adic\u{a} \( Au_{n} = \lambda _{n}u_{n} \) pentru \( n = 1,2,\cdots).\)
\end{theorem}
\begin{proof}
\^{I}n primul r\^{a}nd observ\u{a}m faptul c\u{a} \(\left \| A \right \| > 0 \Leftrightarrow A \neq 0\). S\u{a} demonstr\u{a}m c\u{a} ori \(\left \| A \right \|\) ori \(-\left \| A \right \|\) este o valoare proprie a lui A.

Din propozi\c{t}ia 2. exist\u{a} \(\left ( v_{n} \right )_{n\geq 1}\) cu \(\left \| v_{n} \right \| = 1, \forall n\leq 1\), astfel \^{i}nc\^{a}t \(\left | \left ( Av_{n}, v_{n} \right ) \right | \rightarrow \left \| A \right \|\). De fapt, putem s\u{a} extragem din \( \left (  v_{n} \right )\) un sub\c{s}ir, notat tot \(\left (  v_{n} \right )\), astfel \^{i}nc\^{a}t \(\left ( Av_{n} , v_{n}\right )\) converge ori la \(\left \| A \right \|\) ori la \( - \left \| A \right \|\)
\begin{displaymath}
  \left ( Av_{n}, v_{n} \right ) \rightarrow  \lambda _{1} := \left \| A \right \|. \label{eq:1.16} \tag{1.16 }
\end{displaymath}
Cum A este compact putem s\u{a} lu\u{a}m acum alt sub\c{s}ir, notat\u{a} de asemenea \(\left ( v_{n} \right )\), astfel \^{i}nc\^{a}t
\begin{displaymath}
  Av_{n} \rightarrow u_{1} \label{eq:1.17} \tag{1.17}
\end{displaymath}
\c{s}i acesta este sub\c{s}irul pe care \^{i}l p\u{a}str\u{a}m. Acum, trec\^{a}nd la limit\u{a} \^{i}n
\begin{displaymath}
  0\leq \left \| Av_{n}  - \lambda _{1}v_{n}\right \|^{2} = \left \| Av_{n} \right \|^{2} - 2\lambda _{1}\left ( Av_{n}, v_{n} \right ) + \lambda _{1}^{2} \label{eq:1.18} \tag{1.18}
\end{displaymath}
vom ob\c{t}ine din  1.16 \c{s}i 1.17
\begin{displaymath}
  0 \leq  \left \| u_{1} \right \|^{2} - \lambda _{1}^{2} \Rightarrow \left | \lambda _{1} \right |\leq \left \| u_{1} \right \|.
\end{displaymath}
Deci, \^{i}n particular, \(u_{1}\neq 0\). Este adev\u{a}rat \c{s}i invers, deoarece avem
\begin{displaymath}
  \left \| Av_{n} \right \| \leq \left \| A \right \|\cdot \left \| v_{n} \right \| = \left \| A \right \|,
\end{displaymath}
deci din 1.17
\begin{displaymath}
  \left \| u_{1} \right \|\leq \left \| A \right \| = \left | \lambda _{1} \right |.
\end{displaymath}
Prin urmare,
\begin{displaymath}
  \left \| u_{1} \right \|  = \left | \lambda _{1} \right | = \left \| A \right \|. \label{eq:1.19} \tag{1.19}
\end{displaymath}
Din 1.18, deducem
\begin{displaymath}
  \left \| Av_{n}  - \lambda _{1}v_{n}\right \| \rightarrow 0. \label{eq:1.20} \tag{1.20}
\end{displaymath}
Deci, av\^{a}nd \^{i}n vedere 1.17, \(\left ( \lambda _{1}v_{n} \right )\)  converge la \(u_{1}\) \c{s}i astfel din 1.20 \c{s}i continuitatea lui A ob\c{t}inem
\begin{displaymath}
  Au_{1} = \lambda _{1}u_{1},
\end{displaymath}
adic\u{a}, \(\left ( u_{1},\lambda _{1} \right )\) este o pereche proprie a lui A. O normaliz\u{a}m f\u{a}r\u{a} s\u{a} ne schimb\u{a}m notaţia, \(u_{1} := \left | \lambda _{1} \right |^{-1}u_{1}\), deoarece dorim un sistem ortonormal de vectori proprii.

\noindent Merit\u{a} subliniat c\u{a} orice alt\u{a} valoare proprie \( \lambda\) satisface \(\left | \lambda  \right |\leq \left | \lambda _{1} \right |\). \^{I}ntr-adev\u{a}r, dac\u{a} presupunem prin absurd existenţa unei perechi proprii \(\left ( u,\lambda  \right )\), cu \(\left | \lambda  \right |> \left | \lambda _{1} \right |\) \c{s}i \(\left \| u \right \|=1\), atunci \(\left | \left ( Au,u \right ) \right | = \left | \lambda  \right |\) care contrazice \(\left | \lambda_{1}  \right | = \left \| A \right \|\) acesta fiind supremul de la propozi\c{t}ia 2.

Folosind induc\c{t}ia deducem existen\c{t}a perechilor proprii \(\left ( u_{n},\lambda _{n} \right )\) pentru \(n=2,3,...\).

Not\u{a}m cu Y complementul ortogonal al lui \(\ Span \left \{ u_{1} \right \}\), adic\u{a}
\begin{displaymath}
  Y = \left \{ u \in H; \left ( u,u_{1} \right ) = 0  \right \}.
\end{displaymath}
Cum H este infinit dimensional, la fel este \c{s}i Y. Mai mult, Y este un spa\c{t}iu Hilbert \c{s}i este invariant la A \^{i}n sensul c\u{a} \(AY \subset Y\) deoarece, pentru \( y\in Y\),
\begin{equation} \nonumber
    \begin{split}
        \left ( Ay, u_{1} \right ) &  = \left ( y,Au_{1} \right ),
    \end{split}
\end{equation}
din moment ce A este simetric \c{s}i
\begin{equation} \nonumber
\begin{split}
    \\ & = \left ( y,\lambda _{1}u_{1} \right ) \\ & = \lambda _{1}\left ( y,u_{1} \right ) \\ & = 0.
\end{split}
\end{equation}
Restric\c{t}ia \(A\mid Y\) nu este 0 din moment ce \(N\left ( A \right ) = Y\). De fapt, toate propriet\u{a}\c{t}ile sunt mo\c{s}tenite de la pasul anterior; avem o valoare proprie
\begin{displaymath}
  \lambda _{2} = \pm \sup \left \{ \left | Av,v \right | ; v \in Y, \left \| v \right \| = 1\right \}
\end{displaymath}
\c{s}i func\c{t}ia proprie corespunz\u{a}toare
\begin{displaymath}
  u_{2} \in Y ,\left \| u_{2} \right \| = 1, Au_{2} = \lambda _{2}u_{2}.
\end{displaymath}
Mai mult, \(\left | \lambda _{2} \right |\leq \lambda _{1}\).
Mai departe, lu\u{a}m
\begin{displaymath}
  Z = \left \{ u \in Y ; \left ( u,u_{2} \right ) = 0 \right \} = \left ( \ Span\left \{ u_{1}, u_{2} \right \} \right )^{\perp }
\end{displaymath}
care este un subspa\c{t}iu infinit dimensional (Hilbert) al lui H \c{s}i ob\c{t}inem
o nou\u{a} pereche proprie \(\left ( u_{3},\lambda _{3} \right )\), cu \(\left \| u_{3} \right \| = 1, \left | \lambda _{3} \right |\leq \left | \lambda _{2} \right |\). Putem continua \c{s}i ob\c{t}inem de fiecare dat\u{a} un subspa\c{t}iu infinit dimensional. Vom construi astfel un \c{s}ir de valori proprii \(\left ( \lambda _{n} \right )\) astfel \^{i}nc\^{a}t
\begin{displaymath}
  \left | \lambda _{1} \right |\geq \left | \lambda _{2} \right |\geq \cdots \geq \left | \lambda _{n} \right |\geq \cdots \label{eq:1.21} \tag{1.21}
\end{displaymath}
\c{s}i \c{s}irul corespunz\u{a}tor de vectori proprii \(\left ( u_{n} \right )\),
\begin{displaymath}
  Au_{n} = \lambda _{n}u_{n}, \left \| u_{n} \right \| = 1, n\geq 1
\end{displaymath}
care formeaz\u{a} prin construcţie un sistem ortonormal.

\noindent Mai departe, vom demonstra c\u{a}
\begin{displaymath}
  Au = \sum_{n=1}^{\infty }\lambda _{n}\left ( u,u_{n} \right )u_{n}, \forall u \in H . \label{eq:1.22} \tag{1.22}
\end{displaymath}
Definim spa\c{t}iul
\begin{displaymath}
   V_{m}:= \left \{ u \in H;\left ( u,u_{j} \right ) = 0, j = 1,\cdots,m \right \} = \ Span\left \{ u_{1},...,u_{m} \right \}^{\perp },
\end{displaymath}
care este un spa\c{t}iu Hilbert infinit dimensional, invariant la A. Din pasul anterior din demonstra\c{t}ia noastr\u{a}, exist\u{a} o pereche proprie \(\left ( u_{m+1}, \lambda _{m+1} \right )\) a lui A astfel \^{i}nc\^{a}t
\begin{displaymath}
  \left | \lambda _{m+1} \right | = \left \| A\mid V_{m} \right \| = \ sup \left \{ \left | \left ( Av,v \right ) \right |; v \in V_{m+1} , \left \| v \right \| = 1\right \}.
\end{displaymath}
\^{I}n particular,
\begin{displaymath}
  \left \| Av \right \| \geq  \left | \lambda _{m+1} \right | \cdot \left \| v \right \| , \forall v \in V_{m+1}. \label{eq:1.23} \tag{1.23}
\end{displaymath}
Acum, alegem
\begin{displaymath}
  w_{m}= u - \sum_{n=1}^{m}\left ( u,u_{n} \right )u_{n}
\end{displaymath}
\c{s}i observ\u{a}m c\u{a} \(w_{m} \in V_{m}\) deoarece \(\left ( v_{m} ,u_{j}\right ) = \left ( u,u_{j} \right ) - \left ( u,u_{j} \right ) = 0, \forall j = 1,...,m. \) Calcul\u{a}m
\begin{displaymath}
  \left \| w_{m} \right \|^{2} = \left \| u \right \|^{2} - \sum_{n=1}^{m}\left | \left ( u,u_{n} \right ) \right |^{2} \leq \left \| u \right \|^{2}. \label{eq:1.24} \tag{1.24}
\end{displaymath}
Combin\^{a}nd 1.23 \c{s}i 1.24 avem
\begin{equation} \nonumber
    \begin{split}
      Aw_{m} & = Au - \sum_{n=1}^{m}\left ( u,u_{n} \right )Au_{n}  \\ &  Au - \sum_{n=1}^{m}\lambda _{n}\left ( u,u_{n} \right )u_{n},
    \end{split}
\end{equation}
de unde
\begin{equation} \nonumber\label{eq:1.25} \tag{1.25}
    \begin{split}
        \left \| Aw_{m} \right \| \leq \left \| A\mid V_{m} \right \| \cdot \left \| w_{m} \right \| & \left | \lambda _{m+1} \right | \cdot \left \| w_{m} \right \| \\ & \leq \left | \lambda _{m+1} \right | \cdot \left \| u\right \| .
    \end{split}
\end{equation}
Pe de alt\u{a} parte, \(\lambda _{n} \rightarrow 0\). \^{I}ntr-adev\u{a}r, din moment ce \(\left ( \left | \lambda _{n} \right | \right )\) este descresc\u{a}tor,  exist\u{a}
\begin{displaymath}
  \lim_{n\rightarrow \infty }\left | \lambda _{n} \right | = \alpha \geq  0.
\end{displaymath}
S\u{a} presupunem, prin absurd, c\u{a} \(\alpha > 0\). \^{I}n mod evident, \(\left | \lambda _{n} \right |\geq \alpha \) pentru orice \(n\geq 1\) \c{s}i deci
\begin{displaymath}
  \left \| \lambda_{n}^{-1} u_{n}\right \| = \frac{\left \| u_{n} \right \|}{\left | \lambda _{n} \right |} = \frac{1}{\left | \lambda _{n} \right |} \leq \frac{1}{\alpha }, \forall n\geq 1.
\end{displaymath}
Cum A este compact \(u_{n} = A\left ( \lambda _{n}^{-1}u_{n} \right )\) are un sub\c{s}ir convergent. Dar acest lucru este imposibil deoarece,
\begin{displaymath}
  \left \| u_{n}-u_{m} \right \|^{2} = \left \| u_{n} \right \|^{2} + \left \| u_{m} \right \|^{2} = 2, \forall n\neq m.
\end{displaymath}
Deci, \(\alpha  = 0\), adic\u{a} \(\lambda _{n} \rightarrow 0\), a\c{s}a cum am afirmat.

\noindent \^{I}n consecin\c{t}\u{a}, avem din 1.25 c\u{a} \(\left \| Aw_{m} \right \| \rightarrow 0\) pentru \(m\rightarrow \infty \), adic\u{a} 1.22 e adev\u{a}rat\u{a}.

\^{I}n final, s\u{a} demonstr\u{a}m c\u{a} \(\left \{ u_{n} \right \}_{n=1}^{\infty }\) este o baz\u{a} \^{i}n H.

\noindent \c{S}tim c\u{a} pentru orice \( u \in H\) seria \(\sum_{n=1}^{\infty }\left ( u,u_{n} \right )u_{n}\) converge, deci putem s\u{a} scriem
\begin{displaymath}
  v = \sum_{n=1}^{\infty }\left ( u,u_{n} \right )u_{n}
\end{displaymath}
deci  trebuie s\u{a} verific\u{a}m c\u{a} \(u = v\).

Lu\u{a}m \^{i}n considerare \c{s}irul de sume par\c{t}iale \(s_{m} = \sum_{n=1}^{m}\left ( u,u_{n} \right )u_{n}\) care converge la \(v\) pentru \(m\rightarrow \infty \), deci \(As_{m} \rightarrow Av\). Pe de alt\u{a} parte, din 1.22 avem c\u{a}
\begin{displaymath}
  As_{m} = \sum_{n=1}^{m}\lambda _{n}\left ( u,u_{n} \right )u_{n} \rightarrow Au~~ \text{pentru}~ m\rightarrow \infty.
\end{displaymath}
Prin urmare,
\begin{displaymath}
  Av = Au \Rightarrow A\left ( v-u \right ) = 0\Rightarrow v = u,
\end{displaymath}
deoarece \(\ker A = \left \{ 0 \right \}\).

Astfel sistemul \(\left \{ u_{n} \right \}_{n=1}^{\infty }\) este complet, adic\u{a} este o baz\u{a} \^{i}n H.
\end{proof}
\begin{remark}
Dac\u{a} presupunem \^{i}n plus c\u{a} A este pozitiv, atunci acesta are valori proprii \(\lambda _{1} \geq \lambda _{2} \geq \cdots \lambda _{n }\geq \cdots\) cu \(\lambda _{n }> 0 , \forall n \geq 1\). Acest lucru rezult\u{a} din
\begin{displaymath}
  \left ( Au_{n},u_{n} \right ) = \lambda _{n} \left \| u_{n} \right \|^{2} = \lambda _{n}, n\geq 1.
\end{displaymath}
De asemenea, \c{t}inem cont c\u{a}
\begin{displaymath}
  \lambda _{1} = \left \| A \right \| = \ sup \left \{ \left ( Av,v \right ); v\in H, \left \| v \right \| = 1\right \}
\end{displaymath}
\c{s}i
\begin{displaymath}
  \lambda _{n+1} = \left \| A\mid V_{n} \right \| = \ sup \left \{ \left ( Av,v \right ); v\in V_{n}, \left \| v \right \| = 1\right \}, \forall n \geq 1
\end{displaymath}
unde
\begin{displaymath}
  V_{n} = \left ( \ Span \left \{ u_{1},u_{2},....,u_{n} \right \} \right )^{\perp }, n\geq 1.
\end{displaymath}
\end{remark}

\section{Alte rezultate}

\^{I}n aceast\u{a} sec\c{t}iune vom da, f\u{a}r\u{a} demonstra\c{t}ii dou\u{a} teoreme \c{s}i c\^{a}teva defini\c{t}ii pe care le vom utiliza \^{i}n urm\u{a}torul capitol.

Primul dintre acestea este un rezultat de densitate in spatiul $L^{p}\left ( \Omega  \right ):$

\begin{theorem}
Fie \(\Omega \subset \mathbb{R}^{k}\) o mul\c{t}ime nevid\u{a}. Avem \(\overline{C_{0}^{\infty }\left ( \Omega  \right )}^{L^{p}\left ( \Omega  \right )} = L^{p}\left ( \Omega  \right )\)  pentru orice \(1\leq p< \infty \).
\end{theorem}

\textbf{Criteriul Arzelà-Ascoli}

Fie \(\left ( X,d \right )\) \c{s}i \(\left ( X_{1},d_{1} \right )\) spa\c{t}iile metrice \c{s}i fie \(\varnothing \neq A\subset X\). Not\u{a}m cu \(C\left ( A;X_{1} \right )\), mul\c{t}imea tuturor func\c{t}iilor continue din \(A\subset \left ( X,d \right )\) la \(\left ( X_{1},d_{1} \right )\).
\begin{definition}
O familie de func\c{t}ii \(\mathcal{F} \subset C\left ( A;X_{1} \right )\) se nume\c{s}te echicontinu\u{a} dac\u{a} pentru orice \(\varepsilon > 0\) \c{s}i orice \(x \in A\) exist\u{a} \(\delta > 0\) astfel \^{i}nc\^{a}t \(y \in A\) \c{s}i \(d\left ( x,y \right )< \delta \) implic\u{a} \(d_{1}\left ( f\left ( x \right ),f\left ( y \right ) \right )< \varepsilon \) pentru orice \(f \in \mathcal{F} \), adic\u{a} \(\delta =\delta \left ( \varepsilon ,x \right )\) este independent\u{a} fa\c{t}\u{a} de \(f\). 
\end{definition}
\begin{definition}
Dac\u{a}, \^{i}n plus, \(\delta =\delta \left ( \varepsilon  \right )\) ( adic\u{a} \(\delta\) este independent\u{a} fa\c{t}\u{a} de x \c{s}i f), atunci \(\mathcal{F}\) este uniform echicontinu\u{a}, adic\u{a} \(\forall \varepsilon > 0, \forall x,y \in A, d\left ( x,y \right )< \delta \) implic\u{a} \(d_{1}\left ( f\left ( x \right ), f\left ( y \right ) \right )< \varepsilon \), pentru orice \(f\in \mathcal{F}\). 
\end{definition}
\begin{theorem}
Fie \(\varnothing \neq A\subset \left ( X,d \right )\) compact. Presupunem c\u{a} \(\mathcal{F} \subset C\left ( A,\mathbb{R}^{k} \right )\) este echicontinu\u{a} \c{s}i m\u{a}rginit\u{a} \^{i}n \(C\left ( A;\mathbb{R}^{k} \right )\), (adic\u{a} \(\exists M> 0\) astfel \^{i}nc\^{a}t \(\left \| f\left ( x \right ) \right \| \leq M, \forall x \in A, \forall f \in \mathcal{F}\) ) .Atunci \(\mathcal{F}\)este relativ compact\u{a} \^{i}n \(C\left ( A;\mathbb{R}^{k} \right )\) echipat\u{a} cu sup-norm\u{a}. 
\end{theorem}

\begin{definition} \textbf{Operatorul adjunct-Hilbert }
Fie \(T : \mathcal{D}\left ( T \right ) \rightarrow H\) un operator liniar dens definit \^{i}ntr-un spa\c{t}iu Hilbert complex H. Atunci, operatorul ajunct-Hilbert \(T^{\ast } : \mathcal{D}\left ( T^{\ast } \right ) \rightarrow H\) al lui T va fi definit, dup\u{a} cum urmeaz\u{a}. Domeniul \(\mathcal{D}\left ( T^{\ast } \right )\) al lui \(T^{\ast }\) const\u{a} \^{i}n orice \(y \in /H\) astfel \^{i}nc\^{a}t s\u{a} existe un \(y^{\ast } \in H\) care s\u{a} satisfac\u{a} condi\c{t}ia
\begin{displaymath}
\left \langle Tx,y \right \rangle = \left \langle x,y^{\ast } \right \rangle
\end{displaymath}
pentru orice \(x \in \mathcal{D}\left ( T \right )\). Pentru fiecare astfel de \(y \in \mathcal{D}\left ( T^{\ast } \right )\) operatorul adjunct-Hilbert \(T^{\ast }\) este atunci definit \^{i}n termenii acelui \(y^{\ast }\) de
\begin{displaymath}
y^{\ast } = T^{\ast }y.
\end{displaymath}
Cu alte cuvinte , un element \(y\in H\) este \^{i}n \(\mathcal{D}\left ( T^{\ast } \right )\) dac\u{a} (\c{s}i numai dac\u{a}) pentru acel y produsul interior\(\left \langle Tx,y \right \rangle \), considerat ca o func\c{t}ie a lui x, poate fi reprezentat sub forma  \(\left \langle Tx,y \right \rangle = \left \langle x,y^{\ast } \right \rangle\) pentru orice \(v\). De asemenea pentru acel y, formula de mai susdetermin\u{a} corespondentul unic \(y^{\ast }\) din moment ce \(\mathcal{D}\left ( T \right )\) prin presupunerea noastr\u{a} , este dens\u{a} \^{i}n H.
\end{definition}
\begin{definition}\textbf{Operatorul liniar simetric }
Fie \(T :  \mathcal{D}\left ( T \right ) \rightarrow H\) un operator liniar care este dens definit \^{i}ntr-un spa\c{t}iu Hilbert H complex . Atunci T se nume\c{s}te un operator liniar simetric dac\u{a} pentru orice \(x,y\in \mathcal{D}\left ( T \right )\), 
\begin{displaymath}
\left \langle Tx,y \right \rangle = \left \langle x,Ty \right \rangle. 
\end{displaymath}
\end{definition}
\begin{definition}\textbf{Operatorul liniar autoadjunct}
Fie \(T :  \mathcal{D}\left ( T \right ) \rightarrow H \) un operator liniar care este dens definit \^{i}ntr-un spa\c{t}iu hilbert complex. Atunci T se nume\c{s}te operator liniar autoadjunct dac\u{a}
\begin{displaymath}
T = T^{\ast }
\end{displaymath}
Orice operator liniar autoadjunct este simetric. 

\noindent Pentru un operator \(T : H \rightarrow H\) pe un spa\c{t}iu Hilbert H complex, conceptele de simetric \c{s}i autoadjunct sunt identice. 
\end{definition}


\chapter{Ecua\c{t}ii Integrale}
\nocite{micula}
\nocite{gmorosanu}

Acest capitol este o introducere \^{i}n teoriei ecua\c{t}iilor liniare de tip Volterra \c{s}i Fredholm. Sunt abordate \c{s}i unele aspecte legate de anumite extensii neliniare.


\section{Ecua\c{t}ii Volterra}

Vom \^{i}ncepe cu ecua\c{t}ii scalare \c{s}i liniare de tip Volterra. Exist\u{a} doua tipuri de astfel de ecua\c{t}ii care sunt cele mai relevante pentru aplica\c{t}ii \c{s}i anume
\begin{displaymath}
f\left ( t \right ) = \int_{a}^{t}k\left ( t,s \right )x\left ( s \right )ds,    a\leq t\leq b  \label{eq:2.1} \tag{2.1}
\end{displaymath}
\c{s}i
\begin{displaymath}
x\left ( t \right ) = f\left ( t \right ) + \int_{a}^{t}k\left ( t,s \right )x\left ( s \right )ds, a\leq t\leq b, \label{1.2} \tag{2.2}
\end{displaymath}
unde \[a,b \in \mathbb{R}, a< b, f\in C\left [ a,b \right ]:= C\left ( \left [ a,b \right ];\mathbb{R} \right ), k\in C\left (\Delta   \right ):= C\left ( \Delta ;\mathbb{R} \right )\] (numit nucleu), cu \(\Delta =\left \{ \left ( t,s \right )\in \mathbb{R}^{2};a\leq s\leq t\leq b \right \};\) \(x=x\left ( t \right )\) reprezint\u{a} func\c{t}ia necunoscut\u{a} care apar\c{t}ine spa\c{t}iul \(C\left [ a,b \right ]\).

\noindent Ecua\c{t}ia \ref{eq:2.1} este cunoscut\u{a} ca  ecua\c{t}ie Volterra de prima spe\c{t}\u{a}, \^{i}n timp ce ecua\c{t}ia \ref{1.2} este cunoscut\u{a} ca cua\c{t}ie Volterra de spe\c{t}a a doua. \^{I}n cele ce urmeaz\u{a} vom examina ecua\c{t}ia Volterra de spe\c{t}a a doua. Vom ar\u{a}ta mai t\^{a}rziu faptul c\u{a} ecua\c{t}ia Volterra de spe\c{t}a \^{i}nt\^{a}i se reduce de fapt la cea de spe\c{t}a a doua \^{i}n condi\c{t}ii adecvate.


\begin{theorem}
\textbf{Existeten\c{t}a \c{s}i Unicitatea}

\^{I}n ipotezele de mai sus exist\u{a} o solu\c{t}ie unic\u{a} \(x\in C\in \left [ a,b \right ]\) a ecua\c{t}iei \ref{1.2}.

\end{theorem}
\noindent Vom prezenta \^{i}n cele ce urmeaz\u{a} trei demonstra\c{t}ii diferite.
\begin{proof}

Not\u{a}m \(K = sup_{\left ( t,s \right )\in \Delta }\left | k\left ( t,s \right ) \right |\) care este finit deoarece \(\Delta\) este o submul\c{t}ime compact\u{a} al lui \(\mathbb{R}^{2}\). Presupunem \^{i}ntr-o prim\u{a} etap\u{a} c\u{a}
\begin{displaymath}
K\left ( b-a \right ) < 1. \label{eq:2.3} \tag{2.3}
\end{displaymath}
Consider\u{a}m \(X = C\left [ a,b \right ]\)  cu norm sup \(\left \| g \right \| = sup_{a\leq t\leq b}\left | g\left ( t \right ) \right |\)  \c{s}i metrica corespunzatoare, \(d\left ( g_{1}, g_{2} \right ) = \left \| g_{1} - g_{2} \right \|\).

\noindent Definim \(T : X \rightarrow X\) prin

\begin{displaymath}
\left ( Tg \right )\left ( t \right ) = f\left ( t \right ) + \int_{a}^{t}k\left ( t,s \right )g\left ( s \right )ds, t\in \left [ a,b \right ], g\in X. \label{eq:2.4} \tag{2.4}
\end{displaymath}
Este clar din ecua\c{t}ia \ref{eq:2.4} c\u{a} \(T\) duce \(X\) \^{i}n  \(X\). De asemenea
\begin{equation}\nonumber
\begin{split}
\left | \left ( Tg_{1} \right )\left ( t \right ) - \left ( Tg_{2} \right )\left ( t \right )  \right | &= \left | \int_{a}^{t}k\left ( t,s \right )\left [ g_{1}\left ( s \right ) - g_{2}\left ( s \right ) \right ]ds \right | \\ & \leq \ \int_{a}^{t}\left | k\left ( t,s \right ) \right |\cdot \left | g_{1}\left ( s \right ) - g_{2}\left ( s \right ) \right |ds \\ & \leq K \left ( b-a \right )\left \| g_{1} - g_{2} \right \|
\end{split}
\end{equation}
pentru orice \(g_{1}, g_{2} \in X\) \c{s}i orice  \(t\in \left [ a,b \right ]\).
Prin urmare
\begin{displaymath}
d\left ( Tg_{1} , Tg_{2}\right )\leq K\left ( b-a \right )d\left ( g_{1}, g_{2} \right ),
\end{displaymath}
adic\u{a} \(T\) este o contrac\c{t}ie (conform \ref{eq:2.3}).

Conform Principiului contrac\c{t}iilor al lui Banach (din Capitolul 2), \(T\) are un punct fix unic \(x \in  X\) care este \^{i}n mod clar unica solu\c{t}ie a ecua\c{t}iei  \ref{1.2}).

\noindent Dac\u{a} condi\c{t}ia \ref{eq:2.3} nu este \^{i}ndeplinit\u{a}, atunci consider\u{a}m o subdiviziunea a intervalului \(\left [ a,b \right ]\), de exemplu
\begin{displaymath}
  a = t_{0}< t_{1}< \cdots < t_{N-1}< t_{N} = b,
\end{displaymath}
unde \(t_{j} = a + jh\) pentru \(j = 1,2,….,N, h = \frac{\left ( b-a \right )}{N}\), cu \(N\) suficient de mare astfel \^{i}nc\^{a}t \(Kh < 1\). \^{I}n particular, \(K\left ( t_{1}  - t_{0}\right ) = Kh< 1\), deci de mai sus rezult\u{a} ca \ref{1.2} are o solu\c{t}ie unic\u{a} \(x_{1} = x_{1}\left ( t \right )\) pe intervalul \(\left [ t_{0} , t_{1} \right ] = \left [ a, t_{1} \right ]\), adic\u{a},
\begin{displaymath}
x_{1}\left ( t \right ) = f\left ( t \right ) + \int_{a}^{t}k\left ( t,s \right )x_{1}\left ( s \right )ds, t \in \left [ a, t_{1} \right ].
\end{displaymath}

\noindent Fie ecua\c{t}ia
\begin{displaymath}
x_{1}\left ( t \right ) =\underbrace {f\left ( t \right ) + \int_{a}^{t}k\left ( t,s \right )x_{1}\left ( s \right )ds}_{=:f_{1}\left ( t \in C\left [ t_{1}, t_{2} \right ] \right )} + \int_{t_{1}}^{t_{2}}k\left ( t,s \right )x\left ( s \right )ds,  t\in \left [ t_{1}, t_{2} \right ],
\end{displaymath}
\begin{displaymath}
f\left ( t \right ) + \int_{a}^{t}k\left ( t,s \right )x_{1}\left ( s \right )ds =:f_{1}\left ( t \right ) \in C \left [ t_{1} , t_{2} \right ].
\end{displaymath}
Cum \(K\left ( t_{2} - t_{1} \right ) = Kh < 1\), rezult\u{a} din argumentul de mai sus c\u{a} aceast\u{a} ecua\c{t}ie are o solu\c{t}ie unic\u{a} \(x_{2} \in C\left [ t_{1}, t_{2} \right ]\)  \c{s}i evident, \(x_{2} \left ( t_{1} \right ) = x_{1} \left ( t_{1} \right )\).

\^{I}n mod similar, exist\u{a} o func\c{t}ie unic\u{a} \(x_{3} \in C\left [ t_{2}, t_{3} \right ]\) care satisfice urm\u{a}toarea ecua\c{t}ie, pentru orice \(t\in \left [ t_{2} , t_{3} \right ]\)
\begin{displaymath}
x_{3}\left ( t \right ) = f\left ( t \right ) + \int_{a}^{t_{1}}k\left ( t,s \right )x_{1}\left ( s \right )ds + \int_{t_{1}}^{t_{2}}k\left ( t,s \right )x_{2}\left ( s \right )ds + \int_{t_{2}}^{t}k\left ( t,s \right )x_{3}\left ( s \right )ds
\end{displaymath}
\c{s}i \(x_{3}\left ( t_{2} \right ) = x_{2}\left ( t_{2} \right )\).

Continu\^{a}nd aceast\u{a} procedur\u{a} ob\c{t}inem o solu\c{t}ie \(x\in C\left [ t_{0}, t_{N} \right ] = C\left [ a,b \right ]\) a ecua\c{t}iei \ref{1.2} definit\u{a} de \(x\left ( t \right ) = x_{j}\left ( t \right )\) pentru \(t\in \left [ t_{j-1}, t_{j} \right ], j = 1,2,\cdots,N\). Solu\c{t}ia \(x\) este evident unic\u{a}.
\end{proof}
\begin{proof}

Din nou, lu\u{a}m \^{i}n considerare operatorul \texttt{T} definit de ecua\c{t}ia \ref{eq:2.4} unde \(X\) este acela\c{s}i ca mai sus. Se vede u\c{s}or c\u{a}
\begin{displaymath}
\left | \left ( Tg_{1} \right )\left ( t \right )  - \left ( Tg_{2} \right )\left ( t \right )\right |\leq K\left \| g_{1}  - g_{2}\right \|\left ( t-a \right ), \forall t\in \left [ a,b \right ], g_{1}, g_{2} \in X.
\end{displaymath}
\^{I}n consecin\c{t}\u{a}, pentru \(T^{2} = T \circ T\) ob\c{t}inem


\begin{equation} \nonumber
    \begin{split}
        \left | \left ( T^{2}g_{1} \right )\left ( t \right )  - \left ( T^{2}g_{2} \right )\left ( t \right )\right |& \leq \int_{a}^{t}\left | k\left ( t,s \right ) \right |\cdot \left | \left ( Tg_{1} \right )\left ( s \right ) - \left ( Tg_{2} \right )\left ( s \right ) \right |ds \\ & \leq K^{2}\left \| g_{1} - g_{2} \right \|\int_{a}^{t}\left ( s-a \right )ds \\& = \frac{K^{2}\left ( t-a \right )^{2}}{2!}\left \| g_{1} - g_{2} \right \|
    \end{split}
\end{equation}

\end{proof}

\noindent Se poate demonstra prin induc\c{t}ie faptul c\u{a}:
\begin{equation}\nonumber
    \begin{split}
        \left | \left ( T^{k}g_{1} \right ) \left ( t \right ) - \left ( T^{k}g_{2} \right )\left ( t \right )\right | & \leq \frac{K^{k}\left ( t-a \right )^{k}}{k!}\left \| g_{1} - g_{2} \right \| \\ & \leq \frac{K^{k}\left ( b-a \right )^{k}}{k!}\left \| g_{1}- g_{2} \right \|
    \end{split}
\end{equation}
pentru orice  \(t\in \left [ a,b \right ], g_{1}, g_{2} \in X, k = 1,2,\cdots\)
\noindent
Lu\u{a}m supremum-ul \c{s}i ob\c{t}inem
\begin{displaymath}
d\left ( T^{k}g_{1} , T^{k}g_{2}\right ) \leq \frac{K^{k}\left ( b - a  \right )^{k}}{k!}\left \| g_{1} - g_{2}\right \|, \forall g_{1}, g_{2} \in X, k = 1,2,... \label{eq:2.5} \tag{2.5}
\end{displaymath}

\noindent Cum \(K^{k} \left ( b-a \right )^{\frac{k}{k!}}\rightarrow \infty\), pentru \(k\rightarrow \infty, T^{k}\) este o contrac\c{t}ie dac\u{a} \(k\) suficient de mare (conform \ref{eq:2.5}). Conform observa\c{t}iei 3, \(T\) are un punct fix unic \(x \in X\), care este solu\c{t}ia unic\u{a} a ecua\c{t}iei \ref{1.2}.

\begin{proof}

Fie \(T\) acela\c{s}i operator ca mai \^{i}nainte, dar lu\u{a}m \^{i}n considerare o alt\u{a} norm\u{a} pe \(X = C\left [ a,b \right ]\), norma Bielecki, definit\u{a} de
\begin{displaymath}
\left \| g \right \|_{B} = \sup_{a\leq t\leq b} e^{-Lt}\left | g\left ( t \right ) \right |
\end{displaymath}
cu \(L\) o constant\u{a} pozitiv\u{a} astfel \^{i}nc\^{a}t \(\frac{K}{L} < 1\). Aceasta este \^{i}ntr-adevar o norm\u{a} pe \(X\) care este echivalent\u{a} cu norma sup uzual\u{a}. Not\u{a}m cu \(d_{B}\) metrica generat\u{a} de \(\left \| \cdot \right \|_{B}\).

Vom avea pentru orice \(t\in \left [ a,b  \right ]\) \c{s}i \(g_{1}, g_{2} \in X\)
\begin{equation} \nonumber
    \begin{split}
     \left | \left ( Tg_{1} \right )\left ( t \right ) - \left ( Tg_{2} \right )\left ( t \right ) \right | &  \leq  \int_{a}^{t} \left | k\left ( t,s \right ) \right |e^{Ls}e^{-Ls}\left | g_{1}\left ( s \right ) - g_{2}\left ( s \right ) \right |ds \\&   \leq K\left \| g_{1} - g_{2} \right \|_{B} \int_{a}^{t}e^{Ls}ds  \\ & = \frac{K \left \| g_{1} - g_{2} \right \|_{B}}{L}\left ( e^{Lt} - e^{La}\right ),
    \end{split}
\end{equation}

\noindent de unde

\begin{equation} \nonumber
    \begin{split}
     e^{-Lt}\left | \left ( Tg_{1} \right ) \left ( t \right ) - \left ( Tg_{2} \right )\left ( t \right )\right | &    \leq \frac{K}{L}\left \| g_{1} - g_{2}\right \|_{B}\left ( 1 - e^{-L\left ( t-a \right )} \right ) \\ & \leq  \frac{K}{L}\left \| g_{1}  - g_{2}\right \|_{B}.
    \end{split}
\end{equation}

\noindent Acum lu\u{a}m supremum-ul pentru \(t \in \left [ a,b \right ]\)
\begin{displaymath}
d_{B}\left ( Tg_{1}, Tg_{2} \right ) \leq \frac{K}{L}d_{B}\left ( g_{1} , g_{2}\right ), \forall g_{1}, g_{2}\in X.
\end{displaymath}
Cum \(\frac{K}{L} < 1, T\) este o contrac\c{t}ie \^{i}n raport cu \(d_{B}\), prin urmare, concluzia teoremei rezult\u{a} din nou din Principiul contrac\c{t}iilor al lui Banach.
\end{proof}
\textbf{Nucleul resolvent}

S\u{a} presupunem c\u{a} sunt \^{i}ndeplinite condi\c{t}iile de mai sus pentru \(f\) si \(k\). Pentru \(n\in \mathbb{N}, t\in \left [ a,b \right ]\) vom avea
\begin{equation} \nonumber
\begin{split}
x_{n}\left ( t \right ) &= f\left ( t \right ) + \int_{a}^{t}k\left ( t,s \right )x_{n-1}\left ( s \right )ds, \\
x_{0}\left ( t \right ) &= f\left ( t \right ).
\end{split}
\end{equation}
\^{I}n mod clar, \(x_{n} \in X = C\left [ a,b \right ]\) pentru orice \(n\). De fapt, \c{s}irul de mai sus \(\left ( x_{n} \right )_{n\geq 0}\) poate fi exprimat ca
\begin{displaymath}
x_{n} = Tx_{n-1}, n\in \mathbb{N}; x_{0} = f,
\end{displaymath}
unde \(T : X \rightarrow X\) este operatorul definit de \ref{eq:2.4}. Deci, \(\left (x_{n}  \right )\) este de fapt \c{s}irul de aproxim\u{a}ri succesive (asociate operatorului T) care a fost folosit \^{i}n demonstrarea Principiului contrac\c{t}iilor al lui Banach (Capitolul 1.2). Aici lu\u{a}m \^{i}n considerare o anumit\u{a} func\c{t}ie de \^{i}nceput, \(x_{0} = f\). Din demonstra\c{t}ia Principiului contrac\c{t}iilor al lui Banach \c{s}tim c\u{a} \(\left ( x_{n} \right )\) converge \^{i}n \(\left ( C\left [ a,b \right ], \left \| \cdot  \right \|_{B} \right )\) c\u{a}tre punctul s\u{a}u fix unic \(x\), adic\u{a} \(\left ( x_{n} \right )\) converge uniform \^{i}n \(\left [ a,b \right ]\) la solu\c{t}ia unic\u{a} \(x\) a ecuatiei \ref{1.2}. Pe de alta parte, avem pentru orice \(t\in \left [ a,b \right ]\)

\begin{equation} \nonumber
    \begin{split}
        x_{1}\left ( t \right ) &= f\left ( t \right ) + \int_{a}^{t}k\left ( t,s \right )f\left ( s \right )ds, \\
        x_{2}\left ( t \right ) &  = f\left ( t \right ) + \int_{a}^{t}k\left ( t,s \right )\left [ f\left ( s \right ) + \int_{a}^{s} k\left ( s,  \tau  \right )f\left (\tau  \right )d\tau  \right ]ds \\
        & =  f\left ( t \right ) + \int_{a}^{t}k\left ( t,s \right )f\left ( s \right )ds + \int_{a}^{t}\int_{a}^{s}k\left ( t,s \right )k\left ( s,\tau  \right )f\left (\tau  \right ) d\tau ds.
    \end{split}
\end{equation}

\noindent Putem schimba integrarea pentru a afla vedea c\u{a} ultima integral\u{a} este egal\u{a} cu
\begin{displaymath}
\int_{a}^{t}\left [ \int_{\tau }^{t}k\left ( t,\tau  \right )k\left ( s,\tau  \right )ds \right ]f\left (\tau   \right )d\tau,
\end{displaymath}
deci prin simpla reetichetare a lui  \(\tau\) \c{s}i \(s\) vom avea
\begin{displaymath}
\int_{a}^{t}\underbrace{\left [ \int_{s}^{t}k\left ( t,\tau  \right )k\left ( \tau ,s \right )d\tau  \right ]} _ {=: k_{2}\left ( t,s \right )} f\left ( s \right )d
\end{displaymath}
\c{s}i vom avea un nucleu nou,  \(k_{2}\). \^{I}n general, dac\u{a} \^{i}l vom lua  \(n = 2,3,…\)
\begin{equation} \nonumber
    \begin{split}
       k_{n}\left ( t,s \right ) := \int_{s}^{t}k\left ( t,\tau  \right )k_{n-1}\left ( \tau ,s \right )d\tau, \\
        k_{1}\left ( t,s \right ) := k\left ( t,s \right ),
    \end{split}
\end{equation}

\noindent avem, pentru \(n = 1,2,…..\)
\begin{displaymath}
x_{n}\left ( t \right ) = f\left ( t \right ) + \int_{a}^{t}\left [ \sum_{j= 1}^{n}k_{j}\left ( t,s \right ) \right ]f\left ( s \right )ds. \label{eq:2.6} \tag{2.6}
\end{displaymath}
Deoarece \(k\) este continuu pe mul\c{t}imea compact\u{a} \(\Delta\), avem pentru orice \(\left ( t,s \right ) \in \Delta\),
\begin{equation} \nonumber
    \begin{split}
        \left | k_{1}\left ( t,s \right ) \right |&\leq K < \infty \\
        \left | k_{2}\left ( t,s \right ) \right |&\leq K^{2}\left ( t-s \right ), \\
        \left | k_{3}\left ( t,s \right ) \right | & \leq K^{3}\int_{s}^{t}\left | \tau - s \right |d\tau \\
        & = K^{3}\frac{\left ( t-s \right )^{2}}{2!}, \\
        \vdots \\
        \left | k_{n}\left ( t,s \right ) \right | & \leq K^{n}\frac{\left ( t-s \right )^{n-1}}{\left ( n-1 \right )!} \\ & \leq K^{n}\frac{\left ( b-a \right )^{n-1}}{\left ( n-1 \right )!}.
    \end{split}
\end{equation}



\noindent Din M-testul lui Weierstrass, seria \(\sum_{n=1}^{\infty }k_{n}\left ( t,s \right )\) converge uniform pe \(\Delta\) deoarece
\begin{displaymath}
\sum_{n=1}^{\infty }\frac{K^{n}\left ( b-a \right )^{n-1}}{\left ( n-1 \right )!}< \infty.
\end{displaymath}

\noindent Ca urmare
\begin{displaymath}
R\left ( t,s \right ) = \sum_{n=1}^{\infty }k_{n}\left ( t,s \right ),
\end{displaymath}
este in \(C\left ( \Delta  \right ).\) Lu\^{a}nd \(n\rightarrow \infty\) \^{i}n \ref{eq:2.6} deducem faptul c\u{a}
\begin{displaymath}
x\left ( t \right ) = f\left ( t \right ) + \int_{a}^{t}R\left ( t,s \right )f\left ( s \right )ds, t\in \left [ a,b \right ]. \label{eq:2.7} \tag{2.7}
\end{displaymath}
Vom lua  \(R\left ( t,s \right )\) nucleul rezolvent. Acesta depinde de \(k\) dar este independent de \(f\), astfel \^{i}nc\^{a}t odat\u{a} ce g\u{a}sim \(R\left ( t,s \right )\) avem solu\c{t}ia \ref{1.2} pentru orice \(f\) (conform \ref{eq:2.7}).

\noindent Oberv\u{a}m faptul c\u{a}
\begin{displaymath}
\sum_{n=2}^{N+1}k_{n}\left ( t,s \right ) = \int_{s}^{t}k\left ( t,\tau  \right )\sum_{n=2}^{N+1}k_{n-1}\left ( \tau ,s \right )d\tau,
\end{displaymath}
ceea ce implic\u{a}
\begin{displaymath}
-k\left ( t,s \right ) + \sum_{n=1}^{N+1}k_{n}\left ( t,s \right ) = \int_{s}^{t}k\left ( t,\tau  \right )\sum_{n=1}^{N}k_{n}\left ( \tau ,s \right )d\tau .
\end{displaymath}
Dac\u{a} lu\u{a}m  \(n\rightarrow \infty\) vom constata c\u{a} \(R\) satisface
\begin{displaymath}
R\left ( t,s \right ) = k\left ( t,s \right ) + \int_{s}^{t}k\left ( t,\tau  \right )R\left ( \tau ,s \right )d\tau , \forall \left ( t,s \right )\in \Delta ,
\end{displaymath}
care este o ecua\c{t}ie Volterra similar\u{a} cu \ref{1.2}.

\noindent Acum s\u{a} examin\u{a}m ecua\c{t}ia \ref{eq:2.1}. Presupunem c\u{a}
\begin{displaymath}
f\in C^{1}\left [ a,b \right ] \c{s}i k,~ \frac{\partial k}{\partial t}\in C\left ( \Delta  \right ), k\left ( t,t \right )\neq 0,~ \text{pentru orice}~  t \in \left [ a,b \right ]. \label{eq:H} \tag{H}
\end{displaymath}

Presupunem de asemenea c\u{a} \(f\left ( a \right ) = 0\) care este o condi\c{t}ie necesar\u{a} pentru ca ecua\c{t}ia \ref{eq:2.1} s\u{a} admit\u{a} solu\c{t}ie. Dac\u{a} ecua\c{t}ia \ref{eq:2.1} are o solu\c{t}ie \(x\in C\left [ a,b \right ]\), atunci deriv\^{a}nd ecua\c{t}ia \ref{eq:2.1} deducem
\begin{displaymath}
{f}'\left ( t \right ) = \frac{d}{dt}\int_{a}^{t}k\left ( t,s \right )x\left ( s \right )ds \label{eq:2.8} \tag{2.8}
\end{displaymath}
\begin{displaymath}
= k\left ( t,t \right )x\left ( t \right ) + \int_{a}^{t} k_{t}\left ( t,s \right )x\left ( s \right )ds, t\in \left [ a,b \right ],
\end{displaymath}
care este echivalent\u{a} cu urm\u{a}toarea ecua\c{t}ie integral\u{a} de tip Volterra de spe\c{t}a a doua
\begin{displaymath}
x\left ( t \right ) = \frac{{f}'\left ( t \right )}{k\left ( t,t  \right )} + \int_{a}^{t} \left [ \frac{-k_{t}\left ( t,s \right )}{k\left ( t,t \right )} \right ]x\left ( s \right )ds. \label{eq:2.9} \tag{2.9}
\end{displaymath}
Deci x este  de asemenea  o solu\c{t}ie a ecua\c{t}iei \ref{eq:2.9}. Pe de alt\u{a} parte, \c{s}tim din teorema anterioar\u{a} faptul c\u{a} ecua\c{t}ia \ref{eq:2.9} are o solu\c{t}ie unic\u{a} \(x\in C\left [ a,b \right ]\). Acest x  este  de asemenea  o solu\c{t}ie a ecua\c{t}iei \ref{eq:2.1}. Aceasta rezult\u{a} din integrarea lui \ref{eq:2.8} pe \(\left [ a,t \right ]\) \c{s}i folosind condi\c{t}ia \(f\left ( a \right ) = 0\). Astfel am demonstrat urm\u{a}torul rezultat


\begin{theorem}
\^{I}n ipotezele \ref{eq:H} de mai sus \c{s}i, \^{i}n plus, \(f\left ( a \right ) = 0\), ecua\c{t}ia \ref{eq:2.1} are o solu\c{t}ie unic\u{a} \(x \in C\left [ a,b \right ]\).
\end{theorem}
\noindent Continu\u{a}m cu ecua\c{t}ia Volterra neliniar\u{a}
\begin{displaymath}
x\left ( t \right ) = f\left ( t \right ) + \int_{a}^{t}k\left ( t,s,x\left ( s \right ) \right )ds, t\in \left [ a,b \right ], \label{eq:2.10} \tag{2.10}
\end{displaymath}
\c{s}i demonstr\u{a}m urm\u{a}torul rezultat general.


\begin{theorem}
Presupunem c\u{a} \(f\in C\left [ a,b \right ]\), \(k\in C\left ( D \right )\), unde
\begin{displaymath}
D:= \Delta \times \mathbb{R} = \left \{ \left ( t,s,v \right )\in \mathbb{R}^{3}; a\leq s\leq t\leq b, v\in \mathbb{R}\right \}.
\end{displaymath}
Dac\u{a} exist\u{a} \(K> 0\) astfel \^{i}nc\^{a}t
\begin{displaymath}
\left | k\left ( t,s,v \right )  - k\left ( t,s,w \right )\right |\leq K\left | v-w \right |\forall a\leq s\leq t\leq b; v,w\in \mathbb{R}, \label{eq:2.11} \tag{2.11}
\end{displaymath}
atunci exist\u{a} o unic\u{a} func\c{t}ie  \(x\in C\left [ a,b \right ]\) care satisfice ecua\c{t}ia \ref{eq:2.10} pe \(\left [ a,b \right ]\).
\end{theorem}

\begin{proof}

Fie \(X = C \left [ a,b \right ]\)  cu norma Bielecki \c{s}i definim \(T : X \rightarrow X\) prin
\begin{displaymath}
\left ( Tg \right )\left ( t \right ) = f\left ( t \right ) + \int_{0}^{t}k\left ( t,s,g\left ( s \right ) \right )ds, \forall t \left [ a,b \right ], g\in X.
\end{displaymath}
Concluzia rezult\u{a} din Principiului contrac\c{t}iilor al lui Banach \^{i}n mod similar ca \^{i}n demonstara\c{t}ia 3 a teoremei 12.
\end{proof}
\noindent Teorema 14 ofer\u{a} o solu\c{t}ie global\u{a} \^{i}n sensul c\u{a} intervalul de existen\c{t}\u{a} este \^{i}ntregul interval \(\left [ a,b \right ]\). \^{I}n mod evident, aceasta este o generalizare a teoremei 12.

\^{I}ntr-adevar, pentru a ob\c{t}ine teorema 12 este suficient s\u{a} presupunem c\u{a} nucleul k este liniar \^{i}n raport cu a treia variabil\u{a}, adic\u{a} \(k:= k\left ( t,s \right )v, a\leq s\leq t\leq b, v\in \mathbb{R}\), cu \(k \in C\left ( \Delta  \right )\) astfel \^{i}nc\^{a}t condi\c{t}ia Lipschitz \ref{eq:2.1} este satisfacut\u{a} automat.

\noindent Acum s\u{a} examin\u{a}m un caz \^{i}n care solutia rezultat\u{a} este doar una local\u{a}, adic\u{a} domeniul s\u{a}u poate s\u{a} nu fie \^{i}ntregul interval \(\left [ a,b \right ]\).


\begin{theorem}
Presupunem c\u{a} \(f\in C\left [ a,b \right ], k = k\left ( t,s,v \right ) \in C\left ( D \right )\), unde \(D := \Delta \times \left [ x_{0} - c, x_{0} + c \right ] = \left \{ \left ( t,s,v \right ) \in \mathbb{R}^{3} ; a \leq s\leq t\leq b, \left | v - x_{0} \right |\leq c\right \}\), cu \(x_{0} \in \mathbb{R}\) \c{s}i \(c\in \left ( 0, \infty  \right )\). Dac\u{a} \^{i}n plus, exist\u{a}  \(K > 0\) astfel \^{i}nc\^{a}t
\begin{displaymath}
\left | k\left ( t,s,v \right ) - k\left ( t,s,w \right )\right | \leq K \left | v - w \right | \forall \left ( t,s,v \right ), \left ( t,s,w \right ) \in D \label{eq:2.12} \tag{2.12}
\end{displaymath}
\c{s}i exist\u{a} un \(d \in \left [ 0,c \right )\) astfel \^{i}nc\^{a}t
\begin{displaymath}
\left | f\left ( t \right ) - x_{0}\right | \leq d, \forall t \in \left [ a,b \right ], \label{eq:2.13} \tag{2.13}
\end{displaymath}
atunci exist\u{a} o unic\u{a} func\c{t}ie  \(x\in C \left [ a, a + \delta  \right ]\) care satisface ecua\c{t}ia \ref{eq:2.10} \^{i}n  \(\left [ a, a + \delta  \right ]\), unde
\begin{displaymath}
\delta  = min \left \{ b-a, \frac{\left ( c-d \right )}{M} \right \}, M = sup \left \{ \left | k\left ( t,s,v \right ) \right |;\left ( t,s,v \right ) \in D \right \}.
\end{displaymath}
(Se presupune c\u{a} M este pozitiv deoarece cazul M = 0 este trivial)
\end{theorem}

\begin{proof}

Se consider\u{a} spa\c{t}iul \(C \left [ a, a + \delta  \right ]\) cu norma sup uzual\u{a} \c{s}i metrica d generat\u{a} de aceasta. Not\u{a}m
\begin{displaymath}
Y = \left \{ g\in C \left [ a,a+\delta  \right ] ; \left | g\left ( t \right ) - x_{0}\right | \leq c, \forall t \in \left [ a, a+\delta  \right ]\right \}.
\end{displaymath}

\noindent \^{I}n mod clar \(\left ( Y,d \right )\) este un spa\c{t}iu metric complet (deoarece moment ce Y este o submul\c{t}ime \^{i}nchis\u{a} a spa\c{t}iului \(\left ( C \left [ a, a + \delta  \right ] , d \right )\)). Ca de obicei, definim un operator T prin
\begin{displaymath}
\left ( Tg \right )\left ( t \right ) = f\left ( t \right ) + \int_{a}^{t}k\left ( t,s,g\left ( s \right ) \right )ds, t\in \left [ a, a + \delta  \right ], g\in Y.
\end{displaymath}
					
\noindent S\u{a} ar\u{a}t\u{a}m c\u{a}  \(T\) are imaginea \^{i}n \(Y\). \^{I}ntr-adeva\u{a}, pentru orice \(g\in Y\) si \(t\in \left [ a, a+ \delta  \right ]\) vom avea (vezi \ref{eq:2.13}).
\begin{equation} \nonumber
    \begin{split}
      \left | \left ( Tg \right )\left ( t \right ) - x_{0}\right | &   \leq \left | f\left ( t \right )-x_{0} \right | + \int_{a}^{t}\left | k\left ( t,s,g\left ( s \right ) \right ) \right |ds \\ & \leq d + M\left ( t-a \right ) \\ & \leq  d+ M\delta  \leq  c,
    \end{split}
\end{equation}

\noindent ceea ce dovede\c{s}te afirma\c{t}ia. Prin argumente similar cu cele utilizate \^{i}n demonstra\c{t}ia 2 a teoremei 12 deducem c\u{a} \(T^{k}\) este o contrac\c{t}ie pe \(\left ( Y,d \right )\) pentru \(k\) suficient de mare. Deci \(T\) are un punct fix unic \(x \in Y\) care este solu\c{t}ia unic\u{a} a ecua\c{t}iei \ref{eq:2.10} \^{i}n \(\left [ a, a + \delta  \right ]\).
\end{proof}

\noindent Un alt rezultat de existen\c{t}\u{a} \c{s}i unicitate se obtine dac\u{a} nucleul k este definit pe un domeniu diferit,
\begin{displaymath}
\tilde{D} = \left \{ \left ( t,s,v \right ) \in \mathbb{R}^{3}; a\leq s\leq t\leq b, \left | v - f\left ( s \right ) \right | \leq c \right \}, c\in \left ( 0, \infty  \right ),
\end{displaymath}
care este o submul\c{t}ime compact\u{a} a lui \(\mathbb{R}^{3}\).
\begin{theorem}
Presupunem c\u{a} \(f \in C \left [ a,b \right ]\) \c{s}i \(k = k\left ( t,s,v \right ) \in C\left ( \tilde{D} \right )\),  \(M = sup _{\tilde{D}}\left | k \right |> 0\). Dac\u{a}, \^{i}n plus, exist\u{a}  \(K > 0\) astfel \^{i}nc\^{a}t
\begin{displaymath}
\left | k\left ( t,s,v \right ) - k \left ( t,s,w \right ) \right | \leq K \left | v-w \right |, \forall \left ( t,s,v \right ) \in \tilde{D}, \label{eq:2.14} \tag{2.14}
\end{displaymath}
atunci exist\u{a} o unic\u{a} func\c{t}ie \(x \in C \left [ a, a + \delta  \right ]\) care satisfice ecua\c{t}ia \ref{eq:2.10} \^{i}n  \(\left [ a, a + \delta  \right ]\), unde \(\delta = min \left \{ b-a, \frac{c}{M} \right \}\).
\end{theorem}


\begin{proof}

Demonstra\c{t}ia este similar\u{a} cu cea a teoremei 15  de mai sus. Aici domeniul operatorului T este
\begin{displaymath}
\tilde{Y} = \left \{ g \in C \left [ a, a+ \delta  \right ] ; \left | g\left ( t \right ) - f\left ( t \right ) \right | \leq c, \forall t \in \left [ a, a+ \delta  \right ] \right \},
\end{displaymath}
care este bila \^{i}nchis\u{a} \^{i}n \(\left ( C\left [ a, a+ \delta  \right ], d \right )\) centrat\u{a} \^{i}n \(f\) (restric\c{t}ionat\u{a} la \(\left [ a, a+ \delta  \right ]\)) de raz\u{a} \(c\).

\^{I}n mod evident  \(T\) este bine definit pe \(\tilde{Y}\) \c{s}i ia valori \^{i}n \(\tilde{Y}\). De asemenea, se verific\u{a} cu usurin\c{t}\u{a} c\u{a} \(T^{k}\) este o contrac\c{t}ie pentru orice \(k \in \mathbb{N}\)  suficient de mare. Aceasta completeaz\u{a} demonstra\c{t}ia (vezi observa\c{t}ia 3).
					\end{proof}







\section{Ecua\c{t}ii de tip Fredholm}

\^{I}n cele ce urmeaz\u{a} $\mathbb{K}$ este $\mathbb{R}$ sau $\mathbb{C}$. Consider\u{a}m \^{i}n \(\mathbb{K}\) ecua\c{t}ia integral\u{a}

\begin{displaymath}
x\left ( t \right ) = f\left ( t \right ) + \int_{a}^{b}k\left ( t,s \right )x\left ( s \right )ds, t\in \left [ a,b \right ], \label{eq:2.16} \tag{2.16}
\end{displaymath}
unde \(a,b \in \mathbb{R}, a< b, f\in C\left ( \left [ a,b \right ]; \mathbb{K}\right )\) \c{s}i \(k\in C\left ( \left [ a,b \right ] \times \left [ a,b \right ]; \mathbb{K}\right ).\)

 Ecua\c{t}ia (\ref{eq:2.16} este cunoscut\u{a} ca ecua\c{t}ia Fredholm (uneori este numit\u{a} de spe\c{t}a a doua). Ea implic\u{a} un interval fix de integrare \c{s}i este fundamental diferit\u{a} de ecua\c{t}ia \ref{1.2}.  O prima observa\c{t}ie care confirm\u{a} aceast\u{a} afirma\c{t}ie este c\u{a}, \^{i}n timp ce ecua\c{t}ia Volterra corespunz\u{a}toare (de spe\c{t}a a doua) are \^{i}ntotdeauna o solu\c{t}ie (unic\u{a}, continu\u{a}) pe \(\left [ a,b \right ]\), ecua\c{t}ia \ref{eq:2.16} poate s\u{a} nu aib\u{a} solu\c{t}ie \^{i}n unele cazuri.

 De exemplu, presupun\^{a}nd c\u{a} exist\u{a} o solu\c{t}ie \(x \in C\left [ 0,1 \right ] := C\left ( \left [ 0,1 \right ]; \mathbb{R} \right )\) a ecua\c{t}iei
\begin{displaymath}
x\left ( t \right ) = t + \int_{0}^{1} k\left ( t,s \right )x\left ( s \right )ds, t\in \left [ 0,1 \right ], \label{eq:2.17} \tag{2.17}
\end{displaymath}
unde
\begin{displaymath}
k\left ( t,s \right )=\left\{\begin{matrix}
\pi ^{2} \left ( s \right )\left ( 1-t \right ) , & s\leq t\\
& \\ \pi ^{2}t\left ( 1-s \right ),  & t\leq s
\end{matrix}\right.
\end{displaymath}
prin derivarea de dou\u{a} ori a ecua\c{t}iei \ref{eq:2.17}  rezult\u{a} c\u{a} x ar trebui s\u{a} satisfac\u{a} problema

\begin{displaymath}
\left\{\begin{matrix}
{x}'' \left ( t \right ) + \pi ^{2}x\left ( t \right )  = 0, & t \in \left [ 0,1 \right ] \\
x\left ( 0 \right )  =  0, x \left ( 1 \right ) = 1.&
\end{matrix}\right.
\end{displaymath}
Pe de alt\u{a} parte, se vede u\c{s}or c\u{a}  aceast\u{a} problem\u{a} nu are nicio solu\c{t}ie. Prin urmare, ecua\c{t}ia \ref{eq:2.17} nu are solu\c{t}ie.

Merit\u{a} totu\c{s}i subliniat faptul c\u{a} \^{i}n baza ipotezelor de mai sus, ecua\c{t}ia \ref{eq:2.16} are o solu\c{t}ie unic\u{a} \^{i}n \(C\left [ a,b \right ]\) ori de c\^{a}te ori norma sup a lui \(\left | k \right |\) este suficient de mic\u{a}, mai precis, dac\u{a} \(\left ( b-a \right )sup_{\left [ a,b \right ]\times \left [ a,b \right ]}\left | k \right | < 1.\) Acest rezultat se verific\u{a} cu usurin\c{t}\u{a} folosind Principiul Contrac\c{t}iilor al lui Banach. De fapt, problema existen\c{t}ei poate fi discutat\u{a} \^{i}n spa\c{t}iul \(L^{2} \left ( a,b ;\mathbb{K} \right )\) care este un cadru mai larg de lucru. Mai exact, s\u{a} presupunem \(f\in L^{2}\left ( a,b;\mathbb{K} \right ), k\in L^{2}\left ( Q; \mathbb{K} \right ),\) unde \(Q = \left ( a,b \right )\times \left ( a,b \right ).\)

\noindent Solu\c{t}ia $x$ a ecua\c{t}iei \ref{eq:2.16} va fi c\u{a}utat\u{a} \^{i}n \(L_{2}\left ( a,b;\mathbb{K} \right )\) care este un spa\c{t}iu Hilbert \^{i}n raport cu produsul scalar uzual \c{s}i norma indus\u{a} de acesta:
\begin{displaymath}
\left \langle g_{1}, g_{2} \right \rangle_{L^{2}} = \int_{a}^{b}g_{1}\left ( t \right ) \cdot \overline{{g_{2\left ( t \right )}}}dt,~~ \left \| g \right \|_{L^{2}}^{2} = \left \langle g,g \right \rangle.
\end{displaymath}
Desigur, dac\u{a} g\u{a}sim o solu\c{t}ie \(x\in L^{2}\left ( a,b;\mathbb{K} \right )\) a ecua\c{t}iei \ref{eq:2.16} cu \(f\in C\left ( \left [ a,b \right ];\mathbb{K} \right ), k\in C\left ( \left [ a,b \right ]\times \left [ a,b \right ];\mathbb{K} \right ),\) atunci evident \(x\in C\left ( \left [ a,b \right ];\mathbb{K} \right ).\)

\noindent Avem urm\u{a}torul rezultat.

\begin{theorem}
Dac\u{a} \[f\in L^{2}\left ( a,b;\mathbb{K} \right ), -\infty < a< b< +\infty , k\in L^{2}\left ( Q; \mathbb{K} \right )\] \c{s}i \(\int \int _{Q}\left | k\left ( t,s \right ) \right |^{2}dtds< 1,\) unde \(Q = \left ( a,b \right )\times \left ( a,b \right )\), atunci exist\u{a} o unic\u{a} func\c{t}ie \(x\in L^{2} \left ( a,b;\mathbb{K} \right ) \) care satisfice ecua\c{t}ia
\begin{displaymath}
x\left ( t \right ) = f\left ( t \right ) + \int_{a}^{b}k\left ( t,s \right )x\left ( s \right )ds,
\end{displaymath}
aproape peste tot \^{i}n \(\left ( a,b  \right )\).
\end{theorem}	
%
%
%	18/02/21
%
%

\begin{proof}

Fie T operatorul definit de
\begin{displaymath}
\left ( Tg \right )\left ( t \right ) = f\left ( t \right ) + \int_{a}^{b}k\left ( t,s \right )g\left ( s \right )ds,
 \forall g\in L^{2}\left ( a,b;\mathbb{K} \right ), ~~t\in \left ( a,b \right ).
\end{displaymath}
\noindent Se vede cu usurin\c{t}\u{a} c\u{a}   \(T\) ia valori \^{i}n \(L^{2}\left ( a,b;\mathbb{K} \right )\).

Mai mult, \(\left \| k \right \|_{L^{2}\left ( Q, \mathbb{K} \right )}< 1\), deci \(T\) este o contrac\c{t}ie \^{i}n raport cu metrica uzual\u{a} din \(\left \| \cdot  \right \|_{L^{2}}\). Prin urmare, exist\u{a} un unic punct fix  \(x\in L^{2}\left ( a,b;\mathbb{K} \right )\) care este solu\c{t}ia unic\u{a} \^{i}n \(L^{2}\) a ecua\c{t}iei
\begin{displaymath}
x\left ( t \right ) = f\left ( t \right ) + \int_{a}^{b}k\left ( t,s \right )x\left ( s \right )ds.
\end{displaymath}
\end{proof}

\begin{remark}
Folosind o procedur\u{a} similar\u{a} cu cea utilizat\u{a} pentru ecua\c{t}ia Volterra \ref{1.2}, putem deduce c\u{a} solu\c{t}ia dat\u{a} de teorema 17 poate fi reprezentat\u{a} prin formula
\begin{displaymath}
x\left ( t \right ) = f\left ( t \right ) + \int_{a}^{b}R\left ( t,s \right )f\left ( s \right )ds, pentru t\in \left ( a,b \right ),
\end{displaymath}
unde nucleul rezolvent R este dat de

\begin{displaymath}
R\left ( t,s \right ) = \sum_{i=1}^{\infty }k_{i}\left ( t,s \right ), \label{eq:2.18} \tag{2.18}
\end{displaymath}

\noindent cu

\begin{displaymath}
k_{1}\left ( t,s \right ):= k\left ( t,s \right ), k_{m}\left ( t,s \right ) = \int_{a}^{b}k\left ( t,\tau  \right )k_{m -1}\left ( \tau ,s \right )d\tau , \forall m\geq 2.
\end{displaymath}

\noindent Seria din \ref{eq:2.18} converge \^{i}n \(L^{2}\left ( Q,\mathbb{K} \right )\) \c{s}i aproape peste tot pe \(Q\).

\end{remark}					

\begin{remark}

Teorema 17 poate fi extins\u{a} la ecua\c{t}ia neliniar\u{a} de tip Fredholm
\begin{displaymath}
x\left ( t \right ) = f\left ( t \right ) + \int_{a}^{b}k\left ( t,s,x\left ( s \right ) \right )ds, t\in \left [ a,b \right ]. \label{eq:2.19} \tag{2.19}
\end{displaymath}
\^{I}ntr-adev\u{a}r, dac\u{a} \(f\in L^{2}\left ( a,b;\mathbb{K} \right ), k:Q\times \mathbb{K}\rightarrow \mathbb{K}\) este m\u{a}surabil Lebesque,  \(k\left ( \cdot ,\cdot ,0 \right )\in L^{2}\left ( Q; \mathbb{K} \right ) \) \c{s}i
\begin{displaymath}
\left | k\left ( t,s,v \right ) - k\left ( t,s,w \right ) \right |\leq \alpha \left ( t,s \right )\left | v - w \right |,~ \text{pentru orice}~ v,w \in \mathbb{K}~ \c{s}i~ \left ( t,s \right ) \in Q,
\end{displaymath}
pentru un \(\alpha \in L^{2}\left ( Q \right )\) cu \(\left \| \alpha  \right \|_{L^{2}\left ( Q \right )} < 1\), atunci exist\u{a} un unic \(x\in L^{2}\left ( a,b; \mathbb{K} \right )\) care satisface ecua\c{t}ia\ref{eq:2.19} aproape peste tot \^{i}n \(\left (a,b  \right )\).

Ca \c{s}i \^{i}n celelalte cazuri considerate, concluzia rezult\u{a} din Principiul contrac\c{t}iilor al lui Banach. S\u{a} observ\u{a}m doar c\u{a} pentru orice \(g\in L^{2}\left ( a,b; \mathbb{K} \right )\), func\c{t}ia \(\left (t,s  \right )  \mapsto k\left ( t,s,g\left ( s \right ) \right )\) apar\c{t}ine lui \(L^{2}\left ( Q, \mathbb{K} \right )\) deoarece
\begin{displaymath}
\left | k\left ( t,s,g\left ( s \right ) \right ) \right | \leq \left | k\left ( t,s,0 \right ) \right | + \alpha \left ( t,s \right )\left | g\left ( s \right ) \right |, pentru \left ( t,s \right )\in Q.
\end{displaymath}
\end{remark}

\begin{remark}
\^{I}n cazul ecua\c{t}iilor Fredholm, conceptul de solu\c{t}ie local\u{a} nu are sens deoarece termenul integral implic\u{a} valorile \(x\left ( t \right )\) pentru \(t\in \left ( a,b \right )\). Acest lucru arat\u{a} \^{i}nc\u{a} o dat\u{a} faptul c\u{a} ecua\c{t}iile Fredholm sunt fundamental diferite de ecua\c{t}iile Volterra.

\noindent Pe de alt\u{a} parte, ne putem \^{i}ntreba dac\u{a} ecua\c{t}ia \ref{eq:2.16} mai are solu\c{t}ii atunci c\^{a}nd condi\c{t}ia \(\left \| k \right \|_{L^{2}\left ( Q;\mathbb{K} \right )}< 1\) nu mai este \^{i}ndeplint\u{a}. Un r\u{a}spuns complet este dat de alternativa lui Fredholm (Vezi observa\c{t}ia 6). \^{I}n cazul nostru particular, \(H = L^{2}\left ( a,b;\mathbb{K} \right )\) \c{s}i \(A : H \rightarrow H\) dat de
\begin{displaymath}
\left ( Ag \right )\left ( t \right ) = \int_{a}^{b}k\left ( t,s \right )g\left ( s \right )ds, \forall g\in H, pentru t\in \left ( a,b \right ). \label{eq:2.20} \tag{2.20}.
\end{displaymath}
\end{remark}

\noindent \^{I}n mod clar, \(A \in L\left ( H \right )\).

\begin{lemma}
Dac\u{a} \(k\in L^{2}\left ( Q;\mathbb{K} \right )\), atunci operatorul \(A : H \rightarrow H\) definit de \ref{eq:2.20} este compact.
\end{lemma}


\begin{proof}

S\u{a} presupunem mai \^{i}nt\^{a}i c\u{a} nucleul \(k \in C\left ( \left [ a,b \right ]  \times \left [ a,b \right ]; \mathbb{K}\right )\). Pentru a ar\u{a}ta c\u{a} \(A\) este compact  vom folosi criteiul Arzelà-Ascoli (Vezi Capitolul 1) \c{s}i observ\u{a}m c\u{a} criteriul este valabil cu \(\mathbb{K}\) \^{i}n loc de \(\mathbb{R}^{k}\).

Fie \(B\left ( 0,r \right ), r\in \left ( 0,\infty  \right )\), o bil\u{a} \^{i}n \(H\). Atunci mul\c{t}imea \(F = \left \{ Ag; g\in B\left ( 0,r \right ) \right \}\) este o submul\c{t}ime m\u{a}rginit\u{a} a lui \(C\left ( \left [ a,b \right ];\mathbb{K}  \right )\):
\begin{equation} \nonumber
    \begin{split}
    \left | \left ( Ag \right )\left ( t \right ) \right | & \leq \int_{a}^{b}\left | k\left ( t,s \right ) \right |\cdot \left | g\left ( s \right ) \right |ds  \\ & \leq \left ( \int_{a}^{b} \left | k\left ( t,s \right ) \right |^{2}ds \right )^{\frac{1}{2}}\left \| g \right \|_{L^{2}} \\ & \leq r\left ( b-a \right )^{\frac{1}{2}}\sup_{Q}\left | k \right | < \infty,
   \end{split}
\end{equation}

\noindent pentru orice \(g \in B\left ( 0,r \right )\) \c{s}i orice \(t\in \left [ a,b \right ]\).

Mul\c{t}imea \( \mathcal{F}\) este de asemenea echicontinu\u{a} deoarece  \(k\) este uniform continuu pe \(\left [ a,b \right ] \times \left [ a,b \right ]\), deci (din Criteriul Arzela-Ascoli) \( \mathcal{F}\) este relativ compact \^{i}n \(C\left ( \left [ a,b \right ];\mathbb{K} \right )\), deci \c{s}i \^{i}n \(H = L^{2}\left ( a,b;\mathbb{K} \right )\).
Prin urmare, \(A\) este \^{i}ntr-adev\u{a}r un operator compact.

Acum, presupunem c\u{a} \(k \in L^{2}\left ( Q,\mathbb{K} \right )\).  Atunci, din teorema 11, exist\u{a} un \c{s}ir \(\left (k_{n}  \right )\) \^{i}n \(C\left ( \left [ a,b \right ] \times  \left [ a,b \right ];\mathbb{K} \right )\) astfel \^{i}nc\^{a}t \(\left \| k_{n} -k \right \|_{L^{2}\left ( Q;\mathbb{K} \right )}\rightarrow 0\)  pentru ca \(n\rightarrow \infty\). S\u{a} asociem cu fiecare \(k_{n}\) operatorul \(A_{n} \in L\left ( H \right )\) definit de
\begin{displaymath}
\left (A_{n}g \right )\left ( t \right ) = \int_{a}^{b}k_{n}\left ( t,s \right )g\left ( s \right )ds, \forall g\in H, t\in \left [ a,b \right ],
\end{displaymath}
care este compact, conform argumentului de mai sus.

Un calcul simplu arat\u{a} c\u{a} \(\left \| A_{n}-A \right \|_{L\left ( H \right )}\leq \left \| k_{n}  -k\right \|_{L^{2}\left ( Q,\mathbb{K}  \right )}\) pentru orice n, doarece   \(\left \| A_{n}-A \right \|_{L\left ( H \right )}\rightarrow 0\) pentru  \(n \rightarrow \infty\). Din teorema 3 rezult\u{a} c\u{a} \(A\) este compact.
\noindent
\end{proof}

\noindent Se consider\u{a} ( \^{i}n \( \mathbb{K}\) ) ecua\c{t}ia
\begin{displaymath}
x\left ( t \right ) = f\left ( t \right ) + \lambda \int_{a}^{b}k\left ( t,s \right )x\left ( s \right )ds, t\in \left [ a,b \right ], \label{eq:2.21} \tag{2.21}
\end{displaymath}
\noindent unde \(\lambda \in \mathbb{K}, f\in L^{2}\left ( a,b; \mathbb{K} \right ), k\in L^{2} \left ( Q,\mathbb{K} \right ), Q = \left ( a,b \right )\times \left ( a,b \right )\). Conform teoremei 8, ecua\c{t}ia \ref{eq:2.21} are o solu\c{t}ie unic\u{a} \^{i}n \(L^{2}\left ( a,b;\mathbb{K} \right )\) cu condi\c{t}ia ca \(\left | \lambda  \right |\) sa fie suficient de mic. Mai precis, asta se \^{i}ntampl\u{a} dac\u{a}
\begin{displaymath}
\left | \lambda  \right | \cdot \left \| k \right \|_{L^{2}\left ( Q;\mathbb{K} \right )}< 1. \label{eq:2.22} \tag{2.22}
\end{displaymath}
\noindent Vom ar\u{a}ta \^{i}n cele ce urmeaz\u{a} c\u{a} exist\u{a} solu\c{t}ii pentru ecua\c{t}ia \ref{eq:2.21} chiar dac\u{a} \(\lambda\) nu satisfice condi\c{t}ia \ref{eq:2.22}.

Folosind nota\c{t}ia de mai sus putem scrie ecua\c{t}ia \ref{eq:2.21} ca o ecua\c{t}ie abstract\u{a} \^{i}n \(H = L^{2}\left ( a,b; \mathbb{K} \right )\), \c{s}i anume
\begin{displaymath}
x = f + \lambda Ax. \label{eq:2.23} \tag{2.23}
\end{displaymath}
\noindent Re\c{t}inem faptul c\u{a} \(A^{\ast }\), adjunctul lui \(A\), este dat de
\begin{displaymath}
\left (A^{\ast }h  \right )\left ( t \right ) = \int_{a}^{b}\overline{k\left ( s,t \right )}\cdot h\left ( s \right )ds \forall h \in H.
\end{displaymath}
\noindent \c{s}i de asemenea, \(\left ( \lambda A \right )^{\ast } = \overline{\lambda }A^{\ast }\).

\noindent Conform lemei 4 \c{s}i teoremei 8, operatorul A are o mul\c{t}ime num\u{a}rabil\u{a} de valori proprii,   \(0\) fiind singurul punct de acumulare posibil; \^{i}n plus, pentru orice valoare proprie \(\nu \neq 0\) a lui \(A\), \(\dim N\left ( I-\lambda A \right )> \infty\), unde \(\lambda = \frac{1}{\nu }\). Desigur, afirma\c{t}ii similare sunt valabile pentru \(A^{\ast }\), \^{i}n special \(\dim N\left ( I-\overline{\lambda }A^{\ast } \right )< \infty\). De fapt, putem demonstra c\u{a}, dac\u{a} \(\nu \neq 0\) este o valoare proprie a lui \(A\), atunci
\begin{displaymath}
\dim N\left ( I-\lambda A \right ) = \dim N\left ( I - \overline{\lambda } A^{\ast }\right ), \text{ unde } \lambda  = \frac{1}{\nu }. \label{eq:2.24} \tag{2.24}
\end{displaymath}
\^{I}n primul r\^{a}nd, re\c{t}inem faptul c\u{a} \(\overline{\nu }\) este o valoare proprie a lui \(A^{\ast }\) (conform teoremei 7), deci \(\dim N\left ( I - \overline{\lambda }A^{\ast } \right )\geq 1\).

Fie \(\left \{ \phi _{1}, \phi _{2},\cdots,\phi _{m} \right \}\) \c{s}i \(\left \{ \psi _{1}, \psi _{2},\cdots,\psi _{n} \right \}\)  bazele ortonomale \^{i}n \(N \left ( I-\lambda A \right )\) \c{s}i respectiv \(N \left ( I-\overline{\lambda} A^{\ast } \right )\). Presupunem prin absurd c\u{a} \(m< n\) . Fie \(B\) operatorul asociat cu nucleul
\begin{displaymath}
K\left ( t,s \right ) = k\left ( t,s \right ) - \sum_{j=1}^{m}\overline{\phi _{j}\left ( s \right )}\cdot \psi _{j} \left ( t \right )
\end{displaymath}
\c{s}i fie \(\phi , \psi  \in H\) solu\c{t}iile ecua\c{t}iilor
\begin{equation} \label{eq:2.25} \tag{2.25}
    \begin{split}
        \phi \left ( t \right ) & = \lambda \left ( B\phi  \right )\left ( t \right ) \\ & = \lambda \int_{a}^{b} k\left ( t,s \right )\phi \left ( s \right )ds - \lambda \sum_{j=1}^{m}\psi _{j}\left ( t \right )\int_{a}^{b}\overline{\phi _{j}\left ( s \right )}\cdot \phi \left ( s \right )ds,
    \end{split}
\end{equation}
\begin{equation} \label{eq:2.26} \tag{2.26}
    \begin{split}
        \psi  \left ( t \right )  & = \overline{\lambda} \left ( B^{\ast }\psi  \right )\left ( t \right )  \\ & = \overline{\lambda }\int_{a}^{b}\overline{k\left ( s,t \right )}\psi \left ( s \right )ds - \overline{\lambda }\sum_{j=1}^{m}\phi _{j}\left ( t \right )\int_{a}^{b}\overline{\psi _{j}\left ( s \right )}\cdot \psi \left ( s \right )ds.
    \end{split}
\end{equation}

\^{I}nmul\c{t}ind ecua\c{t}ia \ref{eq:2.25} cu \(\overline{\psi _{k}}\left ( t \right )\) \c{s}i apoi integr\^{a}nd pe \(\left [ a,b \right ]\) ecua\c{t}ia rezultat\u{a} ne conduce la
\begin{displaymath}
\left ( \phi ,\psi _{k} \right )_{L^{2}} = \int_{a}^{b}\underbrace{\left [ \lambda \int_{a}^{b}k\left ( t,s \right ) \cdot \overline{\psi _{k}\left ( t \right )}dt \right ]}_{\psi _{k}\left ( s \right )}\phi \left ( s \right )ds - \lambda \left ( \phi ,\phi _{k} \right )_{L^{2}} = \left ( \phi ,\psi _{k} \right )_{L^{2}} - \lambda \left ( \phi ,\phi _{k} \right )_{L^{2}}
\end{displaymath}
prin urmare
\begin{displaymath}
\left ( \phi ,\phi _{k} \right )_{L^{2}} = 0, k=1,2,\cdots,m. \label{eq:2.27} \tag{2.27}
\end{displaymath}
Din \ref{eq:2.25} \c{s}i \ref{eq:2.27} deducem faptul c\u{a} \(\phi \in N\left ( I-\lambda A \right )\). Astfel \(\phi = \sum_{i=1}^{m}c_{i}\phi _{i}\), cu \(c_{i}\in\mathbb{K}, i = 1,2,\cdots,m\). Acest lucru, combinat cu \ref{eq:2.27} ne conduce la \(\phi = 0\), prin urmare ecua\c{t}ia \ref{eq:2.25}) are doar solu\c{t}ia nul\u{a}.

Pe de alt\u{a} parte, ecua\c{t}ia \ref{eq:2.26} este satisfacut\u{a} de \(\psi _{k}\) pentru orice \(k\in \left \{ m+1,\cdots, n \right \}\).

\^{I}ntr-adevar, deoarece \(\left ( \psi _{k} , \psi _{j}\right )_{L^{2}} = 0\) pentru \(j \in \left \{ 1,\cdots,m \right \}\),  \(k\in \left \{ m+1,\cdots,n \right \}\), ecua\c{t}ia \ref{eq:2.26} cu \(\psi =\psi _{k}, k = m+1,\cdots,n\) poate fi scris\u{a} ca \(\psi _{k} = \overline{\lambda }A^{\ast }\psi _{k}, k = m+1,\cdots,n\). Acest lucru \^{i}nseamn\u{a} c\u{a} \(N\left ( I - \overline{\lambda }B^{\ast } \right ) = N\left ( I - \left (\lambda B   \right )^{\ast } \right )\neq \left \{ 0 \right \}\), \^{i}n timp ce \(N\left ( I - \lambda B \right ) = \left \{ 0 \right \}\), ceea ce contrazice teorema 7. Prin urmare, \(m\geq n\).

Inegalitatea invers\u{a} rezult\u{a} din faptul c\u{a} \(\left ( \overline{\lambda }A^{\ast } \right )^{\ast } = \lambda A\), deci demonstra\c{t}ia lui \ref{eq:2.24} este complet\u{a}.


Observ\u{a}m c\u{a} \^{i}n cazul ecua\c{t}iei \ref{eq:2.21}  alternativa lui Fredholm (vezi observa\c{t}ia 6) are urm\u{a}toarea form\u{a}:

\begin{theorem} (\textbf{Alternativa lui Fredholm})
Presupunem c\u{a} \(\lambda \in \mathbb{K}, f\in H = L^{2}\left (a,b;\mathbb{K} \right ), k\in L^{2}\left ( Q; \mathbb{K} \right )\), unde \(Q = \left ( a,b \right )\times \left ( a,b \right )\) \c{s}i fie \(A : H \rightarrow H\) operatorul definit de
\begin{displaymath}
\left (Ag  \right )\left ( t \right ) = \int_{a}^{b}k\left ( t,s \right )g\left ( s \right )ds, \forall g \in H, t \in \left ( a,b \right ).
\end{displaymath}

\noindent Atunci are loc una din urm\u{a}toarele afirma\c{t}ii:
\begin{itemize}
\item \(N\left ( I - \lambda A \right ) = \left \{ 0 \right \}\) (dac\u{a} \c{s}i numai dac\u{a} \(N\left ( I - \overline{\lambda} A^{\ast } \right ) = \left \{ 0 \right \}\)) \c{s}i \^{i}n acest caz ecua\c{t}ia
\begin{displaymath}
x\left ( t \right ) = f\left ( t \right ) + \lambda \int_{a}^{b}k\left ( t,s \right )x\left ( s \right )ds, t\in \left [ a,b \right ] \label{eq:F} \tag{F}
\end{displaymath}
        		      		      		
are o solu\c{t}ie unic\u{a} pentru orice \(f\in H\),
\item \(\dim N\left ( I - \lambda A \right ) = \dim N\left ( I - \overline{\lambda} A^{\ast } \right ) = m\), cu \(1\leq m\leq \infty\) \c{s}i \^{i}n acest caz ecua\c{t}ia \ref{eq:F} are solu\c{t}ie dac\u{a} \c{s}i numai dac\u{a}
\begin{displaymath}
\left ( f, \psi  \right )_{L^{2}} = \int_{a}^{b}f\left ( t \right )\cdot \overline{\psi \left ( t \right )}dt = 0, \forall \psi \in \ker \left ( I - \overline{\lambda}A^{\ast } \right ),
\end{displaymath}
(echivalent, \(\left ( f, \psi  \right )_{L^{2}} = 0, k\in \left \{ 1,2,\cdots,m \right \}\), unde \(\psi _{k}\) formeaz\u{a} o baz\u{a} ortonomal\u{a} \^{i}n \(N\left ( I - \overline{\lambda}A^{\ast } \right ))\).
\end{itemize}
\end{theorem}

\begin{remark}
Deoarece mul\c{t}imea \(S = \left \{\lambda \in \mathbb{K} ; N\left ( I - \lambda A \right ) = \left \{ 0 \right \}\right \}\) este numarabil\u{a}, rezult\u{a} din teorema 18 c\u{a} exist\u{a} ”multe” \(\lambda\) –uri care nu \^{i}ndeplinesc condi\c{t}ia \ref{eq:2.22}, dar  pentru care ecua\c{t}ia \ref{eq:F} are o (unic\u{a}) solu\c{t}ie pentru orice \(f\in H = L^{2}\left ( a,b;\mathbb{K} \right )\). Chiar \c{s}i pentru \(\lambda \in S\) ecua\c{t}ia \ref{eq:F} poate fi rezolvat\u{a} dac\u{a} \c{s}i numai dac\u{a} \(f \perp N\left ( I - \overline{\lambda }A^{\ast } \right )\).
\end{remark}

\textbf{Cazul nucleelor hermitiene: formula lui Schmidt }

\noindent Pe l\^{a}ng\u{a} condi\c{t}iile
\begin{displaymath}
f\in H = L^{2}\left ( a,b;\mathbb{K} \right ), k\in L^{2}\left ( Q;\mathbb{K} \right ), Q = \left ( a,b \right )\times \left ( a,b \right ),
\end{displaymath}
pe care le-am utilizat anterior, s\u{a} presupunem c\u{a} nucleul \(k\) este hermitian, adic\u{a},
\begin{displaymath}
k\left ( t,s \right ) = \overline{k\left ( s,t \right )}, \forall \left ( t,s \right ) \in Q.
\end{displaymath}
\^{I}n mod evident, \(A = A^{\ast }\). Conform propozi\c{t}iei 8.5, fiecare valoare proprie a lui A este real\u{a}.

\noindent \^{I}n continuare, \^{i}ncerc\u{a}m s\u{a} folosim teorema Hilbert-Schmidt pentru a investiga ecua\c{t}ia Fredholm \^{i}n forma sa abstract\u{a} \ref{eq:2.23}, adic\u{a},
\begin{displaymath}
x = f + \lambda Ax. \label{eq:A} \tag{A}
\end{displaymath}
De fapt, \^{i}n cele ce urmeaz\u{a}, \^{i}n ecua\c{t}ia \ref{eq:A}, operatorul A poate s\u{a} fie orice operator linear, simetric, compact, definit pe un spa\c{t}iu Hilbert separabil, \(\left ( H, \left ( \cdot ,\cdot  \right ), \left \| \cdot  \right \| \right )\), cu valori \^{i}n el \^{i}nsu\c{s}i \c{s}i \(f \in H\).

\noindent Ca un prim pas, s\u{a} presupunem c\u{a} \(N\left ( A  \right ) = \left \{ 0 \right \}\), adic\u{a} zero nu este valoare proprie a lui A. Astfel, Teormea Hilbert – Schimidt (teorema 9) se poate aplica lui \(A\) (vezi lema 4). Not\u{a}m cu \(\lambda _{1}, \lambda _{2}, \cdots,\lambda _{n},\cdots\) valorile proprii ale lui A date de aceast\u{a} teorem\u{a} \c{s}i cu \(u_{1},u_{2},\cdots,u_{n},\cdots\) vectorii proprii corespunz\u{a}tori, adic\u{a} \(Au_{n} = \lambda_{n}u_{n}, n = 1,2\cdots\).

Conform demonstra\c{t}iei teoremei Hilbert-ScHmidt, fiecare valoare proprie este luat\u{a} \^{i}n considerare de k-ori, unde k \^{i}nseamn\u{a} multiplicitatea sa (dimensiunea spa\c{t}iului propriu corespunz\u{a}tor). Sistemul \(\left \{ u_{n} \right \}_{n\geq 1}\) este o baz\u{a} ortonomat\u{a} \^{i}n \(H\).

\noindent Pentru \(\lambda \in \mathbb{K} - \left \{ 0 \right \}\) distingem dou\u{a} cazuri

\begin{enumerate}[(i)]
\item \(N\left ( I - \lambda A \right ) = \left \{ 0 \right \}\), adic\u{a} \(\frac{1}{\lambda}\) nu este o valoare proprie a lui \(A\);
\item \(N\left ( I - \lambda A \right ) \neq  \left \{ 0 \right \}\), adic\u{a} \(\frac{1}{\lambda}\) este o valoare proprie a lui A.
\end{enumerate}

\noindent S\u{a} discut\u{a}m mai \^{i}nt\^{a}i cazul (i).

Din observa\c{t}ia 6, ecua\c{t}ia \ref{eq:2.23} are o solu\c{t}ie unic\u{a} $x$
pentru orice \(f\in H\). Din formula 1.22 din demonstra\c{t}ia teoremei 9 (Teorema Hilbert – Schmidt) avem
\begin{displaymath}
Ax = \sum_{n=1}^{\infty }\lambda_{n}\left ( x,u_{n} \right )u_{n}. \label{eq:2.28} \tag{2.28}
\end{displaymath}
Pe de alt\u{a} parte, folosind ecua\c{t}ia \ref{eq:2.23} \c{s}i faptul c\u{a} \(A\) este simetric, ob\c{t}inem
\begin{displaymath}
\left ( x,u_{n} \right ) = \left ( f,u_{n} \right ) + \lambda \lambda _{n}\left ( x,u_{n} \right ), n = 1,2,\cdots,
\end{displaymath}
de unde
\begin{displaymath}
\left ( x,u_{n} \right ) = \frac{1}{1 - \lambda \lambda _{n}}\left ( f,u_{n} \right ), n = 1,2,\cdots \label{eq:2.29} \tag{2.29}
\end{displaymath}
Acum, din \ref{eq:2.23}), \ref{eq:2.28} \c{s}i \ref{eq:2.29}  putem deduce urm\u{a}toarea formul\u{a} pentru solu\c{t}ia $x$ a ecua\c{t}iei \ref{eq:2.23} (cunoscut\u{a} ca formula lui Schmidt)
\begin{displaymath}
x = f + \lambda \sum_{n=1}^{\infty }\frac{\lambda_{n}}{1 - \lambda \lambda_{n}}\left ( f, u_{n} \right )u_{n}. \label{eq:2.30} \tag{2.30}
\end{displaymath}
Acum, s\u{a} discut\u{a}m cazul (ii), adic\u{a}  \(\frac{1}{\lambda}\) este o valoare proprie a operatorului A; evident, \(\frac{1}{\lambda} = \lambda_{k}\) pentru un \(k \in \mathbb{N}\), deci formula \ref{eq:2.30} nu are sens \^{i}n acest caz.

\noindent Not\u{a}m \(H_{0}:= N\left ( I - \lambda A \right ) = N \left ( \lambda_{k}I - A, \right ) H_{1}:= H_{0}^{\perp }\), ca urmare \(H = H_{0}\oplus H_{1}\). Din teorema 8.4, \(H_{0}\) este de dimensiune finit\u{a}. Not\u{a}m cu \(m:= \dim H_{0}\in \mathbb{N}\). Fie \(B_{0} = \left \{ v_{1}, v_{2},\cdots,v_{m} \right \}\) o baz\u{a} a lui \(H_{0}\). Cum \(H\) este un spa\c{t}iu separabil, la fel este \c{s}i \(H_{1}\).

\noindent \c{T}in\^{a}nd cont de faptul c\u{a} A este simetric, se vede cu usurin\c{t}\u{a} c\u{a} A duce \(H_{1}\) \^{i}n \(H_{1}\). \^{I}n mod clar, restric\c{t}ia \(A_{1} = A|_{H_{1}}\) este simetric\u{a} \c{s}i \(A_{1} \in K\left ( H_{1} \right )\), adic\u{a} \(A_{1}\) este compact \^{i}n \(H_{1}\)  care este un subspa\c{t}iu Hilbert al lui H cu acela\c{s}i \(\left ( \cdot ,\cdot  \right )\) \c{s}i \(\left \| \cdot  \right \|\). Evident, \(N\left ( A_{1} \right ) = \left \{ 0 \right \}\) deci Teorema Hilbert-Schmidt se poate aplica lui \(H_{1}\) \c{s}i \(A_{1}\) \c{s}i arat\u{a} existen\c{t}a unui \c{s}ir de valori proprii (reale) ale lui \(A_{1}\) (deci ale lui A), care nu includ \(\lambda _{k}\) \c{s}i a unei baze ortonormale corespunzatoare \^{i}n \(H_{1}\), cu
\begin{displaymath}
A_{1}u_{n} = Au_{n} = \lambda _{n}u_{n}, n\in\mathbb{N}, n\neq k.
\end{displaymath}
Conform analizei anterioare corespunz\u{a}toare cazului (i), ecua\c{t}ia \ref{eq:2.23} are o solu\c{t}ie (unic\u{a}) \(x = x_{1}\) \^{i}n \(H_{1}\) ( adic\u{a}, \(x_{1} - \lambda A_{1}x_{1} = f\)) dac\u{a} \c{s}i numai dac\u{a} \(f\in H_{1}\) \c{s}i
\begin{displaymath}
x_{1} = f + \lambda \sum_{\lambda_{n}\neq \lambda_{k}}\frac{\lambda_{n}}{1 - \lambda \lambda _{n}}\left ( f, u_{n} \right )u_{n}.
\end{displaymath}

\noindent	Dac\u{a} lu\u{a}m \^{i}n considerare \ref{eq:2.23} \^{i}n H, atunci \(f\in H_{1}\) \c{s}i pentru orice \(y\in H_{0}\),
\begin{displaymath}
x = f+ \lambda \sum_{\lambda_{n}\neq \lambda_{k}}\frac{\lambda_{n}}{1 - \lambda \lambda _{n}}\left ( f, u_{n} \right )u_{n} + y
\end{displaymath}
este o solu\c{t}ie a ecua\c{t}iei \ref{eq:2.23}. \^{I}n consecin\c{t}\u{a}, formula
\begin{displaymath}
x = f+ \lambda \sum_{\lambda_{n}\neq \lambda_{k}}\frac{\lambda_{n}}{1 - \lambda \lambda _{n}}\left ( f, u_{n} \right )u_{n} + \sum_{i=1}^{m}c_{i}v_{i}, \label{eq:2.31} \tag{2.31}
\end{displaymath}
cu \(c_{1},....,c_{m} \in\mathbb{K}\) ne ofer\u{a} toate solu\c{t}iile ecua\c{t}iei \ref{eq:2.23}.

\noindent Acum ne \^{i}ndrept\u{a}m aten\c{t}ia asupra cazului \^{i}n care \(N\left ( A \right ) \neq \left \{  0\right \}\).

\noindent Not\^{a}nd \(Y_{0} = N\left ( A \right )\) \c{s}i \(Y_{1} = Y_{0}^{\perp }\), putem scrie \(H = Y_{0}\oplus Y_{1}\). Putem presupune c\u{a} \(Y_{0}\) este un subspa\c{t}iu propriu al lui H, \^{i}n caz contrar avem \(A = 0\) care este un caz trivial. Este u\c{s}or de observat c\u{a} \(A\) duce \(Y_{1}\) \^{i}n \(Y_{1}\). Evident, \(Y_{1}\) este un subspa\c{t}iu Hilbert al lui H fa\c{t}\u{a} de aceelea\c{s}i \(\left ( \cdot ,\cdot  \right )\) \c{s}i \(\left \| \cdot  \right \|\), iar restric\c{t}ia \(\tilde{A} = A|_{Y_{1}}\) este simetric\u{a}, compact\u{a} \c{s}i \(N\left ( \tilde{A} \right ) = \left \{ 0 \right \}\). Dac\u{a} \(Y_{1}\)  este infinit dimensional, atunci putem aplica teorema Hilbert – Schmidt lui \(Y_{1}\) \c{s}i \(\tilde{A}\). Pentru a rezolva ecua\c{t}ia \ref{eq:2.23} folosim descompunerea \(x = x_{0}+x_{1}, f = f_{0} + f_{1}\), unde \(x_{0},f_{0} \in Y_{0}\) \c{s}i \(x_{1},f_{1} \in Y_{1}\). Astfel \ref{eq:2.23} devine
\begin{displaymath}
x_{0} - f_{0} = -x_{1} + f_{1} + \lambda Ax_{1},
\end{displaymath}
prin urmare ambele p\u{a}r\c{t}i sunt egale cu \(0\), deci \(x_{0} = f_{0}\) \c{s}i
\begin{displaymath}
x_{1} = f_{1} + \lambda \tilde{A}x_{1}. \label{eq:2.32} \tag{2.32}
\end{displaymath}
\^{I}n mod clar, pentru fiecare \(f\in H, f = f_{0} + f_{1}, x\) este solu\c{t}ie unic\u{a} e ecua\c{t}iei \ref{eq:2.23} dac\u{a} \c{s}i numai dac\u{a} \(x = f_{0} + x_{1}\), unde \(x_{1} \in Y_{1}\) satisfice ecua\c{t}ia \ref{eq:2.32}.

\noindent Merit\u{a} subliniat faptul c\u{a} ecua\c{t}ia \ref{eq:2.32} , cu \(\tilde{A} : Y_{1}\rightarrow Y_{1}, N\left ( \tilde{A} \right ) = \left \{ 0 \right \}\), se afl\u{a} \^{i}n situa\c{t}ia pe care am avut-o mai \^{i}nainte, deci se poate discuta \^{i}n mod similar despre rezolvarea ecua\c{t}iei \ref{eq:2.32} folosind vectorii proprii ai lui \(\tilde{A}\) (adic\u{a} vectorii proprii ai lui A corespunz\u{a}tori la valori proprii nenule).

Dac\u{a} \(Y_{1}\) este finit dimensional, atunci ecu\c{t}ia \ref{eq:2.32} se reduce la un sistem algebraic linear care poate fi rezolvat folosind calcule algebrice elementare.

\begin{example}
Fie \(H = L^{2}\left ( -\pi ,\pi  \right )\) cu produsul scalar \c{s}i norma uzuale. Lu\u{a}m \^{i}n cosiderare baza ortonormal\u{a} obi\c{s}nuit\u{a} \^{i}n \(H\), adic\u{a}
\begin{displaymath}
u_{0} = \frac{1}{\sqrt{2\pi }}, u_{2k-1}\left ( t \right ) = \frac{1}{\sqrt{\pi }}\cos \left ( kt \right ),
\end{displaymath}
\begin{displaymath}
u_{2k}\left ( t \right ) = \frac{1}{\sqrt{\pi }}\sin\left ( kt \right ), k= 1,2,\cdots
\end{displaymath}
Pentru un \(m\in \mathbb{N}\) dat, definim
\begin{displaymath}
k\left ( t,s \right ) = \sum_{n=m}^{\infty }\frac{1}{n^{2}}u_{n}\left ( t \right )u_{n}\left ( s \right ), \left ( t,s \right )\in Q = \left ( -\pi ,\pi  \right )\times -\left ( -\pi ,\pi  \right ).
\end{displaymath}
\^{I}n mod clar, \(k\in C\left ( \bar{Q} \right )\subset L^{2}\left ( Q \right )\).

Dac\u{a} \(A\) este operatorul definit de \ref{eq:2.20}, unde \(a = -\pi, b = \pi\), cu acest nucleu (care este simetric, deci hermitian), atunci \(Ag=0\) pentru orice g care este o combina\c{t}ie liniar\u{a} de \(u_{0}, u_{1},\cdots,u_{m-1}\). Prin urmare
\begin{displaymath}
\ Span \left \{ u_{0}, u_{1},\cdots,u_{m-1} \right \} \subset N\left ( A \right ).
\end{displaymath}
Pe de alt\u{a} parte, dac\u{a} \(Af=0\), unde \(f\) este un element din \(H\), adic\u{a} \(f = \sum_{k=0}^{\infty }\left ( f,u_{k} \right )_{L^{2}}u_{k}\) (care este dezvoltarea \^{i}n serie Fourier a lui f ) atunci
\begin{equation} \nonumber
    \begin{split}
     0 &  =  \left ( Af,f \right )_{L^{2}}  \\ & = \left (\sum_{n=m}^{\infty }\frac{1}{n^{2}}\left ( f,u_{n} \right )_{L^{2}}u_{n}, \sum_{k=0}^{\infty }\left ( f,u_{k} \right )_{L^{2}}u_{k}  \right )_{L^{2}} \\ &  = \sum_{n=m}^{\infty }\frac{1}{n^{2}}\left ( f,u_{n} \right )_{L^{2}}^{2},
    \end{split}
\end{equation}
prin urmare \(\left ( f,u_{n} \right )_{L^{2}} = 0\) pentru orice \(n\geq m\) \c{s}i deci \(f = \sum_{k=0}^{m-1}\left ( f,u_{k} \right )_{L^{2}}u_{k}\), adic\u{a} \(f\in \ Span \left \{ u_{0}, u_{2},\cdots,u_{m-1} \right \}\), Prin urmare,
\begin{displaymath}
N\left ( A \right ) = \ Span \left \{ u_{0}, u_{2},\cdots,u_{m-1}. \right \}
\end{displaymath}
Pe de alt\u{a} parte, dac\u{a} alegem, de exemplu,
\begin{displaymath}
k\left ( t,s \right ) = 1+ \sum_{n=1}^{\infty }\frac{1}{n^{2}}u_{n}\left ( t \right )u_{n}\left ( s \right ), \left ( t,s \right )\in Q,
\end{displaymath}
atunci operatorul corespunz\u{a}tor A satisface condi\c{t}ia \(N\left ( A \right ) = \left \{ 0 \right \} \).
\end{example}

\textbf{Comentarii}
\begin{itemize}

\item Dac\u{a} \^{i}n ecuat\c{t}ia \begin{displaymath} x\left ( t \right ) = f\left ( t  \right ) + \lambda \int_{a}^{b}k\left ( t,s \right )x\left ( s \right )\ ds, t \in \left [ a,b \right ], \end{displaymath}

\noindent (care este \ref{eq:2.21} de mai sus) presupunem \(f\in C\left [ a,b \right ]\) \c{s}i \(k\in  C\left (\left [ a,b \right ] \times \left [ a,b \right ] \right )\), atunci \(x\in C\left [ a,b \right ]\). \^{I}n plus, dac\u{a} \(f\) si \(k\) sunt mai regulate, atunci \c{s}i \(x\) este.
        		      		      		
\item Teoria de mai sus func\c{t}ioneaza \c{s}i dac\u{a} \(\left [ a,b \right ]\) este \^{i}nlocuit cu un domeniu m\u{a}rginit \(D\subset \mathbb{R}^{N}\) sau cu frontiera unui astfel de domeniu. Este bine cunoscut faptul c\u{a} principalele probleme la limit\u{a} de tip eliptic (Dirichlet, Neumann, Robin) pot fi reduse, prin utilizarea poten\c{t}ialilor, la ecua\c{t}ii Fredholm \^{i}n domeniile corespuz\u{a}toare. Astfel, teoria de mai sus poate fi folosit\u{a} pentru a rezolva astfel de probleme.
        		      		      		
\end{itemize}



\chapter{Aplica\c{t}ii}
\nocite{sburlan}
\nocite{silov}

\begin{problem}
Calcula\c{t}i nucleul rezolvent al ecua\c{t}iei Volterra \c{s}i apoi g\u{a}si\c{t}i solu\c{t}ia corespunz\u{a}toare:
\[x\left ( t \right ) = e^{t^{2}} + \int_{0}^{t}e^{t^{2} - s^{2}}x\left ( s \right )\ ds , t\geq 0.\]
\end{problem}	
\begin{proof}
Reamintim c\u{a} pentru un nucleu dat
\begin{displaymath}
  k = k\left ( t,s \right ) \in C\left ( \Delta  \right ),  \Delta = \left \{ \left ( t,s \right ) \in \mathbb{R}; a\leq s\leq t\leq b \right \},
\end{displaymath}
nucleul resolvent \(R \left ( t,s \right )\) este definit de
\begin{displaymath}
R \left ( t,s \right ) = \sum_{n = 1}^{\infty } k_{n}\left ( t,s \right ), \left ( t,s \right ) \in \Delta,
\end{displaymath}
unde
\begin{equation} \nonumber
    \begin{split}
       k_{1}\left ( t,s \right ) & = k\left ( t,s \right ), \\
       k_{n}\left ( t,s \right ) & = \int_{s}^{t}k\left ( t,\tau  \right )k_{n-1}\left ( \tau ,s \right )d\tau , \left ( t,s \right ) \in \Delta , n = 2,3,\cdots
    \end{split}
\end{equation}

        		      		      		
Re\c{t}inem faptul c\u{a} intervalul \(\left [ a,b \right ]\) ar putea fi inlocuit cu \(\left [ a,\infty  \right )\) dac\u{a} ecua\c{t}ia Volerra corespunz\u{a}toare este considerat\u{a} \(\left [ a,\infty  \right )\).

 Evident, $k_1(t, s)=e^{t^{2} - s^{2}},~0<s<t,$ de unde
 \[
 k_2(t, s)=\int_s^t e^{t^{2} - \tau^{2}}e^{\tau^{2} - s^{2}} ~d\tau=(t-s)e^{t^{2} - s^{2}},    ~0<s<t,
 \]
  \[
 k_3(t, s)=\int_s^t e^{t^{2} - \tau^{2}}e^{\tau^{2} - s^{2}}(\tau-s) ~d\tau=\frac{(t-s)^2}{2!}e^{t^{2} - s^{2}},    ~0<s<t.
 \]
 Prin induc\c{t}ie ob\c{t}inem c\u{a}
 \[
 k_n(t, s)=\frac{(t-s)^{n-1}}{(n-1)!}e^{t^{2} - s^{2}},    ~0<s<t,~n=2, 3,\cdots.
 \]    			      			      			      	
Astfel g\u{a}sim
\begin{displaymath}
R \left ( t,s \right ) =\sum_{n=1}^\infty  k_n(t, s)=e^{t^{2}- s^{2}} \sum_{n=1}^\infty \frac{(t-s)^{n-1}}{(n-1)!} = e^{t^{2}- s^{2} + t - s },
\end{displaymath}
de unde
\[
x\left ( t \right )= e^{t^{2}}+\int_0^t R(t, s) e^{s^{2}}~ds= e^{t\left ( t+1 \right )}, t\geq 0.
\]
Alternativ, not\^{a}nd \(y\left ( t \right ) = e^{-t^{2}}x\left ( t \right ) \), ecua\c{t}ia dat\u{a} poate fi scris\u{a} astfel
\begin{displaymath}
y\left ( t \right ) = 1 + \int_{0}^{t} y\left ( s \right )ds, t\geq 0
\end{displaymath}
care prin derivare devine echivalent\u{a} cu problema Cauchy pentru o ecua\c{t}ie diferen\c{t}ial\u{a}
\begin{displaymath}
\left\{\begin{matrix}
{y}'\left ( t \right ) = y\left ( t \right ), t\geq 0, \\
y\left ( 0 \right ) = 1,
\end{matrix}\right.
\end{displaymath}
deci ob\c{t}inem din nou solu\c{t}ia \(x\).

\end{proof} 	      			      			      			      			      	
\begin{problem}
Rezolva\c{t}i urm\u{a}toarele ecua\c{t}ii integrale transform\^{a}ndu-le \^{i}n probleme Cauchy pentru ecua\c{t}ii diferen\c{t}iale:
\begin{enumerate}[label=(\alph*)]
\item x\(\left ( t \right ) =  t - \frac{t^{3}}{6} + \int_{0}^{t}\left ( t-s+1 \right )x\left ( s \right ) \ ds , t\geq 0;\)
\item \(x\left ( t \right ) =  t^{3} + 1 - \int_{0}^{t}\left ( t-s \right )x\left ( s \right ) \ ds , t\geq 0;\)
\item \(x\left ( t \right ) =  3t - \int_{0}^{t}e^{t-s}x\left ( s \right ) \ ds , t\geq 0.\)
\end{enumerate}
\end{problem}
\begin{proof}
\begin{enumerate}[label=(\alph*)]
\item  Dac\u{a} \(x\) este o solu\c{t}ie a ecua\c{t}iei integralei date, prin derivare avem c\u{a}  \(x\left ( 0 \right ) = 0\) \c{s}i
\begin{displaymath}
{x}'\left ( t \right ) = a - \frac{t^{2}}{2} + x\left ( t \right ) + \int_{0}^{t}x\left ( s \right )ds, t\geq 0.
\end{displaymath}
Prin urmare \({x}'\left ( 0 \right ) = 1\) \c{s}i deriv\^{a}nd \^{i}nc\u{a} o dat\u{a}
\begin{displaymath}
{x}'' = {x}' \left (t  \right ) + x\left ( t \right ) – t.
\end{displaymath}
Astfel am ob\c{t}inut problema Cauchy pentru o ecua\c{t}ie diferen\c{t}ial\u{a} de ordinul doi liniar\u{a}
\begin{displaymath}
\left\{\begin{matrix}
{x}'' - {x}'\left ( t \right ) - x\left ( t \right ) = -t, t\geq 0\\
x\left ( 0 \right ) = 0, {x}'\left ( 0 \right ) = 1.
\end{matrix}\right.
\end{displaymath}
Evident, dac\u{a} \(x\) este o solu\c{t}ie a aceastei probleme, atunci \(x\) satisfice ecua\c{t}ia integral\u{a} dat\u{a}.

\noindent Prin calcule g\u{a}sim
\begin{displaymath}
x\left ( t \right ) = c_{1}e^{\left (\frac{ 1 + \sqrt{5}}{2}t \right )}+ c_{2}e^{\left ( \frac{1-\sqrt{5}}{2}t \right )} + t – 1,
\end{displaymath}
cu
\begin{displaymath}
c_{1} = \frac{5 - \sqrt{5}}{2}, c_{2} = \frac{5 + \sqrt{5}}{2}.
\end{displaymath}

\item Proced\^{a}nd ca la punctul $(a)$, deriv\^{a}nd de dou\u{a} ori ecua\c{t}ia integral\u{a}, ob\c{t}inem c\u{a} Problema Cauchy echivalent\u{a} cu aceasta este
\begin{displaymath}
        	\left\{\begin{matrix}
        	{x}''\left ( t \right ) + x\left ( t \right ) = 6t,~ t\geq 0\\
        	x\left ( 0 \right ) = 1, {x}'\left ( 0 \right ) = 0.
        	\end{matrix}\right.
        \end{displaymath}

        Prin calcule simple g\u{a}sim c\u{a} solu\c{t}ia acesteia, care coincide cu solu\c{t}ia ecua\c{t}iei integrale, este
\begin{displaymath}
x\left ( t \right ) = \cos t - 6 \sin t  + 6t,~ t\geq 0.
\end{displaymath}
  \item Dac\u{a} $x$ este o solu\c{t}ie, atunci \(x\left ( 0 \right ) = 0\). Prin derivarea  ecua\c{t}iei integrale date, g\u{a}sim
\begin{displaymath}
{x}'\left ( t \right ) = 3 - x\left ( t \right ) - \underbrace{\int_{0}^{t}e^{t-s}x\left ( s \right )\ ds}_{3t -x\left ( t \right )}, t\geq 0,
\end{displaymath}
deci $x$ verific\u{a} problema Cauchy
        \begin{displaymath}
        	\left\{\begin{matrix}
        	{x}'\left ( t \right ) = 3 - 3t , t\geq 0\\
        	x\left ( 0 \right ) = 0
        	\end{matrix}\right.
        \end{displaymath}
care este echivalent\u{a} cu ecua\c{t}ia integral\u{a} dat\u{a} \c{s}i are solu\c{t}ia
\begin{displaymath}
x\left ( t \right ) = \frac{3}{2}t\left ( 2-t \right ),~ t\geq 0.
\end{displaymath}
\end{enumerate}

\end{proof}
\begin{problem}
Rezolva\c{t}i urm\u{a}toarele ecua\c{t}ii Volterra de spe\c{t}a \^{i}nt\^{a}i:
  	      			      			      			      	
\begin{enumerate}[label=(\alph*)]
  \item \(\int_{0}^{t}\left ( 1 - t^{2}  + s^{2}\right ) \cdot  x\left ( s \right ) \ ds  = \frac{t^{2}}{2}, t\geq 0;\)
  \item \(\int_{0}^{t}\cos \left ( t- s \right ) \cdot  x\left ( s \right ) \ ds  = 2t\left ( t+1 \right ), t\geq 0.\)
  \end{enumerate}
\end{problem}
  	      			      			      			      	
\begin{proof}
\begin{enumerate}[label=(\alph*)]
  \item Din ecua\c{t}ia integra\u{a} dat\u{a} ob\c{s}inem prin derivare
\begin{displaymath}
x\left ( t \right ) -2t \int_{0}^{t}x\left ( s \right )\ ds = t, t\geq 0
\end{displaymath}
Evident \(y\left ( t \right ) = \int_{0}^{t}x\left ( s \right )  \ ds\) satisfice problema Cauchy
        \begin{displaymath}
        	\left\{\begin{matrix}
        	{y}'\left ( t \right ) - 2ty\left ( t \right ) = t, t\geq 0\\
        	y\left ( 0 \right ) = 0
        	\end{matrix}\right.
        \end{displaymath}
care are solu\c{t}ia
\begin{displaymath}
y\left ( t \right ) = \frac{1}{2}\left ( e^{t^{2}}- 1 \right ),~ t\geq 0~\Rightarrow x\left ( t \right )=y'(t) = te^{t^{2}},~ t\geq 0.
\end{displaymath}

  \item Din ecua\c{t}ia integral\u{a} dat\u{a} ob\c{t}inem prin derivare
        \begin{displaymath}
        	x\left ( t \right ) - \int_{0}^{t}\sin \left ( t-s \right )\cdot x\left ( s \right ) \ ds = 2\left ( 2t + 1 \right ), t\geq 0
        \end{displaymath}

        \c{s}i deci \(x\left ( 0 \right ) = 2\). O alt\u{a} derivare conduce la
        \begin{displaymath}
        	{x}'\left ( t  \right ) - \underbrace{\int_{0}^{t}\cos \left ( t-s \right ) \cdot  x\left ( s \right ) \ ds}_{2t\left ( t+1 \right )} = 4, t\geq 0.
        \end{displaymath}

        Deci am ob\c{t}inut problema Cauchy
        \begin{displaymath}
        	\left\{\begin{matrix}
        	{x}'\left ( t \right ) = 2\left ( t^{2} + t + 2 \right ), t\geq 0\\
        	x\left ( 0 \right ) = 2,
        	\end{matrix}\right.
        \end{displaymath}

        care este echivalent\u{a} cu ecua\c{t}ia integral\u{a} dat\u{a} \c{s}i are solu\c{t}ia
        \begin{displaymath}
        	x\left ( t \right ) = \frac{2}{3}t^{3} + t^{2} + 4t + 2,~ t\geq 0.
        \end{displaymath}
\end{enumerate}
\end{proof}	

\begin{problem}  Fie \(h \in C\left [ 0,b \right ]\), unde \(b \in \left ( 0,\infty  \right )\). Definim \(k\left ( t,s \right ) = h\left ( t-s \right ), 0\leq s \leq  t\leq b\). S\u{a} se arate c\u{a} nucleul rezolvent \(R\left ( t,s \right )\) asociat nucleului \(k\left ( t,s \right )\) depinde doar de \(t-s\).
\end{problem}
\begin{proof}
\(R\left ( t,s \right )\) este o func\c{t}ie continu\u{a} pe mul\c{t}imea \(\Delta _{0} = \left \{ \left ( t,s \right );~0\leq s\leq t\leq b \right \}\), fiind definit\u{a} ca
\begin{displaymath}
  R\left ( t,s \right ) = \sum_{n = 1}^{\infty }k_{n}\left ( t,s \right ), \left ( t,s \right )\in \Delta _{0},
\end{displaymath}
unde
\begin{equation} \nonumber
    \begin{split}
        k_{1}\left ( t,s \right ) & = k\left ( t,s \right ) = h\left ( t-s \right ) \\
       k_{n}\left ( t,s \right ) & = \int_{s}^{t}k\left ( t,\tau  \right )k_{n-1}\left ( \tau ,s \right )\ d\tau \\ &
         = \int_{s}^{t} h\left ( t - \tau  \right )k_{n-1}\left ( \tau ,s \right ) \ d \tau , \left ( t,s \right ) \in \Delta _{0}, n = 2,3,\cdots
    \end{split}
\end{equation}

\noindent Deoarece \(k_{1}\) depinde numai de \(t-s\), putem observa cu u\c{s}urin\c{t}\u{a} (prin schimbarea de variabli\u{a}) c\u{a} la fel se int\^{a}mpl\u{a} \c{s}i pentru  \(k_{2}\). Prin induc\c{t}ie rezult\u{a} c\u{a} toate \(k_{n}\) – urile depind numai de \(t-s\)  \c{s}i rezult\u{a} c\u{a} la fel se \^{i}nt\^{a}mpl\u{a} \c{s}i cu \(R\).
\end{proof}

 	      			      			      			      	
\begin{problem}
Fie \(a,b \in \mathbb{R}, a< b\) \c{s}i func\c{t}iile ne-negative, \(f,x \in C\left [ a,b \right ], k\in C\left ( \Delta,  \right )\) unde \(\Delta  = \left \{ \left ( t,s \right ) \in \mathbb{R}; a\leq s\leq t\leq b\right \}\). Dac\u{a}
\begin{displaymath}
  x\left ( t \right ) \leq f\left ( t \right ) + \int_{a}^{k}k\left ( t,s \right )x\left ( s \right )\ ds , t\in \left [ a,b \right ]
\end{displaymath}
atunci
\begin{displaymath}
  x\left ( t \right ) \leq f\left ( t \right ) + \int_{a}^{k}R\left ( t,s \right )f\left ( s \right )\ ds , t\in \left [ a,b \right ],
\end{displaymath}
 unde \(R\left ( t,s \right )\) este nucleul rezolvent asociat cu \(k\left ( t,s. \right )\)
\end{problem}
\begin{proof}
  	      			      			      			      	
Putem ar\u{a}ta cu usurin\c{t}\u{a} prin induc\c{t}ie faptul c\u{a} \(R\left ( t,s \right ) \geq 0, 0\leq s\leq t\leq b.\)
Apoi, deducem c\u{a}
\begin{displaymath}
  \phi \left ( t \right ) = f\left ( t \right ) + \int_{a}^{t} k\left ( t,s \right ) x\left ( s \right ) \ ds - x\left ( t \right ) \geq 0, t \in \left [ a,b \right ].
\end{displaymath}
  	      			      			      			      	
\noindent Prin urmare,
\begin{displaymath}
  x\left ( t \right ) = f\left ( t \right ) - \phi \left ( t \right ) + \int_{a}^{t} k\left ( t,s \right )x\left ( s \right ) \ ds, t \in \left [ a,b \right ],
\end{displaymath}
  	      			      			      			      	
\noindent care implic\u{a},
\begin{equation} \nonumber
    \begin{split}
         x\left ( t \right ) &  = f\left ( t \right ) - \phi \left ( t \right ) + \int_{a}^{t} R\left ( t,s \right ) \left [ f\left ( s \right ) - \phi \left ( s \right ) \right ]\ ds \\ &
         = f\left ( t \right ) + \int_{a}^{t} R\left ( t,s \right )f\left ( s \right ) \ ds - \underbrace{ \left [ \phi \left ( t \right ) + \int_{a}^{t}R\left ( t,s \right )\phi \left ( s \right ) \ ds \right ], t \in \left [ a,b \right ]}_{\geq 0}.
    \end{split}
\end{equation}


De unde rezult\u{a} concluzia.
\end{proof}
  	      			      			      			      		
\begin{problem}
Fie a,b \(\in \left ( 0,\infty  \right )\). Definim
\[ D = \left \{ \left ( t,s \right );0\leq t\leq a, 0\leq s\leq b \right \},~
Q = \left \{ \left ( t,s,\xi , \eta  \right );0\leq \xi \leq t\leq a, 0\leq \eta \leq s\leq b \right \}. \]		Se consider\u{a} ecua\c{t}ia integral\u{a}
\begin{displaymath}
  x\left ( t,s \right ) = f\left ( t,s \right ) + \int_{0}^{t}\int_{0}^{s}k\left ( t,s,\xi ,\eta  \right )\ d\xi d\eta , \left ( t,s \right ) \in D. \label{eq:E} \tag{E}
\end{displaymath}
  	      			      			      			      	
\noindent Presupunem c\u{a} nucleul \(k \in C\left ( Q \right ):=C\left ( Q,\mathbb{R} \right )\).

\noindent S\u{a} se arate c\u{a} pentru orice \(f \in C\left ( D \right ):= C\left ( D,\mathbb{R} \right )\) exist\u{a} o unic\u{a} func\c{t}ie  \(x = x\left ( t,s \right ) \in C\left ( D \right )\) care satisfice ecua\c{t}ia \ref{eq:E} pentru orice \(\left ( t,s \right ) \in D\).
\end{problem}
  	      			      			      			      	
\begin{proof}
Folosim urm\u{a}toarea norm\u{a} asem\u{a}n\u{a}toare normei Bielecki \^{i}n $ X = C \left ( D \right ):$
\begin{displaymath}
 \left \| g \right \| _{B} = \underset{\left ( t,s \right )\in Q}{sup}e^{-M\left ( t+s \right )}\left | g\left ( t,s \right ) \right |, g\in X,
\end{displaymath}
unde \(M\) este o constant\u{a} pozitiv\u{a} suficient de mare. Definim  operatorul \(P\) pe \(X\) dat de
\begin{displaymath}
  \left ( Pg \right )\left ( t,s \right ) = f\left ( t,s \right ) + \int_{0}^{t}\int_{0}^{s}k\left ( t,s,\xi ,\eta  \right )g\left ( \xi ,\eta  \right ) \ d\eta  \ d\xi , \left ( t,s \right ) \in D, g \in X.
\end{displaymath}
\^{I}n mod clar, \(P\) duce \(X\) \^{i}n  \(X\) \c{s}i pentru \(g_{1}, g_{2} \in X\) \c{s}i \(\left ( t,s \right )  \in D\) avem
 \begin{equation} \nonumber
     \begin{split}
          \left | \left ( Pg_{1} \right )\left ( t,s \right ) - \left ( Pg_{2} \right )\left ( t,s \right ) \right | &  \leq  C\int_{0}^{t}\int_{0}^{s} \left | g_{1}\left ( \xi ,\eta  \right ) - g_{2}\left ( \xi ,\eta  \right ) \right | \ d\eta  \ d\xi \\ &
          = C\int_{0}^{t}\int_{0}^{s} e^{+M\left ( \eta +\xi  \right )}e^{-M\left ( \eta +\xi  \right )}\left | g_{1}\left ( \xi ,\eta  \right ) - g_{2} \left ( \xi ,\eta  \right )\right | \ d\eta  \ d\xi \\ &
           \leq C\left \| g_{1} - g_{2} \right \|_{B} \int_{0}^{t}\int_{0}^{s} e^{M\left ( \eta +\xi  \right )} \ d\eta  \ d\xi  \\ &
          = \frac{C}{M^{2}}\left \| g_{1} - g_{2} \right \|_{B}\left ( e^{Mt}  - 1\right )\left ( e^{Ms} - 1 \right ),
     \end{split}
 \end{equation}

\noindent unde \(C = \underset{ \left ( t,s,\xi ,\eta  \right ) \in Q}{ sup}\left | k\left ( t,s,\xi ,\eta  \right ) \right | < \infty\). Rezult\u{a} c\u{a}
\begin{equation} \nonumber
    \begin{split}
        e^{-M\left ( t+s \right )} \left | \left ( Pg_{1} \right ) \left ( t,s \right ) - \left ( Pg_{2} \right )\left ( t,s \right )\right | &   \leq \frac{C}{M^{2}}\left \| g_{1} - g_{2} \right \|_{B}\left ( 1 - e ^{-Mt} \right )\left ( 1 - e^{-Ms} \right ) \\ &
        \leq \frac{C}{M^{2}}\left \| g_{1} - g_{2}\right \|_{B},
    \end{split}
\end{equation}

\noindent pentru orice \(\left ( t,s \right )\in D, g_{1}, g_{2} \in X\). Prin urmare
\begin{displaymath}
  \left \| Pg_{1} - Pg_{2} \right \|_{B} \leq \frac{C}{M^{2}}\left \| g_{1} - g_{2} \right \|_{B}, g_{1}, g_{2}\in X,
\end{displaymath}
deci \(P\) este o contrac\c{t}ie pe \(\left ( X, \left \| \cdot  \right \|_{B} \right )\) pentru \(M^{2} > C\). Prin urmare, conform principiului contrac\c{t}iilor al lui Banach, \(P\) are un punct fix unic \(x = x\left ( t,s \right )\in X\) care este solu\c{t}ia unic\u{a} a ecua\c{t}iei \ref{eq:E}.
\end{proof}
  	      			      			      			      	
\begin{problem}
Se consider\u{a} problema
\begin{displaymath}
  \left\{\begin{matrix}
  {x}'\left ( t \right ) = f\left ( t \right ) + \int_{0}^{t}k\left ( t,s \right )x\left ( s \right ) \ ds , t\in \left ( 0,T \right ),
  & \\ x\left ( 0 \right ) = x_{0}
  \end{matrix}\right.
\end{displaymath}
unde \(x_{0} \in \mathbb{R} , T \in \left ( 0,\infty  \right ), f\in L^{1}\left ( 0,T \right ) , k\in C\left ( \Delta  \right )\) si \(\Delta = \left \{ \left ( t,s \right ) \in \mathbb{R}^{2}; 0\leq s\leq t\leq T \right \}\).

\noindent Ar\u{a}ta\c{t}i c\u{a} exist\u{a} o func\c{t}ie unic\u{a} \(x \in W^{1,1}\left ( 0,T \right )\) care satisfice ecua\c{t}ia integro-diferen\c{t}ial\u{a} de mai sus pentru orice \(t \in \left ( 0,T \right )\) \c{s}i condi\c{t}ia ini\c{t}ial\u{a} \(x\left ( 0 \right ) = x_{0}\).
\end{problem}
  	      			      			      	
\begin{proof}
Problema dat\u{a} este echivalent\u{a} cu urm\u{a}toarea ecua\c{t}ie integral\u{a} \^{i}n \(X = C\left [ 0, T \right ]\)
\begin{displaymath}
  x\left ( t \right ) = x_{0} + \int_{0}^{t}f\left ( s \right )\ ds + \int_{0}^{t}\left ( \int_{0}^{s} k\left ( s,\tau  \right )x\left ( \tau  \right ) \ d\tau \right )\ ds, t \in \left [ 0, T \right ].\label{eq:*} \tag{*}
\end{displaymath}
  	      			      			      	
\noindent Definim \(P : X \rightarrow X\) prin
\begin{displaymath}
  \left ( Pg \right )\left ( t \right ) = x_{0} + \int_{0}^{t}f\left ( s \right ) \ ds + \int_{0}^{t}\left ( \int_{0}^{s}k\left ( s,\tau  \right )g\left ( \tau  \right ) \ d\tau  \right ) \ ds, t \in \left [ 0, T \right ], g \in X.
\end{displaymath}
  	      			      			      	
\noindent Se poate ar\u{a}ta printr-o abordare cu punct fix faptul c\u{a} \(P\) are un unic punct  fix  \(x \in X\) care este solu\c{t}ia unic\u{a} a ecua\c{t}iei \ref{eq:*} \c{s}i deci a problemei date.
\end{proof}
  	      			      			      		
\begin{problem}
Rezolva\c{t}i urm\u{a}toarele ecua\c{t}ii integrale, unde \(\lambda\) este un parametru real:
\begin{enumerate}[label=(\alph*)]
  \item \(x\left ( t \right ) = \cos t + \lambda \int_{0}^{\pi }\sin \left ( t-s \right )\cdot x\left ( s \right ) \ ds;\)
  \item \(x\left ( t \right ) =  t + \lambda \int_{0}^{2\pi }\left | \pi - s \right |\sin t \cdot x\left ( s \right ) \ ds;\)
  \item \(x\left ( t \right ) = f \left (t  \right ) + \lambda \int_{0}^{1 }\left ( 1 - 3ts  \right )\cdot x\left ( s \right ) \ ds , f \in L^{2}\left ( 0,1 \right ).\)
\end{enumerate}
\end{problem}
  	      			      			      		
\begin{proof}
\begin{enumerate}[label=(\alph*)]
  \item Ecua\c{t}ia poate fi scris\u{a} ca
        \begin{displaymath}
        	x\left ( t \right ) = \cos t + \lambda c_{1}\sin t + \lambda c_{2}\cos t, \label{eq:*} \tag{*}
        \end{displaymath}
cu
        \begin{displaymath}
        	c_{1} = \int_{0}^{\pi }\cos s \cdot x\left ( s \right ) \ ds,~~ c_{2} = -\int_{0}^{\pi }\sin s \cdot x\left ( s \right ) \ ds. \label{eq:**} \tag{**}
        \end{displaymath}
 Dac\u{a} \^{i}nlocuim \ref{eq:*} \^{i}n \ref{eq:**}, ob\c{t}inem urm\u{a}torul sistem algebraic \^{i}n \(c_{1}, c_{2}\):
        \begin{displaymath}
        	\left\{\begin{matrix}
        	c_{1} - \frac{\lambda \pi }{2}c_{2} = \frac{\pi }{2}\\
        	\frac{\lambda \pi }{2}c_{1} + c_{2} = 0.
        	\end{matrix}\right.
        \end{displaymath}
Observ\u{a}m faptul c\u{a}, determinantul acestui sistem este $1+ (\lambda \pi)^2/4,$ deci pozitiv pentru orice \(\lambda \in \mathbb{R}\), ca urmare exist\u{a} o unic\u{a}  solu\c{t}ie \(\left ( c_{1} , c_{2}\right )\) care ne d\u{a} solu\c{t}ia ecua\c{t}iei integrale date
        \begin{displaymath}
        	x\left ( t \right ) = \frac{2\left ( 2\cos t+ \lambda\pi\sin t \right )}{4 + \lambda^{2}\pi^{2}}.
        \end{displaymath}
        		      		      		      		      		      		
  \item Avem \(x\left ( t \right ) = t + \lambda c \sin t,\) unde \(c = \int_{0}^{2\pi}\left | \pi - s \right |\cdot x\left ( s \right ) \ ds\). \^{I}nlocuind \(x\left ( t \right )\) dat de prima rela\c{t}ie \^{i}n cea de a doua, rezult\u{a}
\begin{displaymath}
  c = \int_{0}^{2\pi} \left | \pi - s \right | \cdot \left ( s+ \lambda c\sin s \right ) \ ds \Leftrightarrow c\left ( 1- 2\lambda \pi \right ) = \pi^{3}.
\end{displaymath}
Prin urmare dac\u{a} \(\lambda = \frac{1}{2\pi}\) ecua\c{t}ia integral\u{a} dat\u{a} nu are solu\c{t}ie. Altfel, ecua\c{t}ia are solu\c{t}ia
\begin{displaymath}
  x\left ( t \right ) = t + \frac{\lambda \pi ^{3}}{1 - 2 \lambda \pi}\sin t.
\end{displaymath}
        		      		      		      		      		      		
  \item Avem
        \begin{displaymath}
        	x\left ( t \right ) = f \left ( t \right ) + \lambda c_{1} - 3 \lambda c_{2}t, \label{eq:*} \tag{*}
        \end{displaymath}
 unde
        \begin{displaymath}
        	c_{1} = \int_{0}^{t} x\left ( s \right ) \ ds,~c_{2} = \int_{0}^{t} sx\left ( s \right ) \ ds.
        \end{displaymath}
Astfel ob\c{t}inem sistemul 
        \begin{displaymath}
        	\left\{\begin{matrix}
        	c_{1} = \int_{0}^{1} \left [ f\left ( s \right ) + \lambda c_{1} - 3\lambda c_{2}s \right ]\ ds,\\
        	c_{2} = \int_{0}^{1}s \left [ f\left ( s \right ) + \lambda c_{1} - 3\lambda c_{2}s \right ]\ ds,
        	\end{matrix}\right.
        \end{displaymath}
 sau
 \begin{displaymath}
        	\left\{\begin{matrix}
        	\left ( 1 - \lambda  \right )c_{1} + \frac{3}{2}\lambda c_{2} = \int_{0}^{1}f\left ( s \right ) \ ds,\\
        	-\frac{1}{2}c_{1} + \left ( 1 + \lambda  \right )c_{2} = \int_{0}^{1}s f\left ( s \right ) ds.
        	\end{matrix}\right.
        \end{displaymath}
Determinantul acestui sistem algebric este \(\Delta  =  \frac{4 - \lambda ^{2} }{4}\).

Deci, pentru fiecare \(\lambda \in \mathbb{R} \setminus \left \{ -2,+2 \right \}, c_{1}, c_{2}\) pot fi determinate \^{i}n mod unic \c{s}i solu\c{t}ia ecua\c{t}iei integrale poate fi exprimat\u{a} explicit folosind formula \ref{eq:*}.

\noindent Dac\u{a} \(\lambda = -2\) sistemul algebric de mai sus are solu\c{t}ii dac\u{a} \c{s}i numai dac\u{a}
        \begin{displaymath}
        	\int_{0}^{1}f\left ( s \right ) \ ds  = 3\int_{0}^{1}sf\left ( s  \right )\ ds \label{eq:**} \tag{**}
        \end{displaymath}
\c{s}i \^{i}n acest caz, exist\u{a} o infinitate de solu\c{t}ii ale ecua\c{t}iei integrale date, \c{s}i anume ,
        \begin{displaymath}
        	x\left ( t \right ) = f\left ( t \right ) + 2c_{1}\left ( 3t-1 \right ) - 2t\int_{0}^{1}f\left ( s \right ) \ ds, c_{1 }\in \mathbb{R}.
        \end{displaymath}
Un exemplu de func\c{t}ie care satisfice condi\c{t}ia \ref{eq:**} de mai sus este \(f\left ( t  \right ) = t-1.\)

\noindent Dac\u{a} condi\c{t}ia \ref{eq:**} nu este \^{i}ndeplinit\u{a}, atunci ecua\c{t}ia integral\u{a} dat\u{a} nu are solu\c{t}ie.


\noindent  Dac\u{a} \(\lambda = +2\) condi\c{t}ia de compatibilitate pentru sistemul algebric de mai sus este
\begin{displaymath}
  \int_{0}^{1}f\left ( s \right ) \ ds = \int_{0}^{1}sf\left ( s \right ) \ ds.
\end{displaymath}
 Dac\u{a} aceast\u{a} condi\c{t}ie este \^{i}ndeplinit\u{a} (de exemplu, pentru \(f\left ( 3t  \right ) = t-1)\), avem din nou o infinitate de solu\c{t}ii pentru ecua\c{t}ia integral\u{a} dat\u{a}, de forma
        \begin{displaymath}
        	x\left ( t \right ) = f\left ( t \right ) + 2c_{1}\left ( 1-t \right ) -2t\int_{0}^{1}f\left ( s \right ) \ ds, c_{1} \in \mathbb{R}.
        \end{displaymath}
        \end{enumerate}

\noindent \^{I}n caz contrar, ecua\c{t}ia integral\u{a} dat\u{a} nu are solu\c{t}ie.
\end{proof}
        		      		      		      		      		      		
  	      			      			      	
\begin{problem}
Se consider\u{a} \^{i}n \(\mathbb{K}\)  urm\u{a}toarea ecua\c{t}ie Fredholm cu nucleu degenerat (separabil):
\begin{displaymath}
  x\left (  t\right ) = f\left ( t \right ) + \lambda \int_{a}^{b}\underbrace{ \left [ \sum_{i = 1}^{n}a_{i}\left ( t \right )b_{i}\left ( s \right ) \right ]}_{k\left ( t,s \right )}x\left ( s \right ) \ ds , \label{eq:F} \tag{F}
\end{displaymath}
unde \(\lambda \in \mathbb{K} , f, a_{i}, b_{i} \in L^{2} \left ( a,b; \mathbb{K} \right ), i = 1,2,\cdots,n.\) Presupunem, f\u{a}r\u{a} a restr\^{a}nge generalitatea c\u{a} sistemele \(\left \{ a_{1},\cdots ,a_{n} \right \}, \left \{ b_{1},\cdots ,b_{n} \right \}\) sunt linear independente. Ob\c{t}inem
\begin{displaymath}
  c_{i} = \int_{a}^{b} b_{i}\left ( s \right )x\left ( s \right )\ ds, i = 1,\cdots ,n, \label{eq:1} \tag{1}
\end{displaymath}
\c{s}i din  \ref{eq:F} deducem c\u{a}
\begin{displaymath}
  x\left ( t \right ) = f\left ( t \right ) + \lambda \sum_{i=1}^{n}c_{i}a_{i}\left ( t \right ). \label{eq:2} \tag{2}
\end{displaymath}
Introduc\^{a}nd \ref{eq:2} \^{i}n \ref{eq:1} ob\c{t}inem sistemul algebric
\begin{displaymath}
  c_{i} = f_{i} + \lambda \sum_{k = 1}^{n}k_{ij}c_{j},~ i = 1,\cdots ,n , \label{eq:3} \tag{3}
\end{displaymath}
unde
\begin{displaymath}
  f_{i} = \int_{a}^{b} b_{i}\left ( s \right )f\left ( s \right )ds,~ k_{ij} = \int_{a}^{b}b_{i}\left ( s \right )a_{j}\left ( s \right )ds, i,i = 1,\cdots,n.
\end{displaymath}
Ar\u{a}ta\c{t}i c\u{a} alternativa lui Fredholm pentru ecua\c{t}ia \ref{eq:F} poate fi exprimat\u{a} ca o alternativ\u{a} echivalent\u{a} pentru sistemul algebric \ref{eq:3}.
\end{problem}
  	      			      	
\begin{proof}
Not\u{a}m
\begin{displaymath}
  c = \begin{bmatrix}
  c_{1}\\
  \vdots \\
  c_{n}
  \end{bmatrix},
  g = \begin{bmatrix}
  f_{1}\\
  \vdots \\
  f_{n}
  \end{bmatrix},
  K = \left ( k_{ij} \right )_{1\leq i,j\leq n}.
\end{displaymath}
Deci \ref{eq:3} poate fi scris\u{a} ca:
\begin{displaymath}
  \left ( I - \lambda K \right )c = g. \label{eq:3 '} \tag{3 '}
\end{displaymath}
Exist\u{a} o coresponden\c{t}\u{a} bijectiv\u{a} \^{i}ntre mul\c{t}imea solu\c{t}iilor ecua\c{t}iei \ref{eq:F} \c{s}i mul\c{t}imea solu\c{t}iilor sistemului \ref{eq:3 '}.

\noindent Urm\u{a}toarea alternativ\u{a} pentru ecua\c{t}ia \ref{eq:3 '} este bine cunoscut\u{a}:
\begin{enumerate}
    \item  dac\u{a} \(\det \left ( I - \lambda K  \right ) \neq 0\), atunci exist\u{a} o solu\c{t}ie unic\u{a} a sistemului \ref{eq:3 '} dat\u{a} de ,
\begin{displaymath}
  c = \left ( I - \lambda K \right )^{-1}g
\end{displaymath}
care d\u{a} solu\c{t}ia lui \ref{eq:F} prin intermedul rela\c{t}iei \ref{eq:2};
\item  \^{i}n caz contrar, \(\ det \left ( I - \lambda K  \right ) = 0\) \c{s}i ecua\c{t}ia \ref{eq:3 '} are solu\c{t}ii dac\u{a} \c{s}i numai dac\u{a} \(g\) este perpendicular\u{a} pe \(N\left ( I - \overline{\lambda} K^{\ast } \right ) = N\left ( I - \overline{\lambda} \overline{K}^{T} \right )\), astfel ecua\c{t}ia \ref{eq:F} are o infinitate de solu\c{t}ii,
\begin{displaymath}
  x \left ( t \right ) = x_{p}\left ( t \right ) + \sum_{i = 1}^{m}\alpha _{i}x_{i}\left ( t \right ),
\end{displaymath}
unde \( x_{p}\) este o solu\c{t}ie particular\u{a} a lui \ref{eq:F}, \(\alpha _{1},\cdots,\alpha _{m} \in \mathbb{K}\) \c{s}i \(x _{1},\cdots,x _{m}\) sunt solu\c{t}ii liniar independente ale ecua\c{t}iei integrale omogene (care pot fi calculate explicit).
\end{enumerate}

\end{proof}
  	      			      			      			      	
  	      			      			      			      	
\begin{problem}
Fie \(\left ( H, \left ( \cdot ,\cdot  \right ), \left \| \cdot  \right \| \right )\) un spa\c{t}iu Hilbert \c{s}i  \(\left \{ e_{1},\cdots, e_{m} \right \} \subset H\) un sistem ortonormat, unde \(m\) este un num\u{a}r natural dat.

\noindent Definim \(A : H \rightarrow H\) prin
\begin{displaymath}
  Ax = \sum_{k = 1 }^{m} k\left ( x,e_{k} \right )e_{k}, x \in H.
\end{displaymath}
Rezolva\c{t}i ecua\c{t}ia abstract\u{a} de tip Fredholm
\begin{displaymath}
  x = f + \lambda Ax,
\end{displaymath}
unde \(f\in H\) \c{s}i \(\lambda \in \mathbb{K}\).
\end{problem}
  	      			      	
\begin{proof}
Dac\u{a} \(\lambda = 0\) atunci exist\u{a} o unic\u{a} solu\c{t}ie a ecua\c{t}iei,  \(x = f\).

Presupunem \(\lambda \in \mathbb{K} - \left \{ 0 \right \}\). Not\u{a}m \(H _{m} = \ Span \left ( \left \{ e_{1},\cdots,e_{m} \right \} \right )\). Operatorul A este simetric (deci are toate valorile proprii numere reale), \(R\left ( A \right ) = H_{m}\) \c{s}i  \(N\left ( A \right ) = H_{m}^{\perp }\). Observ\u{a}m c\u{a} valorle proprii ale lui \(A\) sunt \(\mu _{k} = k, k = 1,....,m ,\) cu \(e_{1},...,e_{m}\) ca vectori proprii corespunz\u{a}tori.

\noindent Distingem dou\u{a} cazuri:
\begin{enumerate}
    \item  Cazul 1.
\(\dim H = m,\) deci \(H = H_{m}\). Ecua\c{t}ia Fredholm dat\u{a} se reduce la  un sistem algebric
\begin{displaymath}
  \left ( I - \lambda A \right )x = f. \label{eq:1} \tag{1}
\end{displaymath}
Dac\u{a} \(\lambda \in \mathbb{K} - \left \{ 1,\frac{1}{2},\cdots,\frac{1}{m} \right \}\), atunci \ref{eq:1} are solu\c{t}ie unic\u{a}
\begin{displaymath}
  x - \sum_{k = 1}^{m}\frac{\left ( f,e_{k} \right )}{1 - \lambda k}e_{k}.
\end{displaymath}
  	      			      	
Dac\u{a} \(\lambda  = \frac{1}{j}\) pentru un \(j \in \left \{ 1,\cdots,m \right \},\) atunci sistemul \ref{eq:1} are solu\c{t}ie dac\u{a} \c{s}i numai dac\u{a} \(\left ( f,e_{j} \right ) = 0\). \^{I}n aceast\u{a} situa\c{t}ie, exist\u{a} o infinitate de solu\c{t}ii \(x\) cu coordonatele \(x_{k} = \frac{j\left ( f,e_{k} \right )}{j-k }, k\neq j\) \c{s}i \(x_{j} \in \mathbb{K}\) este arbitrar.
\item Cazul 2.
\(\dim H > m\). Desigur, \(H = H_{m}\oplus H_{m}^{\perp }\), cu  \(H_{m}^{\perp }\neq \left \{ 0 \right \}\). C\u{a}ut\u{a}m \(x\) de forma \(x = x_{1} + x_{2}, x_{1} \in H_{m}, x_{2} \in H_{m}^{\perp }\). Folosind o descompunere similar\u{a} pentru \(f\), adic\u{a} \(f = f_{1} + f_{2}, f_{1} \in H_{m}, f_{2} \in H_{m}^{\perp }\), deducem din \ref{eq:1} c\u{a} \(x_{2} = f_{2}\) \c{s}i \(\left ( I - \lambda A \right )x _{1} = f_{1}\). Folosind aceea\c{s}i discu\c{t}ie ca mai \^{i}nainte,  g\u{a}sim \(x_{1}\), atunci c\^{a}nd exist\u{a}, deci concluzion\u{a}m c\u{a} \(x = x_{1} + f_{2}\).
\end{enumerate}
\end{proof}
  	      			      	
\begin{problem}
Se consider\u{a} func\c{t}iile
\begin{displaymath}
  u_{n}\left ( t \right ) = \sqrt{2} \cos \left ( \left ( n+ \frac{1}{2} \right )\pi t \right ), t \in\left [ 0,1 \right ], n = 0,1,2,\cdots
\end{displaymath}
Este bine cunoscut faptul c\u{a} sistemul \(\left \{ u_{n} \right \}_{n=0}^{\infty }\) este o baz\u{a} ortonormat\u{a} \^{i}n \(H  = L^{2} \left ( 0,1 \right )\) \^{i}nzestrat cu produsul scalar uzual \c{s}i norma indus\u{a} de acesta. Definim nucleul \(k\left ( t,s \right )\) prin
\begin{displaymath}
  k\left ( t,s \right ) = \sum_{n=m}^{\infty } \frac{1}{\left ( n+1 \right )^{2}}u_{n}\left ( t \right )u_{n}\left ( s \right ), t,s \in \left [ 0,1 \right ],
\end{displaymath}
unde \(m \in \left \{ 0,1,2,\cdots\right \}\) \c{s}i  operatorul integral \(A : H \rightarrow H\),
\begin{displaymath}
  \left ( Ag \right )\left ( t \right ) = \int_{0}^{1} k\left ( t,s \right )g\left ( s \right )ds, g \in H.
\end{displaymath}
Discuta\c{t}i existen\c{t}a solu\c{t}iilor ecua\c{t}iei Fredholm
\begin{displaymath}
  x = f + \lambda Ax, f \in H, \lambda \in \mathbb{R},
\end{displaymath}
\^{i}n dou\u{a} cazuri: \(m = 0\) \c{s}i \(m \geq 1\).
\end{problem}
  	
\begin{proof}
Din M-testul lui Weierstrass, avem
\begin{displaymath}
  k \in C\left ( \overline{Q} \right ) \subset L^{2}\left ( Q \right ), Q = \left ( 0,1 \right )\times \left ( 0,1 \right ).
\end{displaymath}
Evident, A este auto-adjunct \c{s}i compact.

\noindent \textbf{Cazul \(m = 0\)}

\noindent \^{I}n acest caz, \(N\left ( A \right ) = \left \{ 0 \right \}\). \^{I}ntr-adev\u{a}r, \(Ag = 0\) implic\u{a}
\begin{displaymath}
  0 = \left ( Ag,g \right )_{L^{2}} = \sum_{n= 1}^{\infty }\frac{1}{\left ( n+1^{2} \right )}\left ( g , u_{n} \right )_{L^{2}}^{2},
\end{displaymath}
prin urmare,
\begin{displaymath}
  \left ( g,u_{n} \right )_{L^{2}} = 0, \forall n \in \left \{ 0,1,2,\cdots \right \}\Rightarrow g = 0,
\end{displaymath}
deoarece sistemul \(\left \{ u_{n} \right \}_{n=0}^{\infty }\) este o baz\u{a} \^{i}n \(H = L^{2}\left ( 0,1 \right )\).

\noindent Pentru a determina perechile proprii ale lui \(A\) lu\u{a}m in considerare ecua\c{t}ia
\begin{displaymath}
  Ag = \mu g
\end{displaymath}
 care poate fi scris\u{a} ca
\begin{displaymath}
  \sum_{n = 0}^{\infty }\frac{\left ( g, u_{n} \right )_{L^{2}}}{\left ( n+1 \right )^{2}}u_{n} = \mu \sum_{n = 0}^{\infty }\left ( g, u_{n} \right )_{L^{2}}u_{n},
\end{displaymath}
unde am folosit dezvoltarea \^{i}n serie Fourier a lui \(g\). 

Deoarece \(\left \{ u_{n} \right \}_{n = 0}^{\infty }\) este o baz\u{a} \^{i}n \(H\) , avem
\begin{displaymath}
  \left ( \mu  - \frac{1}{\left ( n+1 \right )^{2}} \right )\left ( g,u_{n} \right )_{L^{2}} = 0, n = 0,1,2,\cdots (*)
\end{displaymath}
Dac\u{a} \(\mu \neq \frac{1}{\left ( n+1 \right )^{2}}\) pentru orice \(n\in \left \{ 0,1,2,\cdots, \right \}\) avem
\begin{displaymath}
  \left ( g,u_{n} \right )_{L^{2}} = 0, \forall n\in \left \{ 0,1,2,\cdots \right \}\Rightarrow g = 0,
\end{displaymath}
prin urmare, astfel de  \(\mu\)  -uri nu sunt valori proprii ale lui \(A\). 

Pentru \(\mu   = \mu _{n} = \frac{1}{\left ( n+1 \right )^{2}}\) avem din \ref{eq:*}
\begin{displaymath}
  \left ( g,u_{k} \right )_{L^{2}} = 0, \forall k\in \mathbb{N}, k\neq n,
\end{displaymath}
deci func\c{t}iile proprii corespunzatoare lui  \(\mu   = \mu _{n} = \frac{1}{\left ( n+1 \right )^{2}}\) sunt multipli diferi\c{t}i de zero ai lui \(u_{n}\).

\noindent Conform formulei lui Schmidt pe care o avem pentru \(\lambda \in \mathbb{R}- \left \{ 1,2^{2},3^{2}\cdots \right \}\) \c{s}i pentru \(t \in \left ( 0,1 \right )\),
\begin{displaymath}
  x\left ( t \right ) = f\left ( t \right ) + 2\lambda\sum_{k=0}^{\infty }\frac{\int_{0}^{1}f\left ( s \right )\ cos\left ( \left ( k+\frac{1}{2} \right )\pi s \right ) \ ds}{\left ( k+1 \right )^{2} -\lambda } \times \cos \left ( \left ( k+\frac{1}{2} \right )\pi t \right ) +
\end{displaymath}
  	      	
\begin{displaymath}
  + \alpha  \cos \left ( \left ( n + \frac{1}{2} \right )\pi t \right ), \alpha \in \mathbb{R}.
\end{displaymath}
\textbf{Cazul \(m\geq 1\)}

\noindent \^{I}n acest caz \(Y_{0} := N\left ( A \right ) = \ Span\left ( \left \{ u_{0},u_{1},\cdots,u_{m-1} \right \} \right )\) \c{s}i \(H = Y_{0} \oplus Y_{1}\), unde  \(Y_{1} = N\left ( A \right )^{\perp } = \ Span \left ( \left \{ u_{m}, u_{m+1},\cdots \right \} \right )\).
Not\u{a}m cu \(A_{1}\) restric\c{t}ia lui \(A\) la \(Y_{1}\) care este un spa\c{t}iu Hilbert \^{i}n raport cu produsul scalar \c{s}i norma lui \(H  = L^{2}\left ( 0,1 \right )\). Evident, \(A_{1}\) \^{i}l duce \(Y_{1}\) \^{i}n \(Y_{1}\), e compact, autoadjunct, cu \(N\left ( A_{1} \right ) = \left \{ 0 \right \}\) \c{s}i cu valori proprii \(\mu _{n} = \frac{1}{\left ( n+1 \right )^{2}}\) \c{s}i func\c{t}ii proprii \(u_{n}, n\geq m+1\). De fapt, \(Y_{1}\) \c{s}i \(A_{1}\) joac\u{a} rolurile lui H \c{s}i A pe care le-am v\u{a}zut inainte.

\noindent Ecua\c{t}ia
\begin{displaymath}
  x = f + \lambda Ax
\end{displaymath}

\noindent poate fi scris\u{a} ca
\begin{displaymath}
  x_{0} + x_{1} = f_{0} + f_{1} + \lambda Ax_{1},
\end{displaymath}
unde \(x_{0} ,f_{0} \in Y_{0}\) \c{s}i \(x_{1} ,f_{1} \in Y_{1}\), deci \(x_{0} = f_{0}\)  \c{s}i
\begin{displaymath}
  x_{1} = f_{1} + \lambda A_{1}x_{1}, \label{eq:**} \tag{**}
\end{displaymath}
Pe baza argumentelor de mai sus, avem
\begin{enumerate}
    \item Dac\u{a} \(\lambda \neq \left ( n+1 \right )^{2}\) pentru orice \(n\geq m\) atunci
\begin{equation} \nonumber
    \begin{split}
       x\left ( t \right ) &   = f_{0}\left ( t \right ) + x_{1}\left ( t \right )  \\ &
        = f\left ( t \right ) + 2 \lambda \sum_{k=m}^{\infty }\frac{\int_{0}^{1}f_{1\left ( s \right ) \ cos\left ( \frac{k+1}{2} \right )\pi s }\ ds}{\left ( k+1 \right )^{2} - \lambda }\times \ cos \left ( \frac{k+1}{2}\pi t \right ),
    \end{split}
\end{equation}

 \c{s}i
 \item Dac\u{a} \(\lambda = \left ( n+1 \right )^{2}\), pentru un \(n\geq m\), atunci ecua\c{t}ia Fredholm \ref{eq:**} are solu\c{t}ii dac\u{a} \c{s}i numai dac\u{a} \(f_{1} \perp u_{n} \Leftrightarrow f \perp u_{n}\) iar \^{i}n acest caz
 \begin{equation} \nonumber
     \begin{split}
         x\left ( t \right ) &    = f\left ( t \right ) + \left ( 2n+1 \right )^{2} \\ &
         \times \sum_{k\geq m,k\neq n}\frac{\int_{0}^{1}f_{1}\left ( s \right ) \ cos \left ( \frac{k+1}{2} \right ) \pi s \ ds}{\left ( k+1 \right )^{2} - \lambda } \\ &
          \times  \ cos \left ( \frac{k+1}{2}\ \pi t  \right ) + \alpha  \cos \left ( \frac{n+1}{2} \pi t \right ), \alpha \in \mathbb{R}.
     \end{split}
 \end{equation}


\end{enumerate}

\end{proof}
\chapter* {Concluzii}
\^{I}n cadrul acestei lucr\u{a}ri au fost prezentate ecua\c{t}iile integrale Volterra si fredholm. 

Primul capitol a cuprins defini\c{t}ii \c{s}i teoreme pe care, le-am folosit \^{i}n cadrul lucr\u{a}rii. Printre acestea reg\u{a}sim, Teorema de existen\c{t}\u{a} a lui Peano, Principiul contrac\c{t}iilor Banach , care este o abstractizare a metodei aproxima\c{t}iilor succesive, metod\u{a} utilizat\u{a} \^{i}n mod empiric \^{i}nc\u{a} din antichitate pentru rezolvarea ecua\c{t}iilor numerice. Teorema lui Fredholm, care ne spune c\u{a}, dac\u{a} avem un spa\c{t}iu Hilbert \(\left ( H, \left ( \cdot ,\cdot  \right ), \left \| \cdot  \right \| \right )\)  \c{s}i  \(A \in K\left ( H \right ),\) ecua\c{t}ia \(x - A^{\ast }x = f\) are solu\c{t}ie dac\u{a} \c{s}i numai dac\u{a} \(f \in { \mathcal{N}} ^{\perp }\) unde \({ \mathcal{N}} = N\left ( I - A \right ). \)  De asemenea am abordat teorema Hilbert-Schmidt \c{s}i Criteriul Arzelà-Ascoli.

Cel de al doilea capitol, cuprinde o prezentare a ecua\c{t}iilor integrale  Volterra , de primul \c{s}i de al doilea tip, \c{s}i anume:
\begin{displaymath}
f\left ( t \right ) = \int_{a}^{t}k\left ( t,s \right )x\left ( s \right )ds,    a\leq t\leq b  
\end{displaymath}
\c{s}i
\begin{displaymath}
x\left ( t \right ) = f\left ( t \right ) + \int_{a}^{t}k\left ( t,s \right )x\left ( s \right )ds, a\leq t\leq b, 
\end{displaymath}
unde \[a,b \in \mathbb{R}, a< b, f\in C\left [ a,b \right ]:= C\left ( \left [ a,b \right ];\mathbb{R} \right ), k\in C\left (\Delta   \right ):= C\left ( \Delta ;\mathbb{R} \right )\] (numit nucleu), cu \(\Delta =\left \{ \left ( t,s \right )\in \mathbb{R}^{2};a\leq s\leq t\leq b \right \};\) \(x=x\left ( t \right )\) reprezint\u{a} func\c{t}ia necunoscut\u{a} care apar\c{t}ine spa\c{t}iul \(C\left [ a,b \right ]\)
si ecua\c{t}iile integrale  Fredholm
\begin{displaymath}
x\left ( t \right ) = f\left ( t \right ) + \int_{a}^{b}k\left ( t,s \right )x\left ( s \right )ds, t\in \left [ a,b \right ], 
\end{displaymath}
unde \(a,b \in \mathbb{R}, a< b, f\in C\left ( \left [ a,b \right ]; \mathbb{K}\right )\) \c{s}i \(k\in C\left ( \left [ a,b \right ] \times \left [ a,b \right ]; \mathbb{K}\right ).\)

Cel de al patrulea capitol, a cuprins ecua\c{t}ii Volterra \c{s}i Fredholm, dar \c{s}i probleme Cauchy. 


\bibliographystyle{unsrt}
\setlength{\baselineskip}{\normalbaselineskip}
\setlength{\parskip}{0pt}
\bibliography{refs}
\end{document}

\textbf{Comentarii}

\begin{comentary}
Dac\u{a} \^{i}n teorema 16 presupunem \(d = 0\) ( adica, \(f \equiv x_{0} \)) \c{s}i \(k\) este independent de \(t\), adica \(k\left ( t,s,v \right ) = h \left ( s,v \right )\), atunci ob\c{t}inem din nou o existen\c{t}a binecunoscut\u{a} \c{s}i rezultatul de unicitate pentru problema Cauchy
\begin{displaymath}
{x}'\left ( t \right ) = h\left ( t,x\left ( t \right ) \right ), x\left ( a \right ) = x_{0}.
\end{displaymath}
Vezi preliminarii (subcapitolul 1.1 ) . Acela\c{s}i rezultat poate sa fie derivat din teorema 17.
\end{comentary}
\begin{comentary}
Dac\u{a} toate condi\c{t}iile teoremei 16 sunt \^{i}ndeplinite, cu excep\c{t}ia condi\c{t}iei Lipschitz \ref{eq:2.12} , atunci existen\c{t}a local\u{a} r\u{a}m\^{a}ne valabil\u{a}, dar f\u{a}r\u{a} unicitate. \^{I}ntr-adevar, \(k = k\left ( t,s,v \right )\) poate fi aproximat uniform pe \(D\) printr-o succesiune de func\c{t}ii netede (deci Lipschitzian, chiar \c{s}i \^{i}n toate variabilele), s\u{a} spunem \(\left ( k_{n} \right )_{n\in \mathbb{N}}\). Pentru a ob\c{t}ine o astfel de secven\c{t}\u{a} putem folosi, de exemplu, molificarea lui Friedrichs cu \(\varepsilon = \frac{1}{n}\).(Vezi subcapitolul 1.1)  De fapt, printr-un rezultat clasic, \(k = k\left ( t,s,v \right )\) poate fi chiar aproximat prin polinoame \^{i}n \(t,s,v\). Conform teoremei 16, pentru fiecare \(n \in \mathbb{N}\) exist\u{a} o func\c{t}ie unic\u{a} \(x_{n}\) care satisfice ecua\c{t}ia
\begin{displaymath}
x_{n}\left ( t \right ) = f\left ( t \right ) + \int_{a}^{t}k_{n}\left ( t,s,x_{n}\left ( s \right ) \right )ds, \forall t\in \left [ a,a+\delta  \right ], \label{eq:2.15} \tag{2.15}
\end{displaymath}
unde \(\delta = min\left \{ b-a, \frac{\left ( c-d \right )}{\hat{M}} \right \}\), cu \(\hat{M}\) fiind cea mai mic\u{a} limit\u{a} superioar\u{a} a \(\left \{ sup_{D} \left | k_{n} \right |\right \}_{n\in \mathbb{N}}\), de exemplu, \(\hat{M} = sup_{\left ( t,s,v \right )\in D, n \in \mathbb{N}}\left | k_{n}\left ( t,s,v \right ) \right |\), ( care este finit\u{a} deoarece \(k_{n}\rightarrow k\) uniform \^{i}n D). Desigur, \(\hat{\delta }\) este mai mic dec\^{a}t \({\delta }\) dat de teorema 16. Se vede u\c{s}or c\u{a} \(\left ( x_{n} \right )\)  \^{i}ndeplineste condi\c{t}iile Criteriul Arzelà-Ascoli (Vezi subcapitolul 1.1), deci exist\u{a} o subsecven\c{t}\u{a} \(\left ( x_{n_{j}} \right )_{j\in \mathbb{N}}\) care converge uniform pe \(\left [ a, a+ \delta  \right ]\) la o func\c{t}ie \(x \in C \left [ a, a+ \delta  \right ]\).

\noindent Dac\u{a} lu\u{a}m \(j\rightarrow \infty\) \^{i}n \ref{eq:2.15} cu \(n:= n_{j}\), deducem c\u{a} \(x\) satisfice ecua\c{t}ia \ref{eq:2.10} \^{i}n \(\left [ a,a+ \delta  \right ]\).

\noindent Remarci similar sunt valabile \c{s}i pentru teorma 17.
\end{comentary}
\begin{comentary}
Problemele calitative, precum continuitatea solu\c{t}iilor locale, existen\c{t}e pe semiaxa \(\left [ a, \infty  \right ]\) , comportamentul solu\c{t}iilor la sf\^{a}r\c{s}itul intervalelor de existen\c{t}\u{a}, sunt evitate aici.
\end{comentary}
\begin{comentary}
Toate observa\c{t}iile de mai sus se aplic\u{a} ecua\c{t}iilor Volterra liniare \c{s}i neliniare din \(\mathbb{R}^{k}, k\in \mathbb{N}, k\geq 2\), cu u\c{s}oare modific\u{a}ri evidente.
\end{comentary} 