\documentclass[a4paper,12pt,oneside]{report}
\usepackage{OvidiusFMI}
\usepackage{times}
\usepackage{graphicx}
\usepackage{hyperref}
\usepackage{color,xcolor}
\usepackage{amsmath}
\usepackage{framed}
\usepackage{indentfirst}
\usepackage{enumerate}
\usepackage[shortlabels]{enumitem}
\usepackage{listings}
\usepackage{amsmath,amsfonts,amssymb,amsthm,epsfig,epstopdf,url,array}
\usepackage{multicol,multirow}
\include{custom-lst-style}

\newtheorem{definition}{Defini\c tie}
\newtheorem{proposition}{Propozi\c tie}
\newtheorem{demonstration}{Demonstra\c tie}
\newtheorem{example}{Exemplu}
\newtheorem{theorem}{Teorem\u a}
\newtheorem{solve}{Rezolvare}
\newtheorem{corollary}{Corolar}

\facultatea{Matematic\u a \c si Informatic\u a}
\specializarea{Informatic\u a}
\teza{Licen\c t\u a}
\titlu{Licen\c t\u a}
\coordonatorPrincipal{Cosma Lumini\c ta}
\autor{T\u anase Ramona Elena}
\data{2021}

\begin{document}
\maketitle

\pagenumbering{roman}
\tableofcontents

\pagenumbering{arabic}
%
%
%CAPITOLUL 1
%
%
\chapter{Ecuatii Integrale}

Acest capitol este o introducere la teoria ecuatiilor liniare Volterra si Fresholm. Sunt abordate si unele aspect legate de anumite extensii neliniare.

\section{Ecuatii Volterra}

Incepem cu ecuatii scalare si liniare Volterra. Exista doua tipuri de astfel de ecuatii care sunt cele mai relevante pentru aplicatii, si anume
\begin{displaymath}
  f\left ( t \right ) = \int_{a}^{t}k\left ( t,s \right )x\left ( s \right )ds,    a\leq t\leq b (1.1.1)
\end{displaymath}
si 
\begin{displaymath}
  x\left ( t \right ) = f\left ( t \right ) + \int_{a}^{t}k\left ( t,s \right )x\left ( s \right )ds, a\leq t\leq b (1.1.2)
\end{displaymath}

unde \(a,b \in \mathbb{R}, a< b, f\in C\left [ a,b \right ]:= C\left ( \left [ a,b \right ];\mathbb{R} \right ) , k\in C\left (\Delta   \right ):= C\left ( \Delta ;\mathbb{R} \right )\) (numit nucleu), cu \(\Delta =\left \{ \left ( t,s \right )\in \mathbb{R}^{2};a\leq s\leq t\leq b \right \};\) si \(x=x\left ( t \right )\) denota functia necunoscuta care se cauta in spatiul \(C\left [ a,b \right ]\). 
Ecuatia (1.1.1) este cunoscuta ca prima ecuatie Volterra , in timp ce ecuatia (1.1.2) este cunoscuta ca cea de-a doua ecuatie Volterra. In cele ce urmeaza vom examina ecuatia (1.1.2). Vom arata mai tarziu ca ecuatia (1.1.1) se reduce la (1.1.2) in conditii adecvate. 



\bibliographystyle{unsrt}
\setlength{\baselineskip}{\normalbaselineskip}
\setlength{\parskip}{0pt}
\bibliography{refs}
\end{document}