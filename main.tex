\documentclass[a4paper,12pt,oneside]{report}
\usepackage{OvidiusFMI}
\usepackage{times}
\usepackage{graphicx}
\usepackage{hyperref}
\usepackage{color,xcolor}
\usepackage{amsmath}
\usepackage{framed}
\usepackage{indentfirst}
\usepackage{enumerate}
\usepackage[shortlabels]{enumitem}
\usepackage{listings}
\usepackage{amsmath,amsfonts,amssymb,amsthm,epsfig,epstopdf,url,array}
\usepackage{multicol,multirow}
\include{custom-lst-style}

\newtheorem{definition}{Defini\c tie}
\newtheorem{proposition}{Propozi\c tie}
\newtheorem{demonstration}{Demonstra\c tie}
\newtheorem{example}{Exemplu}
\newtheorem{theorem}{Teorem\u a}
\newtheorem{solve}{Rezolvare}
\newtheorem{comentary}{Comentariu}

\facultatea{Matematic\u a \c si Informatic\u a}
\specializarea{Informatic\u a}
\teza{Licen\c t\u a}
\titlu{Licen\c t\u a}
\coordonatorPrincipal{Cosma Lumini\c ta}
\autor{T\u anase Ramona Elena}
\data{2021}

\begin{document}
\maketitle

\pagenumbering{roman}
\tableofcontents

\pagenumbering{arabic}
%
%
%CAPITOLUL 1
%
%
\chapter{Ecuatii Integrale}

Acest capitol este o introducere la teoria ecuatiilor liniare Volterra si Fresholm. Sunt abordate si unele aspect legate de anumite extensii neliniare.

\section{Ecuatii Volterra}

Incepem cu ecuatii scalare si liniare Volterra. Exista doua tipuri de astfel de ecuatii care sunt cele mai relevante pentru aplicatii, si anume
\begin{displaymath}
  f\left ( t \right ) = \int_{a}^{t}k\left ( t,s \right )x\left ( s \right )ds,    a\leq t\leq b (1.1.1)
\end{displaymath}
si 
\begin{displaymath}
  x\left ( t \right ) = f\left ( t \right ) + \int_{a}^{t}k\left ( t,s \right )x\left ( s \right )ds, a\leq t\leq b (1.1.2)
\end{displaymath}

unde \(a,b \in \mathbb{R}, a< b, f\in C\left [ a,b \right ]:= C\left ( \left [ a,b \right ];\mathbb{R} \right ) , k\in C\left (\Delta   \right ):= C\left ( \Delta ;\mathbb{R} \right )\) (numit nucleu), cu \(\Delta =\left \{ \left ( t,s \right )\in \mathbb{R}^{2};a\leq s\leq t\leq b \right \};\) si \(x=x\left ( t \right )\) denota functia necunoscuta care se cauta in spatiul \(C\left [ a,b \right ]\). 
Ecuatia (1.1.1) este cunoscuta ca prima ecuatie Volterra , in timp ce ecuatia (1.1.2) este cunoscuta ca cea de-a doua ecuatie Volterra. In cele ce urmeaza vom examina ecuatia (1.1.2). Vom arata mai tarziu ca ecuatia (1.1.1) se reduce la (1.1.2) in conditii adecvate. 

\subsection{Teorema}

Existetenta si Unicitatea


In conditiile de mai sus exista o solutie unica \(x\in C\in \left [ a,b \right ]\) la ecuatia (1.1.2).
Vom prezenta mai jos trei demonstratii diferite.

\subsection{Demonstratie}

Notam \(K = sup_{\left ( t,s \right )\in \Delta }\left | k\left ( t,s \right ) \right |\) care este finite deoarece \(\Delta\) este un subset compact al lui \(\mathbb{R}^{2}\). Presupunem intr-o prima etapa ca 
\(K\left ( b-a \right ) < 1. (1.1.3)\)
Se considera \(X = C\left [ a,b \right ]\) echipat cu sup-norma obisnuita \(\left \| g \right \| = sup_{a\leq t\leq b}\left | g\left ( t \right ) \right |\),  si metrica corespunzatoare , \(d\left ( g_{1}, g_{2} \right ) = \left \| g_{1} - g_{2} \right \|\). 
Definim \(T : X \rightarrow X\) prin 
\begin{displaymath}
  \left ( Tg \right )\left ( t \right ) = f\left ( t \right ) + \int_{a}^{t}k\left ( t,s \right )g\left ( s \right )ds, t\in \left [ a,b \right ], g\in X. (1.1.4)
\end{displaymath}
	 Este clar din (1.1.4) ca \(T\) mapeaza \(X\). Avem
\begin{displaymath}
  \left | \left ( Tg_{1} \right )\left ( t \right ) - \left ( Tg_{2} \right )\left ( t \right )  \right | = \left | \int_{a}^{t}k\left ( t,s \right )\left [ g_{1}\left ( s \right ) - g_{2}\left ( s \right ) \right ]ds \right |\leq
\end{displaymath}
\begin{displaymath}
  \leq \int_{a}^{t}\left | k\left ( t,s \right ) \right |\cdot \left | g_{1}\left ( s \right ) - g_{2}\left ( s \right ) \right |ds \leq K \left ( b-a \right )\left \| g_{1} - g_{2} \right \|
\end{displaymath}


pentru orice \(g_{1}, g_{2} \in X\), si orice  \(t\in \left [ a,b \right ]\). 
Prin urmare 
\begin{displaymath}
  d\left ( Tg_{1} , Tg_{2}\right )\leq K\left ( b-a \right )d\left ( g_{1}, g_{2} \right ), 
\end{displaymath}
adica \(T\) este o contractie (conform (1.1.3)).


Conform Principiului Contractiei Banach (cap 2), \(T\) are un punct fix unic \(x \in  X\) care este clar Solutia unica a cuatiei (1.1.2).
	Daca conditia (1.1.3) nu este indeplinita, atunci consideram o subdiviziunea a intervalului \(\left [ a,b \right ]\), de exemplu 
\(a = t_{0}< t_{1}< \cdots < t_{N-1}< t_{N} = b,
\)
unde \(t_{j} = a + jh\) pentru \(j = 1,2,….,N, h = \frac{\left ( b-a \right )}{N}\), cu \(N\) suficient de mare incat \(Kh < 1\). In special, \(K\left ( t_{1}  - t_{0}\right ) = Kh< 1\), deci de mai sus rezulta ca (1.1.2) are o solutie unica \(x_{1} = x_{1}\left ( t \right ) pe intervalul \left [ t_{0} , t_{1} \right ] = \left [ a, t_{1} \right ]\), adica, 
\begin{displaymath}
  x_{1}\left ( t \right ) = f\left ( t \right ) + \int_{a}^{t}k\left ( t,s \right )x_{1}\left ( s \right )ds, t \in \left [ a, t_{1} \right ]. 
\end{displaymath}

	Fie ecuatia 
\begin{displaymath}
  x_{1}\left ( t \right ) = f\left ( t \right ) + \int_{a}^{t}k\left ( t,s \right )x_{1}\left ( s \right )ds + \int_{t_{1}}^{t_{2}}k\left ( t,s \right )x_{2}\left ( s \right )ds + \int_{t_{2}}^{t}k\left ( t,s \right )x_{3}\left ( s \right )ds,  t\in \left [ t_{1}, t_{2} \right ], 
\end{displaymath}
\begin{displaymath}
  f\left ( t \right ) + \int_{a}^{t}k\left ( t,s \right )x_{1}\left ( s \right )ds =:f_{1}\left ( t \right ) \in C \left [ t_{1} , t_{2} \right ].
\end{displaymath}
	Deoarece \(K\left ( t_{2} - t_{1} \right ) = Kh < 1\), rezulta din argumentul de mai sus ca aceatsa ecuatie are o solutie unica \(x_{2} \in C\left [ t_{1}, t_{2} \right ]\) si, evident, \(x_{2} \left ( t_{1} \right ) = x_{1} \left ( t_{1} \right )\). In mod similar, exista o functie unica \(x_{3} \in C\left [ t_{2} , t_{3} \right ]\) care satisfice urmatoarea ecuatie , pentru orice \(t\in \left [ t_{2} , t_{3} \right ],\)
\begin{displaymath}
  x_{3}\left ( t \right ) = f\left ( t \right ) + \int_{a}^{t_{1}}k\left ( t,s \right )x_{1}\left ( s \right )ds + \int_{t_{1}}^{t_{2}}k\left ( t,s \right )x_{2}\left ( s \right )ds + \int_{t_{2}}^{t}k\left ( t,s \right )x_{3}\left ( s \right )ds, 
\end{displaymath}

si \(x_{3}\left ( t_{2} \right ) = x_{2}\left ( t_{2} \right )\). Continuand aceasta procedura obtinem o solutie \(x\in C\left [ t_{0}, t_{N} \right ] = C\left [ a,b \right ]\) a ecuatiei (1.1.2) definite de \(x\left ( t \right ) = x_{j}\left ( t \right )\) pentru \(t\in \left [ t_{j-1}, t_{j} \right ], j = 1,2,....,N\). Solutia \(x\) este evident unica. 

\subsection{Demonstratie}

Din nou, luam in considerare operatorul \texttt{T} este definit de ecuatia (1.1.4), unde \(X\) este acelasi ca mai sus. Se vede usor ca
\begin{displaymath}
  \left | \left ( Tg_{1} \right )\left ( t \right )  - \left ( Tg_{2} \right )\left ( t \right )\right |\leq K\left \| g_{1}  - g_{2}\right \|\left ( t-a \right ), \forall t\in \left [ a,b \right ], g_{1}, g_{2} \in X. 
\end{displaymath}

	In consecinta, pentru \(T^{2} = T \circ T\) obtinem 
\begin{displaymath}
  \left | \left ( T^{2}g_{1} \right )\left ( t \right )  - \left ( T^{2}g_{2} \right )\left ( t \right )\right |\leq \int_{a}^{t}\left | k\left ( t,s \right ) \right |\cdot \left | \left ( Tg_{1} \right )\left ( s \right ) - \left ( Tg_{2} \right )\left ( s \right ) \right |ds\leq
\end{displaymath}

\begin{displaymath}
  \leq K^{2}\left \| g_{1} - g_{2} \right \|\int_{a}^{t}\left ( s-a \right )ds = \frac{K^{2}\left ( t-a \right )^{2}}{2!}\left \| g_{1} - g_{2} \right \|. 
\end{displaymath}


	Se poate demonstra prin inductie faptul ca:
\begin{displaymath}
  \left | \left ( T^{k}g_{1} \right ) \left ( t \right ) - \left ( T^{k}g_{2} \right )\left ( t \right )\right |\leq \frac{K^{k}\left ( t-a \right )^{k}}{k!}\left \| g_{1} - g_{2} \right \|\leq \frac{K^{k}\left ( b-a \right )^{k}}{k!}\left \| g_{1}- g_{2} \right \|. 
\end{displaymath}

pentru orice  \(t\in \left [ a,b \right ], g_{1}, g_{2} \in X, k = 1,2,.....\) Acum luam supremul pentru a gasi
\begin{displaymath}
  d\left ( T^{k}g_{1} , T^{k}g_{2}\right ) \leq \frac{K^{k}\left ( b - a  \right )^{k}}{k!}\left \| g_{1} - g_{2}\right \|, \forall g_{1}, g_{2} \in X, k = 1,2,... ( 1.1.5)
\end{displaymath}

	 Cum \(K^{k} \left ( b-a \right )^{\frac{k}{k!}}\rightarrow \infty\) , deoarece \(k\rightarrow \infty, T^{k}\) este o contractie pentru \(k\) suficient de mare ( conform 1.1.5). Conform remarcei 2, \(T\) are un punct fix unic \(x \in X\), care este solutia unica a lui (1.1.2). 

\subsection{Demonstratie}

Fie \(T\) acelasi operator ca mai ianinte, dar luam in considerer o alta norma pe \(X = C\left [ a,b \right ]\), norma \(/bielecki\) , care este definite de \(\left \| g \right \|_{B} = sup e^{-Lt}\left | g\left ( t \right ) \right |\). 
cu \(L\) o constanta pozitiva astfel incat \(\frac{K}{L} < 1\). Aceasta este intr-adevar o norma pe \(X\), care este echivalenta cu norma obisnuita. Notam cu \(d_{B}\), metrica generata de \(\left \| \cdot \right \|_{B}\). Avem pentru orice \(t\in \left [ a,b  \right ]\) si \(g_{1}, g_{2} \in X\) 
\begin{displaymath}
  \left | \left ( Tg_{1} \right )\left ( t \right ) - \left ( Tg_{2} \right )\left ( t \right ) \right | \leq  \int_{a}^{t} \left | k\left ( t,s \right ) \right |e^{Ls}e^{-Ls}\left | g_{1}\left ( s \right ) - g_{2}\left ( s \right ) \right |ds \leq
\end{displaymath}

\begin{displaymath}
  \leq K\left \| g_{1} - g_{2} \right \|_{B} \int_{a}^{t}e^{Ls}ds = \frac{K \left \| g_{1} - g_{2} \right \|_{B}}{L}\left ( e^{Lt} - e^{La}\right ),
\end{displaymath}

 
astfel incat 
\begin{displaymath}
  e^{-Lt}\left | \left ( Tg_{1} \right ) \left ( t \right ) - \left ( Tg_{2} \right )\left ( t \right )\right |\leq \frac{K}{L}\left \| g_{1} - g_{2}\right \|_{B}\left ( 1 - e^{-L\left ( t-a \right )} \right ) \leq  \frac{K}{L}\left \| g_{1}  - g_{2}\right \|_{B}. 
\end{displaymath}

	Acum luam supremul pentru \(t \in \left [ a,b \right ]\) pentru a gasi
\(d_{B}\left ( Tg_{1}, Tg_{2} \right ) \leq \frac{K}{L}d_{B}\left ( g_{1} , g_{2}\right ), \forall g_{1}, g_{2}\in X. \)
	 Cum \(\frac{K}{L} < 1, T\) este o contractie in raport cu \(d_{B}\), prin urmare concluzia teoremei urmeaza din nou Principiul contradictiei Banach. 
	Sa presupunem ca sunt indeplinite conditiile de mai sus pentru \(f\) si \(k\). Pentru \(n\in \mathbb{N}, t\in \left [ a,b \right ]\), vom avea
\begin{displaymath}
  x_{n}\left ( t \right ) = f\left ( t \right ) + \int_{a}^{t}k\left ( t,s \right )x_{n-1}\left ( s \right )ds,
x_{0}\left ( t \right ) = f\left ( t \right ).
\end{displaymath}
 
	In mod clar, \(x_{n} \in X = C\left [ a,b \right ]\) pentru orice \(n\). De fapt, secventa de mai sus \(\left ( x_{n} \right )_{n\geq 0}\) poate fi exprimata ca 
\begin{displaymath}
  x_{n} = Tx_{n-1}, n\in \mathbb{N}; x_{0} = f,
\end{displaymath}

unde \(T : X \rightarrow X\) este operatorul definit de (1.1.4). Deci, \(\left (x_{n}  \right )\) este secventa de aproximari successive (associate operatorului T) care a fost folosita in demonstrarea Principiului Contradictiei Banach (Capitolul 2) . Aici luam in considerare o anumita functie de pornire, \(x_{0} = f\). Din demonstratia Principiului Contradictiei Banach stim ca \(\left ( x_{n} \right )\) converge in \(\left ( C\left [ a,b \right ], \left \| \cdot  \right \|_{B} \right )\) catre punctul sau fix unic \(T\), adica \(\left ( x_{n} \right )\) converge uniform in \(\left [ a,b \right ]\) la Solutia unica \(x\) a ecuatiei (1.1.2). Pe de alta parte, avem pentru orice \(t\in \left [ a,b \right ]\)
\begin{displaymath}
  x_{1}\left ( t \right ) = f\left ( t \right ) + \int_{a}^{t}k\left ( t,s \right )f\left ( s \right )ds,
\end{displaymath}

\begin{displaymath}
  x_{2}\left ( t \right ) = f\left ( t \right ) + \int_{a}^{t}k\left ( t,s \right )\left [ f\left ( s \right ) + \int_{a}^{s} k\left ( s,  \tau  \right )f\left (\tau  \right )d\tau  \right ]ds
\end{displaymath}

 \begin{displaymath}
  =  f\left ( t \right ) + \int_{a}^{t}k\left ( t,s \right )f\left ( s \right )ds + \int_{a}^{t}\int_{a}^{s}k\left ( t,s \right )k\left ( s,\tau  \right )f\left (\tau  \right ) d\tau ds.
\end{displaymath}
 
	Putem schimba integrarea pentru a afla ca ultima integral este egala cu 
\begin{displaymath}
  \int_{a}^{t}\left [ \int_{\tau }^{t}k\left ( t,\tau  \right )k\left ( s,\tau  \right )ds \right ]f\left (\tau   \right )d\tau, 
\end{displaymath}

Deci prin simpla reetichetare \(\tau\) si \(s\) avem 
\begin{displaymath}
  \int_{a}^{t}\left [ \int_{s}^{t}k\left ( t,\tau  \right )k\left ( \tau ,s \right )d\tau  \right ]f\left ( s \right )ds 
\end{displaymath}

si au un nucleu nou,  \(k_{2}\). In general, daca notam pentru \(n – 2,3,…\).
\begin{displaymath}
  k_{n}\left ( t,s \right ) := \int_{s}^{t}k\left ( t,\tau  \right )k_{n-1}\left ( \tau ,s \right )d\tau,
\end{displaymath}

\begin{displaymath}
  k_{1}\left ( t,s \right ) := k\left ( t,s \right ), 
\end{displaymath}

avem, pentru \(n = 1,2,…..\)
\begin{displaymath}
  x_{n}\left ( t \right ) = f\left ( t \right ) + \int_{a}^{t}\left [ \sum_{j= 1}^{n}k_{j}\left ( t,s \right ) \right ]f\left ( s \right )ds. (1.1.6)
\end{displaymath}

	Deoarece \(k\) este continuu pe multimea compacta \(\Delta\), avem pentru orice \(\left ( t,s \right ) \in \Delta\),
\begin{displaymath}
   \left | k_{1}\left ( t,s \right ) \right |\leq K< \infty
\left | k_{2}\left ( t,s \right ) \right |\leq K^{2}\left ( t-s \right ),
\left | k_{3}\left ( t,s \right ) \right |\leq K^{3}\int_{s}^{t}\left | \tau - s \right |d\tau  = K^{3}\frac{\left ( t-s \right )^{2}}{2!},
\end{displaymath}
 
\begin{displaymath}
  \vdots
\left | k_{n}\left ( t,s \right ) \right |\leq K^{n}\frac{\left ( t-s \right )^{n-1}}{\left ( n-1 \right )!}\leq K^{n}\frac{\left ( b-a \right )^{n-1}}{\left ( n-1 \right )!}. 
\end{displaymath}

	Prin M-testul Weierstrass, seria \(\sum_{n=1}^{\infty }k_{n}\left ( t,s \right )\), converge clar uniform pe \(\Delta\) deoarece 
\begin{displaymath}
  \sum_{n=1}^{\infty }\frac{K^{n}\left ( b-a \right )^{n-1}}{\left ( n-1 \right )!}< \infty.
\end{displaymath}

	Inseamna 
\(R\left ( t,s \right ) = \sum_{n=1}^{\infty }k_{n}\left ( t,s \right ), 
care este in X\left ( \Delta  \right ).\) Avand \(n\rightarrow \infty\) in (1.1.6) deduce faptul ca 
\begin{displaymath}
  x\left ( t \right ) = f\left ( t \right ) + \int_{a}^{t}R\left ( t,s \right )f\left ( s \right )ds, t\in \left [ a,b \right ]. (1.1.7)
\end{displaymath}

	Numim \(R\left ( t,s \right )\) nucleul rezolutiv. Acesta depinde de \(k\) dar este independent fata de \(f\), astfel incat odata ce gasim \(R\left ( t,s \right )\), avem Solutia (1.1.2) pentru orice \(f\) ( conform 1.1.7). 
	Obervam faptul ca 
\begin{displaymath}
  \sum_{n=2}^{N+1}k_{n}\left ( t,s \right ) = \int_{s}^{t}k\left ( t,\tau  \right )\sum_{n=2}^{N+1}k_{n-1}\left ( \tau ,s \right )d\tau,
\end{displaymath}

ceea ce implica 
\begin{displaymath}
  -k\left ( t,s \right ) + \sum_{n=1}^{N+1}k_{n}\left ( t,s \right ) = \int_{s}^{t}k\left ( t,\tau  \right )\sum_{n=1}^{N}k_{n}\left ( \tau ,s \right )d\tau .
\end{displaymath}

	Luand \(n\rightarrow \infty\) vom constatam ca \(R\) satisface 
\begin{displaymath}
  R\left ( t,s \right ) = k\left ( t,s \right ) + \int_{s}^{t}k\left ( t,\tau  \right )R\left ( \tau ,s \right )d\tau , \forall \left ( t,s \right )\in \Delta , 
\end{displaymath}

care este o ecuatie Volterra similara cu (1.1.2). 
	Acum sa examinam ecuatia (1.1.1). Presuounem ca 
\begin{displaymath}
  f\in C^{1}\left [ a,b \right ], si k, \frac{\partial k}{\partial t}\in C\left ( \Delta  \right ), k\left ( t,t \right )\neq 0 pentru orice t \in \left [ a,b \right ]. (H)
\end{displaymath}

	Presupunem de asemenea ca \(f\left ( a \right ) = 0\) care este o conditie necesara pentru (1.1.1) pentru a avea o solutie. Daca (1.1.1) ae o solutie \(x\in C\left [ a,b \right ]\), atunci diferentierea (1.1.1) ne da 
\begin{displaymath}
  {f}'\left ( t \right ) = \frac{d}{dt}\int_{a}^{t}k\left ( t,s \right )x\left ( s \right )ds (1.1.8) 
\end{displaymath}
 
\begin{displaymath}
  = k\left ( t,t \right )x\left ( t \right ) + \int_{a}^{t} k_{t}\left ( t,s \right )x\left ( s \right )ds, t\in \left [ a,b \right ],
\end{displaymath}
 
care este echivalenta cu urmatarea ecuatie integrala de grasul al doilea, 
\begin{displaymath}
  x\left ( t \right ) = \frac{{f}'\left ( t \right )}{k\left ( t,t  \right )} + \int_{a}^{t} \left [ \frac{-k_{t}\left ( t,s \right )}{k\left ( t,t \right )} \right ]x\left ( s \right )ds. (1.1.9)
\end{displaymath}

	Deci x este, de asemenea, o solutie a ecuatiei (1.1.9). Pe de alta parte, stim din teorema anterioara faptul ca (1.1.9) are o solutie unica \(x\in C\left [ a,b \right ]\). Acest x, este de asemenea, o solutie a ecuatiei (1.1.1). Aceasta urmeaza prin integrarea lui (1.1.8) peste \(\left [ a,t \right ]\) si folosind conditia \(f\left ( a \right ) = 0\). Astfel am demonstrate urmatorul rezultat. 

\subsection{Teorema}

In conditiile (H) de mai sus, plus \(f\left ( a \right ) = 0\), ecuatia (1.1.1) are o solutie unica \(x \in C\left [ a,b \right ]\).
	Continuam cu ecuatia neliniara Volterra 
\begin{displaymath}
  x\left ( t \right ) = f\left ( t \right ) + \int_{a}^{t}k\left ( t,s,x\left ( s \right ) \right )ds, t\in \left [ a,b \right ], (1.1.10)
\end{displaymath}

si demonstram urmatorul rezultat general. 

\subsection{Teorema}

Presupunem ca \(f\in C\left [ a,b \right ]\), \(k\in C\left ( D \right )\), unde 
\begin{displaymath}
  D:= \Delta \times \mathbb{R} = \left \{ \left ( t,s,v \right )\in \mathbb{R}^{3}; a\leq s\leq t\leq b, v\in \mathbb{R}\right \}, 
\end{displaymath}
si exista un \(K> 0\) astfel incat
\begin{displaymath}
  \left | k\left ( t,s,v \right )  - k\left ( t,s,w \right )\right |\leq K\left | v-w \right |\forall a\leq s\leq t\leq b; v,w\in \mathbb{R}.(1.1.11)
\end{displaymath}

Atunci exista o functie unica \(x\in C\left [ a,b \right ]\) care satisfice ecuatia (1.1.10) in \(\left [ a,b \right ]\).

\subsection{Demonstratie}

Fie \(X = C \left [ a,b \right ]\) echipat cu norma Bielecki si definim \(T : X \rightarrow X\) prin 
\begin{displaymath}
  \left ( Tg \right )\left ( t \right ) = f\left ( t \right ) + \int_{0}^{t}k\left ( t,s,g\left ( s \right ) \right )ds, \forall t \left [ a,b \right ], g\in X. 
\end{displaymath}

	Concluzia este urmata de Principiului Contradictiei Banach in mod similar, ca in Demonstartia 3, a teoremei 1.
	Teorema 1.3 ofera o solutie globala in sensul ca intervalul de existenta este intregul interval \(\left [ a,b \right ]\). In mod evident, aceasta este o generalizare a Teoremei 1.1. Intr-adevar, pentru a obtine Teorema 1.1 este suficient sa presupunem ca k este linear in a treia variabila, adica \(k:= k\left ( t,s \right )v, a\leq s\leq t\leq b, v\in \mathbb{R}\), cu \(k \in C\left ( \Delta  \right )\) astfel incat conditia Lipschitz (1.1.11) este satisfacuta automat. 
	Acum sa examinam un caz in care Solutia rezultata este doar una locala, adica domeniul sau poate sa nu fie intregul interval \(\left [ a,b \right ]\). 

\subsection{Teorema}

Sa presupunem ca \(f\in C\left [ a,b \right ] , k = k\left ( t,s,v \right ) \in C\left ( D \right )\), unde \(D := \Delta \times \left [ x_{0} - c, x_{0} + c \right ] = \left \{ \left ( t,s,v \right ) \in \mathbb{R}^{3} ; a \leq s\leq t\leq b, \left | v - x_{0} \right |\leq c\right \}\), cu \(x_{0} \in \mathbb{R}\) si \(c\in \left ( 0, \infty  \right )\). Daca in plus, exista \(K > 0\) astfel incat
\begin{displaymath}
  \left | k\left ( t,s,v \right ) - k\left ( t,s,w \right )\right | \leq K \left | v - w \right | \forall \left ( t,s,v \right ), \left ( t,s,w \right ) \in D, (1.1.12)
\end{displaymath}

iar pentru unii \(d \in \left [ 0,c \right )\) 
\begin{displaymath}
  \left | f\left ( t \right ) - x_{0}\right | \leq d, \forall t \in \left [ a,b \right ], (1.1.13)
\end{displaymath}

atunci exista o functie unica \(x\in C \left [ a, a + \delta  \right ]\) care satisfice ecuatia (1.1.10) in  \(\left [ a, a + \delta  \right ]\), unde 
\begin{displaymath}
  \delta  = min \left \{ b-a, \frac{\left ( c-d \right )}{M} \right \}, M = sup \left \{ \left | k\left ( t,s,v \right ) \right |;\left ( t,s,v \right ) \in D \right \}.
\end{displaymath}
(Se presupune ca M este pozitiv deoarece cazul M = 0 este trivial)


\begin{demonstration}
Se considera spatiul \(C \left [ a, a + \delta  \right ]\) cu sup-norma obisnuita si metrica d generate de aceasta.  Indica 
\begin{displaymath}
  Y = \left \{ g\in C \left [ a,a+\delta  \right ] ; \left | g\left ( t \right ) - x_{0}\right | \leq c, \forall t \in \left [ a, a+\delta  \right ]\right \}. 
\end{displaymath}

	In mod clar \(\left ( Y,d \right )\) este un spatiu metric complet ( deoarece \(/y\) este o submultime inchisa a \(\left ( C \left [ a, a + \delta  \right ] , d \right )\). Ca de obicei, definim un operator \(T\) prin
 \begin{displaymath}
  \left ( Tg \right )\left ( t \right ) = f\left ( t \right ) + \int_{a}^{t}k\left ( t,s,g\left ( s \right ) \right )ds, t\in \left [ a, a + \delta  \right ], g\in Y. 
\end{displaymath}

	Sa aratam ca \(Y\) o sa fie inclus in \(T\). Intr-adevar, pentru toate \(g\in Y si t\in \left [ a, a+ \delta  \right ]\) vom avea (vezi (1.1.13)). 
\begin{displaymath}
  \left | \left ( Tg \right )\left ( t \right ) - x_{0}\right | \leq \left | f\left ( t \right )-x_{0} \right | + \int_{a}^{t}\left | k\left ( t,s,g\left ( s \right ) \right ) \right |ds \leq d + M\left ( t-a \right ) \leq  d+ M\delta  \leq  c,
\end{displaymath}
 
ceea ce dovedeste afirmatia. Prin argumente similar cu cele utilizate in Demonstratia 2 a Teoremei 11.1 deducem ca \(T^{k}\) este o contractie pe \(\left ( Y,d \right )\) pentru \(k\) suficient de mare. Deci \(T\) are un punct fix unic \(x \in Y\) care este Solutia unica a ecuatiei ( 1.1.10) in \(\left [ a, a + \delta  \right ]\). 
	Un alt rezultat de existent si unicitate se obtine daca k este definit pe un domeniu diferit, \(\tilde{D} = \left \{ \left ( t,s,v \right ) \in \mathbb{R}^{3}; a\leq s\leq t\leq b, \left | v - f\left ( s \right ) \right | \leq c \right \}, c\in \left ( 0, \infty  \right ), \)
care este o submultime compacta a lui \(\mathbb{R}^{3}\). Urmatorul rezultat face acest lucru precis. 
	
\end{demonstration}

\subsection{Teorema}

Presupunem ca \(f \in C \left [ a,b \right ] si k = k\left ( t,s,v \right ) \in C\left ( \tilde{D} \right )\), cu \(M = sup _{\tilde{D}}\left | k \right |> 0\). Daca, in plus, exista un \(K > 0\) astfel incat 
\begin{displaymath}
  \left | k\left ( t,s,v \right ) - k \left ( t,s,w \right ) \right | \leq K \left | v-w \right |, \forall \left ( t,s,v \right ) \in \tilde{D}, (1.1.14)
\end{displaymath}

atunci exista o functie unica \(x \in C \left [ a, a + \delta  \right ]\) care satisfice ecuatia (1.1.10)  in  \(\left [ a, a + \delta  \right ]\) , unde \(\delta = min \left \{ b-a, \frac{c}{M} \right \}\). 

\begin{demonstration}

Dovada este similara cu cea din Teorema 1.4 de mai sus. Aici domeniul operatorului T este ales convenabil 
\begin{displaymath}
  \tilde{Y} = \left \{ g \in C \left [ a, a+ \delta  \right ] ; \left | g\left ( t \right ) - f\left ( t \right ) \right | \leq c, \forall t \in \left [ a, a+ \delta  \right ] \right \}, 
\end{displaymath}

care este bila inchisa in \(\left ( C\left [ a, a+ \delta  \right ], d \right )\) centrata la \(f\) (restransa la \(\left [ a, a+ \delta  \right ]\) de raza \(c\). In mod evident, \(T\) este bine definit pe \(\tilde{Y}\) si \(\tilde{Y}\) inclus in el. De asemenea, se vede cu usurinta ca \(T^{k}\) este o contractie pentru un \(k \in \mathbb{N}\)  suficient de mare. Aceasta completeaza demonstratia (vezi remarca 1).   

\end{demonstration}

\subsection{Comentarii}

\begin{enumerate}[1.]
\item Daca in Teorema 1.4 presupunem \(d = 0 ( adica, f \equiv x_{0} )\) si \(k\) este independent de \(t\), adica \(k\left ( t,s,v \right ) = h \left ( s,v \right )\), atunci obtinem din nou o existeta binecunoscut si rezultat de unicitate pentru problema Cauchy 
\begin{displaymath}
  {x}'\left ( t \right ) = h\left ( t,x\left ( t \right ) \right ), x\left ( a \right ) = x_{0}.
\end{displaymath}
Vezi partea introductive a Sect 2.5 . Acelasi rezultat poate sa fie derivate din Teorema 1.5. 
\item Daca toate conditiile Teoremei 1.4 sunt indeplinite, cu exceptia conditiei Lipschitz (1.1.12) , atunci existent locala ramane valabila, dar fara unicitate. Intr-adevar, \(k = k\left ( t,s,v \right )\) poate fi aproximat uniform pe \(D\) printr-o succesiune de functii netede (deci Lipschitzian, chiar si in toate variabilele), sa spunem \(\left ( k_{n} \right )_{n\in \mathbb{N}}\). Pentru a obtine o astfel de secventa putem folosi, de exemplu, molificarea lui Friedrichs cu \(\varepsilon = \frac{1}{n}\).(Vezi Cap 5)  De fapt, printr-un rezultat classic, \(k = k\left ( t,s,v \right )\) poate fi chiar aproximat prin olinoame in \(t,s,v\). conform Teoremei 1.4, pentru fiecare \(n \in \mathbb{N}\) exista o funtie unica \(x_{n}\) care satisfice ecuatia
\begin{displaymath}
  x_{n}\left ( t \right ) = f\left ( t \right ) + \int_{a}^{t}k_{n}\left ( t,s,x_{n}\left ( s \right ) \right )ds, \forall t\in \left [ a,a+\delta  \right ], (1.1.15)
\end{displaymath}

unde \(\delta = min\left \{ b-a, \frac{\left ( c-d \right )}{\hat{M}} \right \}\), cu \(\hat{M}\) fiind cea mai mica limita superioara a \(\left \{ sup_{D} \left | k_{n} \right |\right \}_{n\in \mathbb{N}}\), de exemplu, \(\hat{M} = sup_{\left ( t,s,v \right )\in D, n \in \mathbb{N}}\left | k_{n}\left ( t,s,v \right ) \right |\), ( care este finit deoarece \(k_{n}\rightarrow k\) uniform in D). Desigur, \(\hat{\delta }\) este mai mic decat \({\delta }\) dat de Teorema 1.4. Se vede usor ca \(\left ( x_{n} \right )\)  indeplineste conditiile Criteriul Arzelà-Ascoli (Vezi Capitolul 2), deci exista o subsecventa \(\left ( x_{n_{j}} \right )_{j\in \mathbb{N}}\) care converge uniform pe \(\left [ a, a+ \delta  \right ]\) la o functie \(x \in C \left [ a, a+ \delta  \right ]\). Luand \(j\rightarrow \infty\) in (1.1.15) cu \(n:= n_{j}\), deduce ca \(x\) satisfice Ecuatia (1.1.10) in \(\left [ a,a+ \delta  \right ]\). Remarci similar sunt valabile pentru Teorma 1.5. 
\item Problemele calitative, precum continuitatea solutiilor locale, existent pe semiaxa \(\left [ a, \infty  \right ]\) , comportamentul solutiilor la sfarsitul intervalelor de existent, sunt evitate aici. 
\item Toate observatiile de mai sus se aplica ecuatiilor Volterra liniare si neliniare din \(\mathbb{R}^{k}, k\in \mathbb{N}, k\geq 2\), cu usoare modificari evidente. 
\end{enumerate}

\section{Ecuatii Fredholm}

In cele ce urmeaza \(\mathbb{K}\) este fie \(\mathbb{R}\), fie \(\mathbb{C}\). Consideram in \(\mathbb{K}\) ecuatia integrala \(x\left ( t \right ) = f\left ( t \right ) + \int_{a}^{b}k\left ( t,s \right )x\left ( s \right )ds, t\in \left [ a,b \right ], (1.2.16)\)
unde \(a,b \in \mathbb{R}, a< b, f\in C\left ( \left [ a,b \right ]; \mathbb{K}\right )\) si \(k\in C\left ( \left [ a,b \right ] \times \left [ a,b \right ]; \mathbb{K}\right ).\) Aici preferam \(\mathbb{K}\) in loc de \(\mathbb{R}\), deoarece unele aspect specific sunt mai bine descries in acest cadru. Ecuatia (1.2.16) ste cunoscuta ca ecuatia Fredholm (uneori este numita a doua ecuatie a lui Fredholm). Ea implica un interval fix de integrare si este fundamental diferita de Ecuatia (1.1.2).  O prima remarca care confirma aceasta afirmaie este ca , in timp ce ecuatia Volterra corespunzatoare (a doua ecuatie) are intotdeauna o solutie (unica, continua) in \(\left [ a,b \right ]\), Ecuatia (1.2.16) poate sa nu aibao solutie in unele cazuri. De exemplu, presupunand ca exista o solutie \(x \in C\left [ 0,1 \right ] := C\left ( \left [ 0,1 \right ]; \mathbb{R} \right )\) a ecuatiei ( 9 pg 41) 
\begin{displaymath}
  x\left ( t \right ) = t + \int_{0}^{1} k\left ( t,s \right )x\left ( s \right )ds, t\in \left [ 0,1 \right ], (1.2.17)
\end{displaymath}

unde
\begin{displaymath}
  \left\{\begin{matrix}
\pi ^{2} \left ( s \right )\left ( 1-t \right ) , & s\leq t\\ 
 & \\ \pi ^{2}t\left ( 1-s \right ),  & t\leq s
\end{matrix}\right.
\end{displaymath}

rezulta prin diferentierea ecuatiei  (1.2.17) de doua ori faptul ca x ar trebui sa satisfaca problema 

\begin{displaymath}
  \left\{\begin{matrix}
{x}'' \left ( t \right ) + \pi ^{2}x\left ( t \right )  = 0, & t \in \left [ 0,1 \right ] \\ 
x\left ( 0 \right )  =  0, x \left ( 1 \right ) = 1.& 
\end{matrix}\right.
\end{displaymath}

Pe de alta parte, se vede usor ca de fapt aceasta problema nu are nicio solutie. Prin urmare Ecuatia (1.2.17) nu are solutie. Merita totusi subliniat, faptul ca, in baza ipotezelor de mai sus, Ecuatia (1.2.16) are o solutie unica in \(C\left [ a,b \right ]\) ori de cate ori sup-norma lui \(\left | k \right |\) este sufient de mica, mai precis, daca \(\left ( b-a \right )sup_{\left [ a,b \right ]\times \left [ a,b \right ]}\left | k \right | < 1.\) Acest rezultat urmeaza cu usurinta Principiul Contractiei Banach. De fapt, problema existentei poate fi discutata in spetiul \(L^{2} \left ( a,b ;\mathbb{K} \right ),\) care este n cadru mai larg. Mai exact, sa presupunem \(f\in L^{2}\left ( a,b;\mathbb{K} \right ), k\in L^{2}\left ( Q; \mathbb{K} \right ),\) unde \(Q = \left ( a,b \right )\times \left ( a,b \right ).\) 

Solutia x a Ecuatiei (1.2.16) va fi cautata in \(L_{2}\left ( a,b;\mathbb{K} \right )\) care este un spatiu Hilbert in raport cu produsul scalar obisnuit si norma, 
\begin{displaymath}
  \left \langle g_{1}, g_{2} \right \rangle_{L^{2}} = \int_{a}^{b}g_{1}\left ( t \right ) \cdot \overline{{g_{2\left ( t \right )}}}dt, \left \| g \right \|_{L^{2}}^{2} = \left \langle g,g \right \rangle.
\end{displaymath}
	Desigur, daca gasim o solutie \(x\in L^{2}\left ( a,b;\mathbb{K} \right )\) a ecuatiei (1.2.16) cu \(f\in C\left ( \left [ a,b \right ];\mathbb{K} \right ), k\in C\left ( \left [ a,b \right ]\times \left [ a,b \right ];\mathbb{K} \right ),\) atunci evident \(x\in C\left ( \left [ a,b \right ];\mathbb{K} \right ).\) Avem urmatorul rezultat. 

\subsection{Teorema }

Daca \(f\in L^{2}\left ( a,b;\mathbb{K} \right ), -\infty < a< b< +\infty , k\in L^{2}\left ( Q; \mathbb{K} \right )\) si \(\int \int _{Q}\left | k\left ( t,s \right ) \right |^{2}dtds< 1,\) unde \(Q = \left ( a,b \right )\times \left ( a,b \right )\) atunci exista o functie unica \(x\in L^{2} \left ( a,b;\mathbb{K} \right ) \)care satisfice ecuatia
 \begin{displaymath}
   x\left ( t \right ) = f\left ( t \right ) + \int_{a}^{b}k\left ( t,s \right )x\left ( s \right )ds, 
 \end{displaymath}
aproape peste tot in \(\left ( a,b  \right )\). 


%
%
%	18/02/21
%
%





\bibliographystyle{unsrt}
\setlength{\baselineskip}{\normalbaselineskip}
\setlength{\parskip}{0pt}
\bibliography{refs}
\end{document}